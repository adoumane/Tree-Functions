

\section{Appendix on first-order relabelling}~\label{sec:AppendixForat}

\subsection{From first-order queries to temporal logic}

In order to show that first-order relabelling are derivable, we use a decomposition result, which says that first-order queries on trees over a ranked alphabet $\rGamma$ can be decomposed into the temporal operators of a logic that we call $2$-$\mathsf{CTL}$. 
This result can be seen as a direct consequence of \cite{}. But since the framework of this theorem is slightly different from ours, we recall it here, and show how our result can be derived from it.

Let $\rGamma$ be a ranked alphabet. The formulas of 2-$\mathsf{CTL}$ are generated by the following syntax:
\begin{align*}
\varphi, \psi:= \ a\in \rGamma \ | \ \odot_i \phi \ | \ \varphi\untilmod \psi \ | \ \varphi\sincemod\psi  
\end{align*}
The semantics of these formulas are defined for the ranked trees of $\trees \rGamma$ as follows.
\begin{enumerate}
\item A tree of $\trees \rGamma$ satisfies the formula $a\in\rGamma$ at the node $x$ if the label of $x$ is $a$.
\item $t, x\models \odot_i \varphi$ if the $i$-th child of $x$ satisfies $\varphi$.
\item  The formula $\varphi\untilmod \psi$  is valid  in the node $x$ if $x$ has a descendant $y$ with label in $\rDelta$, such that all nodes between $x$ and $y$ have label in $\rGamma$. 
\item $x$ has a descendant $y$ with label in $\rDelta$, such tha all nodes between $x$ and $y$ have label in $\rGamma$.
\end{enumerate}

\begin{lemma}%[Decomposition lemma]
For every first-ordre formula $\phi$, there is a $2$-$\mathsf{CTL}$ formula $T$ such that
$$ \forall t\in \trees \rGamma \qquad t,x\models \varphi \Leftrightarrow t,x\models T.$$
\end{lemma}

Shclingloff considers node labelled, branching bounded and   unordered (there is no order between the children of a node) trees. If $\Gamma$ is an (unranked) alphabet, we denote by $\trees_b\Gamma$ the set of unordered trees, labelled from $\Gamma$ and whose branching is at most $b$.   


The formulas of $4$-$\mathsf{CTL}$ are generated by the following syntax:
\begin{align*}
\varphi, \psi:= \ a\in \Gamma \ | \ \nextmod_i \phi \ | \ \varphi\untilmod \psi \ | \ \varphi\sincemod\psi  
\end{align*}
The semantics of $4$-$\mathsf{CTL}$ formulas is as before. The connective $\nextmod_i$ have the following semantics:

\begin{theorem}\cite{}%[Decomposition lemma]
For every first-ordre formula $\phi$, there is a $4$-$\mathsf{CTL}$ formula $T$ such that
$$ \forall t\in \trees_b \Gamma \qquad t,x\models \varphi \Leftrightarrow t,x\models T.$$
\end{theorem} 
 Let us explain how Lemma~\ref{} can be derived from Theorem~\ref{}. First Comment plonger les arbres de $\trees \rGamma$ dans le framework de Schlingloff. Soit $b$ l'arite maximale des elements de $\rGamma$. On denote par $\overline{\rGamma}$ le (unranked) set underlying $\rGamma$. 
Il y a un plongement naturel 
$$ \overline{\bullet}: \trees\rGamma \to \trees_b\Gamma$$
 
\subsection{First-order relabelling are derivable}

\begin{lemma}\label{lem:nextmod}
  For every $\rSigma$ and  $i \in \set{1,2,\ldots}$, the characteristic function of the  unary query 
        \begin{align*}
            \underbrace{\child i (x).}_{\text{$x$ is an $i$-th child}}
        \end{align*}
        is derivable.
\end{lemma}
\begin{proof}
To show that $\chi_{\nextmod_i \Gamma}:\tmonad \Sigma\to \tmonad{(\Sigma_{\sf{Bool}})}$,  the characteristic function of $\nextmod_i \Gamma$, is a first order tree function, we start applying the sibling function $\sf{Sib}:\tmonad\Sigma\to \tmonad \Sigma^{\sf{Sib}}$. Consider the function $f$
\begin{align*}
f:\Sigma^{\sf{Sib}} &\to \Sigma_{\sf{Bool}}\\
(a,l)& \mapsto (a,1) \qquad\text{if } l[i]\in\Gamma, \\
& \mapsto (a,0) \qquad\text{otherwise.}   \end{align*}
The function $f$ is clearly first order since it is arity preserving with finite domain.    

We get $\chi_{\nextmod_i \Gamma}$ by liftinf $f$ to trees and composing it with $\sf{Sib}$. 
\end{proof}

\begin{lemma}\label{lem:untilmod}
For every finite $\rGamma, \rDelta \subseteq \rSigma$,  the characteristic function of the unary query
         \begin{align*}
              \underbrace{\exists y\ y > x \land \rDelta(y) \land  \forall z \ (x < z < y \Rightarrow \rGamma(z)).}_{\substack{\text{$x$ has a descendant $y$ with label in $\rDelta$, such that}\\ \text{all nodes strictly between $x$ and $y$ have label in $\rGamma$}}} 
              \end{align*}
is derivable.
\end{lemma}
\begin{proof}
Let $\chi$ be the characteristic function of $\Gamma \untilmod \Delta$.
We start by decomposing our tree into blocks, depending on whether their node labels are in $\Gamma\cup\Delta$ or not. We will show that the restriction of $\chi$ to $\tmonad (\Gamma\cup\Delta)$, denoted $\chi_1$; and the restriction of $\chi$ to $\tmonad (\Sigma\setminus(\Gamma\cup\Delta))$, denoted $\chi_2$, are first-order tree functions. The function $\chi$ is obtained by applying $\chi_1$ or $\chi_2$ to each block, depending on its type, then by applying a flattening.
This decomposition is relevant because the value of a node depends only on the node labels of its block. %(it is constant equal to $0$ in the case of $\Sigma\setminus(\Gamma\cup\Delta)$ blocks).

The function $\chi_2:\tmonad (\Sigma\setminus(\Gamma\cup\Delta))\to \tmonad(\Sigma_{\sf{Bool}})$ is clearly a first-order tree function, as it assigns uniformly $0$ to every node label. 

To see that $\chi_1:\tmonad (\Gamma\cup\Delta)\to \tmonad(\Sigma_{\sf{Bool}})$ 
is a first-order function, we start by applying the function $\mathsf{Desc}_\Delta$ from
Example~\ref{}, which adds $1$ to the node labels if they have a descendent in $\Delta$ and adds $0$ otherwise. Then we decompose the obtained tree into blocks, depending on whether their node labels contain $0$ or not. For the block of nodes having $0$ in their label, their value w.r.t. $\Gamma\untilmod \Delta$ is indeed $0$, since they do not have a descendent in $\Delta$. To these blocks we apply the identity function. For the blocks that received $1$, we assign $1$ to the nodes whose labels are in  $\Gamma$ and $0$ to the others (whose labels are in $\Delta\setminus\Gamma$). Indeed, inside these blocks, a node $n$ labeled from $\Gamma$ has by construction a descendent in $\Delta$, moreover the path connecting $n$ to such descendent lies in the same block as $n$. If we consider the nearest such descendent $m$, the path between $n$ and $m$  is labeled in $\Gamma$ (since we are in blocks having only $\Gamma\cup\Delta$ labels). Thus $n$ satisfies $\Gamma\untilmod\Delta$. Nodes labeled in $\Delta\setminus\Gamma$ have obviously value $0$.      
\end{proof}


\begin{lemma}\label{lem:sincemod}
For every finite ranked sets $\Gamma, \Delta \subseteq \Sigma$  the characteristic function of $\Gamma \sincemod \Delta$ is a first-order tree function.
\end{lemma}
\begin{proof}
The same proof as above, one only needs to replace the use of $\mathsf{Desc}_\Delta$ by that of $\mathsf{Anc}_\Delta$.
\end{proof}
