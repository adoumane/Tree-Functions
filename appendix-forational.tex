\section{Appendix on first-order relabelling}~\label{sec:AppendixForat}

The goal of this section is to show Proposition~\ref{prop:forat}, which says that first-order relabeling are derivable. As discussed in the body of the paper, the proof of this proposition is based on an equivalence result between first-order queries on trees  and a temporal logic, as stated in Lemma~\ref{lem:schlingloff}. While this result is deaply inspired from a similar result of Schlingloff \cite{schlingloff1992expressive}, our frameworks are not exactly the same (he uses for ainstance unranked trees).  
In the rest of this section, we provide more details about the reduction from Schilgloff's result to our lemma (Section~\ref{sec:reduction-schilingloff}). Then we show in Section~\ref{sec:relabeling}  how to use it in order to prove Proposition~\ref{prop:forat}.  




\subsection{Reduction to Schilgloff's theorem}\label{sec:reduction-schilingloff}

Let us proceed to the proof of Lemma~\ref{lem:schlingloff}.
Clearly the functions in the lemma are first-order tree relabeling, and first-order tree relabeling are easily seen to be closed under composition, which gives the right-to-left inclusion in the lemma. The hard part is the left-to-right inclusion, which says that every first-order tree relabeling can be decomposed into functions as in items~\ref{it:relabeling} -- \ref{it:since}. 
The first  step in the proof of  the right-to-left inclusion is the observation that  every first-order tree relabeling can be decomposed as 
\begin{align*}
    g \circ f_1 \circ \cdots \circ f_n
\end{align*}
where $g$ is a relabeling as in item~\ref{it:relabeling} of the lemma and each $f_i$ is a  characteristic function of some unary query (not necessarily of the simple form indicated in items~\ref{it:child} -- \ref{it:since} in the lemma). This is a simple observation: the functions $f_1,\ldots,f_n$ annotate the tree with the truth values of the unary queries used in the definition of the  first-order relabeling, and $g$ uses these truth values to select the appropriate output label. The hard part of the lemma is showing that each $f_i$  can be further decomposed into functions as indicated in the lemma. This is where we us the result of Schlingloff~\cite[Theorem 2.6]{schlingloff1992expressive}, which says that all first-order definable tree properties can be defined using a temporal logic that has operators similar to the ones used in items~\ref{it:child} -- \ref{it:since} of the lemma. 

The following table summarizes our framework (first column) and Schlingloff's one (second column). The first row describes the models under consideration, the second row the corresponding version of first-order logic, and the third row the corresponding temporal logic. 
\begin{center}
\begin{tabular}{|C{1.8cm}|C{2.9cm}|C{2.8cm}|}
\cline{1-3}
\textbf{Models} & \textbf{Trees over a finite ranked alphabet $\rGamma$}: finitely branching, ranked trees, labeled from $\rGamma$. & \textbf{Models over a set of propositions $P$}: finitely branching, unranked trees, labeled from $2^P$.\\
\hline
    \multirow{3}{1.8cm}{\begin{center}\textbf{First-order logic}\end{center}}&\multicolumn{2}{C{5.6cm}|}{Usual first-order connectives ($\exists, \vee, \neg$) with the descendant predicate $x\leq y$ and the following predicates:}\\
    \cline{2-3} 
                                 & $a(x)$:  $x$ is labeled $a$  $(a \in \rGamma)$.  & $p(x)$: label of $x$ contains $p$
                                  ($p\in P$). \\
                                  &\hspace{-.14cm}$\child i(x)$: $x$ is an $i$-th child. &\\
   &       \textbf{We call it $\rGamma$-FO.}    &  \textbf{We call it $P$-FO. }            \\
    \hline
    \multirow{4}{1.8cm}{\begin{center}
    \textbf{Temporal logic}
    \end{center}}& \multicolumn{2}{C{5.6cm}|}{Usual CTL connectives ($S$ (Since), $U$ (Until), $\vee$, $\neg$) together with:}\\
    \cline{2-3} 
    
     & $a \in \rGamma$, & $p \in P$, \\
    & $\odot_i \phi$: the $i$-th child satisfies $\phi$. & $X_i \phi$: at least $i$ children satisfy $\phi$.\\ 
    &       \textbf{ We call it 2-CTL. }   &   \textbf{We call it 4-CTL.            } \\
    \hline
\end{tabular}
\end{center}
What is named $\rGamma$-FO in the table is what we simply called first-order logic along the paper. The operators of 2-CTL are those of Lemma~\ref{lem:schlingloff}.  Using the notation of the table, 
Schlingloff's theorem says that P-FO formulas are equivalent to 4-CTL formulas, and  Lemma~\ref{lem:schlingloff} states that $\rGamma$-FO formulas are equivalent to 2-CTL ones. To deduce the later from the former, we will show how to translate every ranked tree $t$ over $\rGamma$ into a model $[t]$ over a well chosen set of propositions $P$, then we will apply the following scheme
    $$\xymatrix@C=2.5cm{
        \varphi \in \rGamma\text{-FO} 
        \ar@{<->}[r]^{\forall t, \ t\models \varphi \ \leftrightarrow \ [t] \models \psi}_{\text{Lemma}~\ref{lem:from-Gamma-to-P-FO}}
        &
        \psi \in P \text{-FO}
        \ar@{<->}[d]^{\substack{\text{Schlingloff's}\\\text{theorem}}} \\
       \theta \in \text{2-CTL}
        \ar@{<->}[r]^{\forall t, \ t\models \theta \ \leftrightarrow \ [t] \models \delta}_{\text{Lemma}~\ref{lem:from-4-CTL-to-2CTL}}
        &
        \delta \in \text{4-CTL}
    }$$    
The translation $[\_]$ and Lemmas~\ref{lem:from-Gamma-to-P-FO} and \ref{lem:from-4-CTL-to-2CTL} are explained below.

\paragraph{From ranked trees to Schlingloff's models.} Let us fix a ranked alphabet $\rGamma$. Let $P$ be the following set of propositions
\begin{align*}
P\eqdef \rGamma\cup\set{i\text{-th-child}\ |\ i\in[1, \text{max arity of }\rGamma]}
\end{align*}
 Let $t$ be a ranked tree over $\rGamma$. The translation $[t]$ of $t$ is the model defined as follows. It has the same set of nodes and the same descendant relation as $t$. The label of a node contains $a$ if its label in $t$ is $a$. It contains the proposition $i\text{-th-child}$ if it is an $i$-th child in $t$.

\paragraph{From $\rGamma$-FO to $P$-FO.} Let $\rGamma$ and $P$ be as above. We show that
\begin{lemma}\label{lem:from-Gamma-to-P-FO}
For every $\rGamma$-FO formula $\phi$, there is a $P$-FO formula $\psi$ such that
$$ \forall t\in \trees\rGamma, \qquad t\models \phi \leftrightarrow [t]\models \psi$$
and conversely.
\end{lemma}
\begin{proof}
To show this lemma its is enough to show how to translate the specific predicates of each formalism into the other. The predicate $a(x)$ of $\rGamma$-FO can be translated by the same predicate in P-FO and conversely. The predicate $\child i(x)$ can be translated by $i\text{-th-child}(x)$ and conversely.  It is clear that these translations preserve the semantics.
\end{proof}

\paragraph{From 4-CTL to 2-CTL}
Let $\rGamma$ and $P$ be as above. We show that
\begin{lemma}\label{lem:from-4-CTL-to-2CTL}
For every $\rGamma$-FO formula $\phi$, there is a $P$-FO formula $\psi$ such that
$$ \forall t\in \trees\rGamma, \qquad t\models \phi \leftrightarrow [t]\models \psi$$
and conversely. \end{lemma}
\begin{proof}
Here again, it is enough to translate the specific connectives of each formalism into the other. The connective $X_i$ can be encoded in 2-CTL as follows, where $b$ is the maximal arity of $\rGamma$
\begin{align*}
\underset{\begin{array}{c}
{\scriptstyle I\subseteq [1,b]}\\ {\scriptstyle\#I=i}
\end{array}}{\bigvee} \underset{j\in I}{\bigwedge} \odot_j \phi
\end{align*}
Conversely, the connective $\odot_i$ can be encoded in 4-CTL as follows:
\begin{align*}
X_1(i\text{-th-child}\wedge \phi)
\end{align*}
\end{proof}
%Let $\rGamma$ be a ranked alphabet. The formulas of 2-$\mathsf{CTL}$ are generated by the following syntax:
%\begin{align*}
%\varphi, \psi:= \ a\in \rGamma \ | \ \odot_i \phi \ | \ \varphi\untilmod \psi \ | \ \varphi\sincemod\psi  
%\end{align*}
%The semantics of these formulas are defined for the ranked trees of $\trees \rGamma$ as follows.
%\begin{enumerate}
%\item A tree of $\trees \rGamma$ satisfies the formula $a\in\rGamma$ at the node $x$ if the label of $x$ is $a$.
%\item $t, x\models \odot_i \varphi$ if the $i$-th child of $x$ satisfies $\varphi$.
%\item  The formula $\varphi\untilmod \psi$  is valid  in the node $x$ if $x$ has a descendant $y$ with label in $\rDelta$, such that all nodes between $x$ and $y$ have label in $\rGamma$. 
%\item $x$ has a descendant $y$ with label in $\rDelta$, such tha all nodes between $x$ and $y$ have label in $\rGamma$.
%\end{enumerate}
%
%\begin{lemma}%[Decomposition lemma]
%For every first-ordre formula $\phi$, there is a $2$-$\mathsf{CTL}$ formula $T$ such that
%$$ \forall t\in \trees \rGamma \qquad t,x\models \varphi \Leftrightarrow t,x\models T.$$
%\end{lemma}
%
%Shclingloff considers node labeled, branching bounded and   unordered (there is no order between the children of a node) trees. If $\Gamma$ is an (unranked) alphabet, we denote by $\trees_b\Gamma$ the set of unordered trees, labelled from $\Gamma$ and whose branching is at most $b$.   
%
%
%The formulas of $4$-$\mathsf{CTL}$ are generated by the following syntax:
%\begin{align*}
%\varphi, \psi:= \ a\in \Gamma \ | \ \nextmod_i \phi \ | \ \varphi\untilmod \psi \ | \ \varphi\sincemod\psi  
%\end{align*}
%The semantics of $4$-$\mathsf{CTL}$ formulas is as before. The connective $\nextmod_i$ have the following semantics:
%%
%%\begin{theorem}\cite{}%[Decomposition lemma]
%%For every first-ordre formula $\phi$, there is a $4$-$\mathsf{CTL}$ formula $T$ such that
%%$$ \forall t\in \trees_b \Gamma \qquad t,x\models \varphi \Leftrightarrow t,x\models T.$$
%%\end{theorem} 
%% Let us explain how Lemma~\ref{} can be derived from Theorem~\ref{}. First Comment plonger les arbres de $\trees \rGamma$ dans le framework de Schlingloff. Soit $b$ l'arite maximale des elements de $\rGamma$. On denote par $\overline{\rGamma}$ le (unranked) set underlying $\rGamma$. 
%%Il y a un plongement naturel 
%%$$ \overline{\bullet}: \trees\rGamma \to \trees_b\Gamma$$
%
%\paragraph*{Encoding our trees by Schlingloff's trees}
%\paragraph*{Translating first-order formulas}
%\begin{lemma} Let $\rGamma$ be a ranked set and $\phi$ be a first-order formula.
%\begin{align*}
%\forall t\in \trees \rGamma, \qquad t\models \phi \leftrightarrow t \models [\phi]
%\end{align*} 
%\end{lemma}
%
%\paragraph*{From 4-CLT to 2-CTL}
% \begin{lemma} Let $\rGamma$ be a ranked set and $\phi$ be a first-order formula.
%\begin{align*}
%\forall t\in \trees \rGamma, \qquad t\models \overline{\phi} \leftrightarrow [t] \models \phi
%\end{align*} 
%\end{lemma}
%
\subsection{First-order relabelling are derivable}\label{sec:relabeling}
To show Proposition~\ref{prop:forat}, saying that first-order relabelling are derivable, we will show that each function appearing in Lemma~\ref{lem:schlingloff} and corresponding to each operator of 2-CTL is derivable. This is the role of Lemmas\ref{lem:nextmod}--\ref{lem:sincemod} presented below.
\begin{lemma}\label{lem:nextmod}
  For every finite $\rSigma$, $\rGamma\subseteq \rSigma$ and $i \in \set{1,2,\ldots}$, the characteristic function $\ranked{f:\tmonad \rSigma\to\tmonad(\rSigma+\rSigma)}$ of the  unary query 
        \begin{align*}
\text{``The }i\text{-th child of }x\text{ is in }\rGamma\text{''}
        \end{align*}
        is derivable.
\end{lemma}
\begin{proof}
To show that $\ranked{f}$ is derivable, we start applying the children function \begin{align*}
\ranked{\mathsf{Children}:\tmonad\rSigma\to \tmonad (\rSigma\product (\rSigma_0+0)^*)}
\end{align*} from Example~\ref{ex:sibling} which tags every nodes by the list of its children. Consider the function $g$
$$\begin{array}{rlll}
\ranked{g:}  \ranked{\rSigma\product (\rSigma_0+0)^*} &\ranked{\to}& \ranked{\Sigma +\Sigma}&\\
            \tensorpair{a,l}                           &
     \mapsto& \tensorpair{a,1}              &\text{if } l[i]\in\ranked{\rGamma_0}, \\
                    &   \mapsto& \tensorpair{a,2} &\text{otherwise.}   
\end{array}$$
which maps an element of $\rSigma$ tagged by a list to the first copy of $\rSigma$ if the $i$-th element of the list is in $\rGamma$ and to the second copy otherwise. The function $g$ is derivable since its domain is finite.
We finally get $\ranked{f}$ by lifting $\ranked{g}$ to terms.
\end{proof}

\medskip
\begin{lemma}\label{lem:untilmod}
For every finite $\rGamma, \rDelta \subseteq \rSigma$,  the characteristic function $\ranked{f:\tmonad\rSigma\to\tmonad(\rSigma+\rSigma)}$ of the unary query
         \begin{align*}
              \underbrace{\exists y\ y \geq x \land \rDelta(y) \land  \forall z \ (x < z < y \Rightarrow \rGamma(z)).}_{\substack{\text{$x$ has a descendant $y$ with label in $\rDelta$, such that}\\ \text{all nodes between $x$ and $y$ have label in $\rGamma$}}} 
              \end{align*}
is derivable.
\end{lemma}
\begin{proof}
We start by applying the factorization
\begin{align*}
\ranked{\ancfact : \tmonad \Sigma \to \tmonad(\tmonad(\Sigma\setminus(\Gamma\cup\Delta)) +\tmonad(\Gamma\cup\Delta))}
\end{align*}
which decomposes our terms into factors, depending on whether their node labels are in $\ranked{\Gamma\cup\Delta}$ or not. 
Note that the value of a node w.r.t. the until query depends only on the node labels of its factor. %(it is constant equal to $0$ in the case of $\Sigma\setminus(\Gamma\cup\Delta)$ blocks).

The nodes of the $\ranked{\tmonad(\rSigma\setminus(\rGamma\cup\rDelta))}$ factors do not satisfy the query, thus we will apply to them the function $\ranked{\tmonad g}$ obtained by lifting the function 
$$\begin{array}{rrll}
 \ranked{g:}& \ranked{\rSigma\setminus(\rGamma\cup\rDelta)}& \ranked{\to} &\ranked{\rSigma+\rSigma}\\
&a&\mapsto& \tensorpair{a,2}.
\end{array}$$
 
  
Nodes of the $\ranked{\tmonad(\Gamma\cup\Delta)}$  factors satisfy the query if and only if they have a descendant in  $\rDelta$.
Consider the function $\ranked{h}$ obtained by composing the descendant function $\ranked{\mathsf{Descendant}_\Delta}$ from Example~\ref{ex:descendant} with an injection $\ranked{\tmonad(\iota+\iota)}$
\begin{align*}
\underbrace{\ranked{\tmonad(\Gamma\cup\Delta) \xrightarrow{\ranked{\mathsf{Descendant}_\Delta}} \tmonad(\Gamma\cup\Delta+\Gamma\cup\Delta) \xrightarrow{\ranked{\tmonad(\iota+\iota)}}\tmonad(\Sigma+\Sigma)}}_{\ranked{h}}
\end{align*}
Finally, to get the characteristic function $\ranked{f}$, we apply $\ranked{\tmonad{g}}$ to the $\ranked{\tmonad(\Sigma\setminus(\Gamma\cup\Delta))}$ factors and $\ranked{h}$ to the other factors using the co-pairing combinator, then we flat the obtained term. 
\end{proof}


\begin{lemma}\label{lem:sincemod}
For every finite $\rGamma, \rDelta \subseteq \rSigma$,    the characteristic function of the unary query
         \begin{align*}
              \underbrace{\exists y\ y \leq x \land \rDelta(y) \land  \forall z \ (y < z < x \Rightarrow \rGamma(z)).}_{\substack{\text{$x$ has a descendant $y$ with label in $\rDelta$, such that}\\ \text{all nodes strictly between $x$ and $y$ have label in $\rGamma$}}}  
         \end{align*} 
         is derivable.
\end{lemma}
\begin{proof}
The same proof as above, one only needs to replace the use of the function $\ranked{\mathsf{Descendant}_\Delta}$ by that of $\ranked{\mathsf{Ancestor}_\Delta}$, introduced in Example~\ref{ex:descendant}.
\end{proof}
