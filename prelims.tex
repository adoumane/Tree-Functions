
\section{First-order tree functions}
The goal of this paper is to describe the first-order tree-to-tree functions as the smallest class which contains certain atomic functions, and which is closed under applying certain combinators. To achieve this goal, we discuss functions that transform not just trees, but also \emph{terms}, which are trees with distinguished ports that can be substituted with trees (or other terms). One of the main points of using terms is that  they have a substitution structure, i.e.~one can build a terms of terms. 

\paragraph*{Types.} In the end we are interseted about tree-to-tree functions. However, to construct such functions, we will use intermediate operations which transform terms,  pairs of terms, or terms of terms, etc.
 The following definition describes the types that can be used by the intermediate operations.



\begin{definition}[Types]\label{def:types}
    Define a \emph{type} to be any ranked set which can be obtained from the atomic types by applying the type constructors, as described below. We denote by $\Tt$ the set of all types. 
    \begin{itemize}
        \item {\bf Atomic types.} 
           \begin{enumerate}
            \item every ranked set with finitely many elements;
            \item a ranked set, call it  $\llone$, which has one element on every arity;
        \end{enumerate}
    \item {\bf Type constructors.} 
     \begin{enumerate}
        \item {\bf Disjoint union $\Sigma+\Gamma$.} An element of $\Sigma+\Gamma$ is either an element of $\Sigma$ or an element of $\Gamma$, with the arities inherited from $\Sigma,\Gamma$. 
        \item {\bf Cartesian\footnote{This is the Cartesian product in the category where objects are ranked sets, and morphisms are arity preserving functions.}  product $\Sigma \times \Gamma$.} An  $n$-ary element of $\Sigma \times \Gamma$ is a pair of elements from $\Sigma, \Gamma$, respectively, whose arities are both $n$.
        \item {\bf Tensor product $\Sigma \otimes \Gamma$.} An  $n$-ary element of $\Sigma \times \Gamma$ is a pair of elements from $\Sigma, \Gamma$, respectively, whose arities  sum up to $n$.
        \item {\bf Terms.} For a ranked set  $\Sigma$, define a \emph{term}\footnote{The terms that we use are special kinds of terms, where each variable is used exactly once.} to be tree over the alphabet $\Sigma + \set \portletter$, where $\portletter$ is a fresh symbol of arity zero. Define $\trees \Sigma$ to be the ranked set of terms over $\Sigma$, where the arity of a term is the number of labels $\portletter$.
    \end{enumerate}  
    \end{itemize}
  
\end{definition}

\newcommand{\tensorpair}[2]{\langle #1, #2 \rangle}

The tensor constructor can be seen as a special case of the term constructor, since a tensor pair $\tensorpair a b$  can be viewed as a tree of depth two like in this picture:
\begin{center}
    (picture)
\end{center}
In fact, the term constructor can be viewed as a least fixpoint of tensor products, as expressed in the following isomorphism:
\begin{align*}
    \trees \Sigma \quad \simeq \quad  \set \portletter + \coprod_{a \in \Sigma} \underbrace{\trees \Sigma \otimes \cdots \otimes \trees \Sigma}_{\text{arity of $a$ times}}
\end{align*}

\begin{example}[The type $n$, $n\in\mathbb{N}$]
The ranked set whose underlying set is $\{0,\dots,n\}$ and the arity of each symbol is $0$ is a type. We denote it by $0$. 
\end{example}

\begin{example}[The type $\llzero$]
$\llzero=\llone\otimes 0$
\end{example}

\paragraph*{Atomic functions.} Having described the types, we now describe the atomic functions that transform elements of these types.




\begin{definition}[Atomic functions]\label{def:atomic-functions} The \emph{atomic functions} are the following functions (for every types $\Sigma,\Gamma,\Sigma_0,\Sigma_1$): 
\begin{enumerate}
    \item The unique arity-preserving function $! : \Sigma \to \llone$
    \item Every arity-preserving function when the domain $\Sigma$ is finite:
    \begin{align*}
        f : \Sigma \to \Gamma
    \end{align*}
    \item Projection and co-projection:
    \begin{align*}
        \pi_i : \Sigma_0 \times \Sigma_1 \to \Sigma_i \qquad \iota_i : \Sigma_i \to \Sigma_0 + \Sigma_1
    \end{align*}
    \item Distribution of $+$ across $\times$:
\begin{align*}
    (\Sigma_0 + \Sigma_1 ) \times \Gamma \to (\Sigma_0 \times \Gamma) + (\Sigma_1 \times \Gamma)
\end{align*}
\item Distribution of $+$ across $\otimes$:
\begin{align*}
    (\Sigma_0 + \Sigma_1 ) \otimes \Gamma \to (\Sigma_0 \otimes \Gamma) + (\Sigma_1 \otimes \Gamma)
\end{align*}
\item Assuming that $\Sigma$ is finite, the two canonical bijections (one in each direction) between the ranked sets:
 \begin{align*}
    \trees \Sigma \qquad \text{and} \qquad    \set \portletter + \coprod_{a \in \Sigma} \underbrace{\trees \Sigma \otimes \cdots \otimes \trees \Sigma}_{\text{arity of $a$ times}}
\end{align*}
\item Flattening:
\begin{align*}
    \trees \trees \Sigma \to \trees \Sigma
\end{align*}
\item Blocking
    \begin{align*}
        \trees (\Sigma + \Gamma) \to \trees (\trees \Sigma + \trees \Gamma)
    \end{align*}
    \item Comb
    \begin{align*}
        \trees \llone \to \trees \llone
    \end{align*}
    \item Unfolding: 
\begin{align*}
    \trees (\Sigma \otimes \Gamma) \to (\trees \Sigma) \otimes (\trees \Gamma)
\end{align*}
    \item Unit:
\begin{align*}
    \Sigma  \to \trees \Sigma
\end{align*}

\item Some kind of port swapping
\end{enumerate}
\end{definition}



The atomic functions can be combined into more complex functions using the following combinators.
\begin{definition}
     [Combinators] \label{def:combinators} \ 
    \begin{enumerate}
    \item Function composition:
    \begin{align*}
    \frac{f : \Sigma \to \Gamma \quad g : \Gamma \to \Delta} {g \circ f : \Sigma \to \Delta}
\end{align*}

\item Lifting functions to trees:
\begin{align*}
    \frac{f : \Sigma \to \Gamma} {\trees f : \trees \Sigma \to \trees \Gamma}
\end{align*}
\item Cases
\begin{align*}
    \frac{f_0 : \Sigma_0 \to \Gamma \quad f_1 : \Sigma_1 \to \Gamma} {[f_0,f_1] : \Sigma_0 + \Sigma_1 \to \Gamma}
\end{align*}

\item Pairing functions:
\begin{align*}
    \frac{f_0 : \Sigma \to \Gamma_0 \quad f_1 : \Sigma \to \Gamma_1} {(f_0,f_1) : \Sigma \to \Gamma_0 \times \Gamma_1}
\end{align*}

\item Tensor product of functions:
\begin{align*}
    \frac{f_0 : \Sigma_0 \to \Gamma_0 \quad f_1 : \Sigma_1 \to \Gamma_1} {\langle f_0,f_1 \rangle : \Sigma_0 \otimes \Sigma_1 \to \Gamma_0 \otimes \Gamma_1}
\end{align*}
\end{enumerate}
\end{definition}

We are now ready to define the first-order tree functions, which are the topic of this paper. 

\begin{definition}[First-order tree functions] \label{def:fo-tree-functions} \ 
    \begin{itemize}
        \item A \emph{first-order term functions} is any function that can be  generated from the atomic functions from Definition~\ref{def:atomic-functions} by applying the  combinators from Definition~\ref{def:combinators}.
        \item  A \emph{first-order tree function} is any first-order term function restricted to trees, i.e.~terms of arity zero.
    \end{itemize}    
\end{definition}
%\begin{remark}
Note that first-order term functions are arity preserving. 
%\end{remark}

To illustrate these first-order tree functions, we present a series of examples, many of which will be useful later.

\bigskip
\noindent \begin{example}[Identity] For every type $\Sigma$, the identity function $x\in\Sigma\mapsto x\in\Sigma$ is a first-order tree function. This is achieved by induction on the types: the identity function over a finite set of elements is first-order since its domain is finite, and the identity function over $\llone$ is the first-order function $!$. For $\Sigma$ and $\Gamma$ in $\Tt$ , the identity
function over $\Sigma+\Gamma$ is the disjoint union of the co-projections $\Sigma \to \Sigma + \Gamma$ and $\Gamma \to \Sigma + \Gamma$. The
identity function over $\Sigma \times \Gamma$ is the pairing of the projections $\Sigma \times \Gamma \to \Sigma$ and $\Sigma \times \Gamma \to \Gamma$. The
identity function over $\Sigma \otimes \Gamma$ is the tensor product of the identity over $\Sigma$ and the identity over $\Gamma$. Finally,
the identity function over $\trees\Sigma$ is constructed from the identity function over $\Sigma$ using the combinator $\trees$.
\end{example}


\bigskip
\noindent\begin{example}[Filter] Let $\Gamma, \Sigma\in\Tt$, where $\Gamma$ is a finite set of unary symbols. Consider the function:
$$ f:\trees (\Sigma+\Gamma)\to\trees \Sigma$$
which erases the elements of $\Gamma$ from the inupt tree. This function is well defined since erasing unary symbols does not break the tree structure of the input. 
Let us explain why this is a first-order list function. 
Consider the function $unit:\Sigma\to\trees\Sigma$ and the constant function $Empty:\Gamma\to\trees\Sigma$ which associates to every element of $\Gamma$ the empty tree. The function $Empty$ is a first order tree function since its domain is finite. Using the cases combinator, we get a tree in $\trees\trees\Sigma$, which we transform into a tree in $\trees \Sigma$ using flattening.
\end{example}

\bigskip
\noindent \begin{example}[If then else] Suppose that $f : \Sigma \to \llzero+\llone$ and $g_0, g_1 : \Sigma \to\Gamma$ are first-order tree functions. Then the function:
 \begin{align*}
  g\colon \Sigma &\to \Gamma \\
  x &\mapsto g_0(x)  \text{ if } f(x)\in\llzero\\
  x &\mapsto g_1(x)  \text{ if } f(x)\in\llone.
\end{align*}
 is also a first-order tree function. This is done as follows. On input
$x\in\Sigma$, we first apply the pairing of f and the identity function, yielding a result:
$$ (f(x),x)\in(\llzero+\llone)\times\Sigma$$
Next we apply the function distribute, transforming the type into:
$$ \llzero\times\Sigma+\llone\times\Sigma$$
To this result we apply the disjoint union $h_0 + h_1$ where $h_0:\llzero\times\Sigma\to \Gamma$ and
$h_01:\llone\times\Sigma\to \Gamma$ are defined by $h_i=\pi_2(id_i,g_i)$, where $id_0$ and $id_1$ are the identity function on $\llzero$ and $\llone$ respectively, yielding the desired result.
\end{example}

\bigskip
\noindent \begin{example}[Pattern matching]
\end{example}

\bigskip
\noindent \begin{example}[Characteristic function of a finite type]
Let $\Sigma, \Delta\in \Tt$ and suppose that $\Sigma$ is finite. 
The order-preserving function $\chi_\Sigma:\Delta\to \llzero+\llone$ which satisfies $\chi_\Sigma(s)\in \llone$ if and only if $s\in \Sigma$ is a first-order tree function. 
\end{example}

\bigskip
\noindent \begin{example}[Siblings and parent information]
Let $\Gamma$ be a finite type, and $n$ the maximal arity of its symbols. We set $(\Gamma\cup\bot)^{\leq n}$ to be the ranked set of \emph{nullary} symbols, whose elements are lists of lenght at most $n$ of $\Gamma$ symbols together with a symbol $\bot$.

\medskip
Consider the function:
$$ Sib: \trees \Gamma \to \trees (\Gamma\otimes (\Gamma\cup\bot)^{\leq n})$$
which adds to every node of a tree in $\trees \Sigma$ the list of its siblings (the missing siblings are witnessed by $\bot$ in the list).

% from  $(\Gamma\cup\bot)^{\leq n}$, whose lenght is the arity of the node, and such that the $i^{th}$ element of the  list is the symbol of its $i^{th}$ sibling if it has one, otherwise it is $\bot$.
Let us explain why $Sib$ is a first-order tree function. 
\begin{enumerate}
\item For every symbol $a\in \Gamma$, consider the unary symbols $a_\#$ and $a_\flat$.
We set $\Gamma_\#=\{a_\#\mid a\in\Gamma\}$ and $\Gamma_\flat=\{a_\flat\mid a\in\Gamma\}$.
Let $g$ be the following function:
 \begin{align*}
  g\colon \Gamma &\to \trees(\Gamma+\Gamma_\#+\Gamma_\flat)\\
  a &\mapsto a_\#[a[a_\flat,\dots,a_\flat]].
\end{align*}
The action of $g$ on a $\Gamma$ symbol looks like this:
\begin{center}
Picture
\end{center}
\item We lift $g$ to the trees of $\trees \Gamma$ using the combinator $\trees$, then apply a flattenig. After this operation, a tree of $\trees\Gamma$ is transformed this way:
\begin{center}
Picture
\end{center}
\item We apply a block, to separate the symbols of $\Gamma$ form the others, we get then a tree in $\trees(\trees\Gamma+\trees(\Gamma_\#+\Gamma_\flat))$. Those trees look like this:
\begin{center}
Picture
\end{center}
\item Consider the function $h:\trees(\Gamma_\#+\Gamma_\flat)\to \trees(\Gamma_\#+\Gamma_\flat)$ which transforms the trees of the following shape as follows:
\begin{center}
Picture
\end{center}
and leaves the others unchanged. 
The function $h$ is a first-order tree function because \todo{pattern matching on a finite set}.

\item To the trees of $\trees(\trees\Gamma+\trees(\Gamma_\#+\Gamma_\flat))$ we apply the identity to the blocks $\trees \Gamma$ and $h$ to the others, then we apply a flettening to get again trees of the form:
\begin{center}
Picture
\end{center} 
\item We apply again a block, to separate the symbols of $\Gamma_\flat$ form the others. We get blocks where each $\Gamma$ symbol is in the same block with its siblings (marked with $\#$):
\begin{center}
Picture
\end{center}
\item Consider the function $k:\trees(\Gamma_\#+\Gamma_\flat)\to (\Gamma\cup\bot)^{\leq n}$ which transforms every tree of the following shape as follows:
\begin{center}
Picture
\end{center}
and leaves the other unchanged. We apply the function $k$ to the $\trees(\Gamma_\#+\Gamma_\flat)$
blocks and the identity to the others. The we apply a flattening and erase the symbols from $\Gamma_\#$. Doing so we get the desired function.
\end{enumerate}
\medskip

Now consider the function $$Par:\trees\Gamma\to \trees (\Gamma\otimes \Gamma^{\leq 1})$$ which adds to every node of a tree in $\trees \Sigma$ the list (of lenght at most 1) containing the symbol of its parent.
\end{example}



\bigskip
\noindent  \begin{example}[Root and leaves]
\begin{itemize}
\item The combinator which applies $f$ to the root and $g$ to the rest of the tree:
\begin{align*}
    \frac{f : \Sigma \to \Gamma_0 \quad g : \Sigma \to \Gamma_1}
     {\mathsf{root}_{f,g} : \trees\Sigma \to \trees(\Gamma_0 + \Gamma_1)}
\end{align*}
\item The combinator which applies $f$ to the leaves and $g$ to the rest of the tree:
\begin{align*}
    \frac{f : \Sigma \to \Gamma_0 \quad g : \Sigma \to \Gamma_1}
     {\mathsf{leaf}_{f,g} : \trees\Sigma \to \trees(\Gamma_0 + \Gamma_1)}
\end{align*}
\end{itemize}
\end{example}

\bigskip
\noindent\begin{example}[Descendent and ancestors]
\begin{itemize}
\item The function $\mathsf{Desc}_\Gamma: \trees \Sigma \to \trees (2\otimes\Sigma)$ which 
adds $1$ to the label of each node depending on whether it has a descendent in $\Gamma$.
\item The function $\mathsf{Anc}_\Gamma: \trees \Sigma \to \trees (2\otimes\Sigma)$ which 
adds $1$ to the label of each node depending on whether it has an ancestor in $\Gamma$.
\end{itemize}
\end{example}


\bigskip
\noindent\begin{example}[Depth-first traversal]
\end{example}

We are now ready to state the main result of this paper. 
\begin{theorem}\label{thm:main}
    The following classes of functions are equal\begin{itemize}
        \item First-order tree-to-tree transductions;
        \item First-order tree functions.
    \end{itemize}
\end{theorem}

The bottom-up inclusion, i.e.~the proof that every first-order tree function is a first-order tree-to-tree transduction, is proved by induction in Section~\ref{sec:to-transductions}.  The bottom-up inclusion is true by design, i.e.~the atomic functions and combinators were designed so that they could be easily  simulated using first-order transductions. 

The hard part of the theorem is showing that every first-order tree-to-tree transduction can be broken up, using the combinators, into the atomic functions. The proof strategy for the hard part is presented in Section~\ref{sec:strategy}, and is then realised in Sections~\ref{sec:fo-translation} and~\ref{sec:one-register}.