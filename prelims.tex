
\section{First-order tree functions}
The goal of this paper is to describe the first-order tree-to-tree transductions using a small number of atomic functions and certain closure properties. To describe these functions, we will use not just trees, but more complex types, e.g.~pairs of trees, trees of trees, etc. Also, instead of trees, we use terms, which are trees with ports where other trees can be inserted. 
The types that we used are described in the following definition. 


\begin{definition}[Type constructors]\label{def:type-constructors}
    Consider the following operations, which are used to define new ranked sets based on existing ones.
\begin{enumerate}
    \item {\bf Disjoint union $\Sigma+\Gamma$.} An element of $\Sigma+\Gamma$ is either an element of $\Sigma$ or an element of $\Gamma$, with the arities inherited from $\Sigma,\Gamma$. 
    \item {\bf Cartesian\footnote{This is the Cartesian product in the category where objects are ranked sets, and morphisms are arity preserving functions.}  product $\Sigma \times \Gamma$.} An arity $n$ element of $\Sigma \times \Gamma$ is a pair $(a,b)$ such $a \in \Sigma$ and $b \in \Gamma$ are elements whose arities are both $n$.
    \item {\bf Tensor product $\Sigma \otimes \Gamma$.} An arity $n$ element of $\Sigma \otimes \Gamma$ is a pair $(a,b)$ such $a \in \Sigma$ and $b \in \Gamma$ are elements whose arities sum up to $n$.
    \item {\bf Terms.} For a ranked set  $\Sigma$, define a \emph{term}\footnote{The terms that we use are special kinds of terms, where each variable is used exactly once, sometimes such terms are called \emph{linear}.} to be tree over $\Sigma + \set \portletter$, where $\portletter$ is a fresh symbol of arity zero. Define $\trees \Sigma$ to be the ranked set of terms over $\Sigma$, where the arity of a term is the number of ports (a port is a node with label $\portletter$).
\end{enumerate}    
\end{definition}


An alternative definition of $\trees \Sigma$ is that it is the least set which satisfies the following recursion
\begin{align*}
    \trees \Sigma = \set \portletter + \coprod_{a \in \sigma} \underbrace{\trees \Sigma \otimes \cdots \otimes \trees \Sigma}_{\text{arity of $a$ times}}
\end{align*}

