
\section{Ranked sets and operations on them}
\paragraph*{Ranked sets.}
Fix a set of variables $\varnames$.
\begin{definition}[Ranked set]
    A \emph{ranked set} consists of a set, called the \emph{underlying set}, together with a mapping  which assigns to each element of the underlying set an \emph{arity}, which is a finite subset of the variables $\varnames$. 
\end{definition}
When talking about elements of a ranked set, we mean elements of the underlying set.   For a ranked set $A$ and a finite set of variables $X \subseteq \varnames$, we write $\slice A X$ for the elements of $A$ that have arity $X$. 

\paragraph*{Type constructors.}
Consider the following operations, which are used to define new ranked sets based on existing ones.
\begin{enumerate}
    \item {\bf Disjoint union $A+B$.} An element of $A+B$ is either an element of $A$ or an element of $B$, with the arities inherited from $A,B$. 
    \item {\bf Cartesian\footnote{This is the Cartesian product in the category where objects are ranked sets, and morphisms are arity preserving functions.}  product $A \times B$.} An arity $X$ element of $A \otimes B$ is a pair $(a,b)$ such $a \in A$ and $b \in B$ are elements whose arities are both $X$.
    \item {\bf Tensor product $A \otimes B$.} An arity $X$ element of $A \otimes B$ is a pair $(a,b)$ such $a \in A$ and $b \in B$ are elements whose arities form a partition of $X$. 
    \item {\bf Trees.} For a ranked set  $A$, define a \emph{term} over $A$ of arity $X$ to be a tree, where nodes are labelled by  $A$ or $X$, and edges are labelled by  variables, such that:
    \begin{itemize}
        \item labels from $X$ appear only in the leaves, and each label from $X$ appears exactly once;
        \item if a node has a label $a \in A$ of arity $Y$, then outgoing edges of the node are labelled by variables from $Y$ in one-to-one fashion. 
    \end{itemize}
    Define $\trees A$ to be the ranked set of terms over $A$. Here is a picture: \mypic{1}
\end{enumerate}    

An alternative definition of $\trees \sigma$ is that it is the least set which satisfies the following recursion
\begin{align*}
    \trees \sigma = \varnames + \coprod_{a \in \sigma} \underbrace{\trees \sigma \otimes \cdots \otimes \trees \sigma}_{\text{arity of $a$ times}}
\end{align*}

We write $\treesz \Sigma$ for the (unranked) set of terms over $\Sigma$ which have arity $\emptyset$. 
