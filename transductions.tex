\section{Trees and transductions}
\label{sec:trees-transductions}
A \emph{ranked set} is a set where each element has an associated \emph{arity} in $\set{0,1,2,\ldots}$. The following picture describes the notion of trees that use and some terminology:
\mypic{1}
% When talking about elements of a ranked set, we mean elements of the underlying set.   For a ranked set $A$ and a finite set of variables $X \subseteq \varnames$, we write $\slice A X$ for the elements of $A$ that have arity $X$. 

We use standard tree terminology, such as ancestor, descendant, child, parent. 

We define a tree language over a ranked alphabet $\Sigma$ to be any subset of $\trees \Sigma$. A $k$-ary query over $\Sigma$ is defined to be any subset of 
\begin{align*}
    \set{(t,x_1,\ldots,x_n) : t \in \treesz \Sigma \text{ and $x_1,\ldots,x_n$ are nodes in $t$}}.
\end{align*}
In particular, a tree language can be viewed as a nullary query, i.e.~one of arity zero.  A tree-to-tree transduction is defined to be any function
\begin{align*}
    f : \treesz \Sigma \to \treesz \Gamma
\end{align*}
for finite ranked sets $\Sigma, \Gamma$. We are interseted in languages, queries and transductions which can be defined using logic, mainly first-order logic, as described below.

  
\paragraph*{Logic on trees.} For an introduction to the  logic-automata connection, see~\cite{thomas1997languages}. The basic idea is to view a tree as a model, and then to use logics such as first-order logic to describe properties of this model.

\begin{definition}[Tree as a model]\label{def:tree-model}
    The \emph{model} associated to a  tree $t$ over a ranked alphabet $\Sigma$ has as its universe the nodes of the tree, and is equipped with the following predicates:
    \begin{itemize}
        \item $x$ is a descendant of $y$;
        \item $x$ is an $i$-th child, for every $i \in \set{1,2,\ldots}$;
        \item $x$ has label $a$, for every $a \in \Sigma$.
    \end{itemize}        
\end{definition}
Technically speaking, there are infinitely many ``$i$-th child'' predicates, but only the ones where $i$ is at most the maximal arity of $\Sigma$ are going to be interpreted in a nonempty way.

 To express properties of trees, we use first-order logic and monadic second-order logic \mso. For definitions of these logics, see~\cite[Section 3]{thomas1997languages}.
 A sentence of first-order logic (or \mso) describes a tree language, i.e.~a set of trees over a fixed alphabet, while a formula with $n$ free first-order variables describes an $n$-unary query.  One could also consider queries with free set variables; these are not used in this paper.

The following definition shows how to describe  a tree-to-tree function using logic. A \emph{relational vocabulary} is a set of relation names, each one with associated arity -- formally speaking this is the same thing as a ranked set. 

\begin{definition}[First-order transduction]\label{def:fo-transduction}
    The syntax of a \emph{first-order transduction}  is given by:
\begin{enumerate}
    \item Input and output relational vocabularies $\tau,\sigma$;
    \item A copying constant $k \in \set{1,2,\ldots}$;
    \item For every $i \in \set{1,\ldots,k}$ a \emph{universe formula} $\varphi_i(x)$ over $\sigma$
    \item For every relation name $R \in \sigma$ of arity $n$, and every tuple $\bar i \in \set{1,\ldots,k}^n$, a first-order formula $\varphi_R^{\bar i}(x_1,\ldots,x_n)$ over $\tau$
\end{enumerate}
\end{definition}

The semantics of a first-order transduction is  a function
\begin{align*}
    \text{models over $\sigma$} \quad \to \quad \text{models over $\tau$}
\end{align*}
defined as follows. If the input model is $\mathbb A$, then the output model is defined as follows. The universe consists of pairs $(a,i)$ such that $a$ is an element of the input model and $i \in \set{1,\ldots,k}$ is such that $\varphi_i(a)$ is true. The predicates are interpreted analogously, using the formulas from item 4. 
\begin{align*}
    \set{(a,i) : a \in \mathbb A, i \in \set{1,\ldots,k} , \mathbb A \models \varphi_i(a)}
\end{align*}

We care about first-order transductions that input and output trees. By abuse of notations, we say that a  function 
    \begin{align*}
        f : \treesz \Sigma \to \treesz \Gamma
    \end{align*}
    is  a first-order transduction if there is a first-order transduction which gives the same result, assuming the trees are encoded as models as in Defintion~\ref{def:tree-model}.   

    We will also talk about \mso tree-to-tree transductions, which are defined the same way, except that the formulas of item 4 are allowed to use \mso. 
