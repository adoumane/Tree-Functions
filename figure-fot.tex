
\newcommand{\simplefunfig}[4]{
    \begin{tabular}{cc}
        $\ranked{
        \xymatrix@C=1.5cm{
#2 \ar[r]^-{#1}& #3
        }}$
        \\
        {#4}
    \end{tabular}   
 }

 \newcommand{\reversiblefunfig}[4]{
    \begin{tabular}{cc}
        $\ranked{
        \xymatrix@C=1.5cm{
#2 \ar@<.5ex>[r]^-{#1}& #3
\ar@<.5ex>[l]
        }}$
        \\
        {#4}
    \end{tabular}   
 }



 \newcommand{\laterfunfig}[3]{
    \begin{tabular}{cc}
        $\ranked{
        \xymatrix@C=1.5cm{
#1 & #2
        }}$
        \\
#3
    \end{tabular}   
 }
 



\begin{figure}
    \begin{tabular}{cc}
        \simplefunfig
        {\iota_i}
        {\Sigma_i}
        {\Sigma_1 + \Sigma_2}
        {$a \mapsto (a,i)$}
        &
        \simplefunfig
        {\mathrm{forget}}
        {\Sigma+\Sigma}
        {\Sigma}
        {$(a,i) \mapsto a$}
        \\ \\
        \reversiblefunfig
        {\mathrm{swap}}
        {\Sigma \otimes \Gamma}
        {\Gamma \otimes \Sigma}
        {$\tensorpair{a,b} \mapsto \tensorpair{b,a}$}
        &
        \reversiblefunfig
        {\distrtensor}
        {(\Sigma_1 + \Sigma_2)\otimes \Gamma}
        {(\Sigma_1 \otimes \Gamma) + (\Sigma_2 \otimes \Gamma)}
        {$\tensorpair{(a,i),b} \mapsto (\tensorpair{a,b},i)$}
        \\ \\
        \reversiblefunfig
        {\distrtensor}
        {\shallowterm {(\Sigma_1 + \Sigma_2)} \Gamma}
        {(\shallowterm {\Sigma_1} \Gamma) + (\shallowterm {\Sigma_2} \Gamma)}
        {$(a,i)\tensorpair{b_1,\ldots,b_n} \mapsto (a\tensorpair{b_1,\ldots,b_n})$ }
        &
        \reversiblefunfig
        {\distrtensor}
        {\shallowterm {(\Sigma_1 \otimes \Sigma_2)} \Gamma}
        {(\shallowterm {\Sigma_1} \Gamma) \otimes (\shallowterm {\Sigma_2} \Gamma)}
        {
        \begin{tabular}{c}
            $\tensorpair{a_1,a_2}\tensorpair{b_1,\ldots,b_n} \mapsto$ \\
            $\tensorpair{a_1 \tensorpair{b_1,\ldots,b_{n_1}}, a_2\tensorpair{b_{n_1+1},\ldots,b_n} }$  \\
            where $n_1$ is the arity of $a_1$
        \end{tabular}    
        }
        \\ \\
        \reversiblefunfig
        {}
        {\rSigma}
        {\shallowterm 1 {\tmonad \Sigma}}
        {$a \mapsto 1.\tensorpair a$}
        &
        \reversiblefunfig
        {\composeterm}
        {1 + \shallowterm \Sigma {\tmonad \Sigma} }
        { \tmonad \Sigma}
        {
        \begin{tabular}{c}
            Every term is either just a port,\\ or has a root and child subterms.    
        \end{tabular}    
        } 
        \\ \\
        \simplefunfig
        {}
        {\Sigma}
        {\reduce k \Sigma}
        {$a \mapsto a/(i \mapsto (1,i))$ } &
        \simplefunfig
        {}
        {\reduce {k_1} \reduce {k_2} \Sigma}
        {\reduce {k_1 \cdot k_2} \Sigma}
        {$(a/f)/g \mapsto a/(g \circ f)$}
        \\ \\
        \simplefunfig
        {\unit_\Sigma}
        {\Sigma}
        {\tmonad \Sigma}
        {\tablepic{69}}
        & \\ \\
        \simplefunfig
        {\flatt_\Sigma}
        {\tmonad \tmonad \Sigma}
        {\tmonad \Sigma} 
        {\tablepic{73}}
        &
        \simplefunfig
        {\distrtf}
        { \tmonad \reduce 1 \Sigma}
        {\reduce 1 \tmonad \Sigma}
        {\tablepic{74}}
        \\ \\
        \laterfunfig
        {\tmonad(\Sigma_1+\Sigma_2) \ar@<.5ex>[r]^{ \ancfact}
        \ar@<-.5ex>[r]_{\decfact}}
        {\tmonad(\tmonad \Sigma_1 + \tmonad \Sigma_2)} 
        {see Section~\ref{sec:atomic-and-combinators}} 
        &
        \laterfunfig
        {\tmonad \Sigma \ar[r]^-{\preorder}}
        {\reduce 1 \tmonad(\rSigma + 0 + 2)}
        {see Section~\ref{sec:atomic-and-combinators}} 
        \\ \\ 
        \laterfunfig
        {\shallowterm{(\reduce k \Sigma)}{\Gamma^k} \ar[r]^-{\unfold}}
        {(\shallowterm \Sigma \Gamma)^k}
        {see Section~\ref{sec:atomic-and-combinators}} 
    \end{tabular} 
    \caption{    \label{fig:fo-term}The atomic functions. The functions are parametrised by types  $\rSigma, \ranked{\Sigma_1}, \ranked{\Sigma_2}, \rGamma$. In the above, the type $\ranked i$ for $i \in \set{0,1,2}$ represents a  ranked set   with one element of arity $i$. Some of the functions are bijections, as indicated by double arrows, in these cases both the function and its inverse are atomic functions. }
\end{figure}


