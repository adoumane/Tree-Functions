


\section{Register tree transducers}
\label{sec:stt}
We now begin the proof of the harder implication in Theorem~\ref{thm:main}, which says that every first-order tree-to-tree transduction is derivable. Our proof passes  through an automaton model, which  is roughly based on existing transducer models for \mso transductions from~\cite{bloem_comparison_2000,alur2017streaming}.
The automaton  uses registers to store parts of the output tree. The semantics of the automaton involves two phases: (a) mapping the input tree to an expression that uses register updates; (b) evaluating the expression. These phases are described in more detail below, and each of them represents a derivable function.

%Before giving a formal definition, we present an example based on  pre-order traversal from Example~\ref{ex:pre-order}.
%\begin{example}\label{ex:preorder-register} In Example~\ref{ex:pre-order}, we considered a tree-to-tree function which computed pre-order traversal, using the following input and output alphabets:
%    \mypic{21}
%We now show a transducer which recognises the same function, and which uses  a single register.  After reading processing an input tree, the register stores a unary term (not a tree), which represents the pre-order traversal of the  input tree, and which has a single port in the rightmost leaf. The updates of this transducer are illustrated as follows. If the input tree is a single leaf, then the register is set to the following term:
%    \mypic{47}
%    For a tree which is not a single leaf, and its root symbol has $n>0$ children, we first compute the register value  for each of the $n$  child subtrees, resulting in unary terms $r_1,\ldots,r_n$,  and then we combine these terms into a single one, which is illustrated below for the case of $n=2$:
%    \mypic{34}
%    Once the whole input tree has been processed, resulting in a register value $r$ in the root node, the output tree is obtained from this register value by plugging its port with a node, as illustrated in the following picture:
%    \mypic{48}
%\end{example}
%
%

% We use an algebraic syntax for register tree transducers. 
% \emph{Linear polynomials.} Suppose that $\Sigma$ is a finite ranked alphabet.  For a ranked set $X$, define a \emph{polynomial with variables $X$ over $\Sigma$} to be any term in $\tmonad (\Sigma + X)$. Given such a polynomial $p$, and an arity preserving valuation $v : X \to \tmonad \Sigma$, we define $p(v)\in \tmonad \Sigma$ to be the term obtained from $p$ by replacing each variable with its value under $v$. The arity of $p(v)$ is the same as the arity of $p$.  A polynomial is called \emph{linear}, or non-duplicating, if each variable from $X$ is used at most once.

% \begin{definition}
%     For a ranked set $\Sigma$, define $\aalg_\Sigma$ to be the following multisorted algebra. 
%     \begin{itemize}
%         \item The sorts are arities, i.e.~numbers in $\set{0,1,2,\ldots}$, and elements of sort $n$ are $n$-ary terms in $\tmonad  \Sigma$;
%         \item The operations are the polynomial operations. 
%     \end{itemize}
% \end{definition}


%We now proceed with a more formal definition of register tree transducers.

\paragraph*{Register valuations and updates.} We begin by explaining how the registers work.  The registers store terms, which are parts of the output tree. Each register has an arity: registers of arity zero  store trees, registers of arity one store unary terms, etc.  

Fix two finite ranked sets: the \emph{register names} $\regnames$ and the \emph{output alphabet} $\rGamma$.
A \emph{register valuation} is defined to be any arity-preserving function from the register names $\regnames$ to terms $\tmonad \rGamma$. 
% Here is an example: 
% \mypic{31}
% The set of register valuations is not a ranked set, hence we write it in black.
To transform register valuations, we use \emph{register updates}.  A register update is an operation which inputs several  register valuations and outputs a single register valuation. For  $n \in \set{0,1,\ldots}$, an \emph{$n$-ary register update}  is defined to be any arity-preserving function
\begin{align*}
    %\label{eq:reg-operation}
    % \ranked {u : \regnames \rto \tmonad ( \rGamma + \underbrace{{\regnames + \cdots + \regnames}}_{\text{$n$ times}})}
    \ranked {u : \regnames \rto \tmonad ( \rGamma + n\regnames)},
\end{align*}
where $\ranked{nR}$ stands for the disjoint union of $n$ copies of $\regnames$. The $i$-th copy of $\regnames$ represents the register contents in the $i$-th argument.
% Register valuations can be seen as the special case of register updates of arity zero.  
Here is a picture of a register update which has arity 3 and uses two registers $r$ and $s$:
\mypic{32}
An $n$-ary register update $\ranked u$ induces a operation, which inputs  $n$ register valuations and  outputs the  register valuation obtained by  taking $\ranked u$ and replacing the $i$-th copy of a register name with the  contents of that register in the $i$-th input register valuation. 
Register updates have arities, and therefore the ranked set of register updates is written in red, and can be used for labels in a tree. For such a  tree   
\begin{align*}
    t \in  \trees(\ranked{\text{register updates}}),
\end{align*}
define its \emph{evaluation} to be the register valuation defined by induction in the natural way. Note that register updates of arity zero are the same as register valuations, which gives the induction base.
%  the evaluation of a tree is the register valuation obtained by applying the register update in the root to the register updates obtained by evaluating the child subtrees. (For a tree which is just a leaf, the leaf is labelled by a register update of arity zero, which is the same thing as a register valuation). 

% A register update can be applied not only to register valuations, but also to register updates (which generalise register valuations, since register valuations are register updates of arity 0): if $\ranked{u}$ is an $n$-ary register update, and $\ranked{u_1,\ldots,u_n}$ are register updates of arities $m_1,\ldots,m_n$, then the register update
% \begin{align*}
% \ranked{u}(\ranked {u_1},\ldots,\ranked{u_n})  \qquad \text{of arity $m_1+\cdots + m_n$}
% \end{align*}

\paragraph*{First-order relabellings.} Our automaton has no states. Instead, it uses a first-order relabelling, as defined below, to  directly assign to each node of the input tree a register update that will be applied in that node. A similar model is used by    Bloem and Engelfriet~\cite[Theorem 17]{bloem_comparison_2000}, except that in their case, the first phase uses  \mso relabellings, and the second phase is   a attribute grammar. 


% (we call such formulas first-order  unary tree queries over $\rSigma$)
\begin{definition}[First-order relabelling] \label{def:forat}  A \emph{first-order relabelling} is given by two finite ranked sets $\rSigma$ and $\rGamma$, called the \emph{input and output alphabets}, and a family 
    \begin{align*}
    \set{\varphi_a(x)}_{a \in \rGamma}
    \end{align*}
    of first-order formulas, which have one free variable and use the vocabulary of trees over  $\rSigma$. These formulas need to satisfy the following restriction:
    \begin{enumerate}
        \item[(*)] for every tree over the input alphabet and node in that tree, there is a unique output letter $a \in \rGamma$ such that $\varphi_a(x)$ selects the node; furthermore, the arity of $a$ is the same as the arity of (the label of) the  node. 
    \end{enumerate}
The semantics of a  first-order tree relabelling is a function 
\begin{align*}
\trees \rSigma \to \trees \rGamma,
\end{align*}
which changes the label of every node in the input tree to the unique letter described in  (*). 
      \end{definition}


A first-order tree relabelling is a very special case of a first-order tree-to-tree transduction, where only the labelling of the input tree is changed, while the universe as well as the child and descendant relations are not affected. 

\paragraph*{Register transducers.} Having defined registers, register updates, and first-order tree relabellings, we are now ready to define our automaton model.


\begin{definition}[First-order register transducer]\label{def:stt}
The syntax of a first-order register transducer consists of: 
\begin{itemize}
    \item \emph{Input and output alphabets} $\rSigma$ and $\rGamma$, which are finite ranked sets.
    \item A finite ranked set of \emph{registers} $\regnames$, equipped with a total order.
    \item A designated \emph{output register} in $\regnames$, of arity zero.
    \item A \emph{transition function}, which is a  first-order  relabelling
    \begin{align*}
      \trees{\rSigma} \to \trees{\rDelta},
    \end{align*}for some finite set $\rDelta$ of  register updates over registers $\regnames$ and output alphabet $\rGamma$. We require  all register updates in  $\rDelta$ to be single-use and monotone, as defined below:
    \begin{enumerate}
        \item \emph{Single-use\footnote{The  single-use restriction  is a standard feature of transducer models with linear size increase~\cite{bloem_comparison_2000, alurStreamingStringTransducers2011,alur2017streaming}.   It prohibits iterated duplication of registers, which would lead to exponential size outputs.    
        }.}  An $n$-ary register update $\ranked{u}$ is  called  \emph{single-use} if   every $r \in \ranked{n \regnames}$
        appears in at most one term from $\set{\ranked u(s)}_{s \in \regnames}$, and it appears at most once in that term. 
        \item \emph{Monotone\footnote{This is notion of monotonicity corresponds to the one used in Section~\ref{sec:unfolding}, see the comments on page~\pageref{page:monotone-discussed}. A similar notion  appears in~\cite[p. 7]{bojanczykRegularFirstOrderList2018}.}.} This condition uses the total order on registers.  An $n$-ary  register update $\ranked u$ is called monotone  if for every $i \in \set{1,\ldots,n}$, the binary relation $\to_i$ on register names $r,s \in \regnames$  defined by
        \begin{align*} 
            r \to_i s \quad \text{if} \quad  \text{the $i$-th copy of $r$ appears in $\ranked u(s)$},
        \end{align*}
        which is a partial function from $r$ to $s$ when $\ranked u$ is single-use, is monotone:
        \begin{align*}
            r_1 \leq r_2 \land r_1 \to_i s_1  \land  r_2 \to_i s_2  \quad \Rightarrow \quad  s_1 \leq s_2
        \end{align*}
    \end{enumerate}
    % \item A \emph{final function}
    % \begin{align*}
    %  F : \trees \rSigma \to \trees (\ranked{\regnames + \rGamma}),
    % \end{align*}
    % which has finitely many outputs  and such that for every output $t$, the   inverse image $F^{-1}(t)$ is first-order definable. The final function has no single-use or monotonicity requirements, it is also not necessarily a tree relabelling (in fact, it cannot be a tree relabelling, because it has finitely many outputs).
\end{itemize}
\end{definition}


The semantics of the transducer is  a tree-to-tree function, defined as follows. The input is a tree over the input alphabet. To this tree,  apply the transition function, yielding a tree of register updates. Next, evaluate the tree of register updates, yielding a register valuation. The output tree is defined to be the contents of the designated output register.
% the composition the following three operations:
% \begin{align*}
% \xymatrix@C=1.6cm{
%     \trees{\rSigma} {\ar[r]^{\text{transition}}_{\text{function}}} & \trees {\rDelta} \ar[r]^-{
%     \text{evaluate}
%     } & \rDelta {\ar[r]^{\text{content of }}_{\text{output register}}} & \trees \rGamma
% }
% \end{align*}


The main difference of our model with respect to prior work is that we want to capture  tree transformations defined in first-order logic, as opposed to \mso used in~\cite{bloem_comparison_2000,alurStreamingStringTransducers2011,alur2017streaming}. This is why we use first-order relabellings instead of  \mso relabellings.  For the same reason, we require the register updates to be monotone, see the discussion in Section~\ref{sec:unfolding}.  
% We conjecture that monotonicity -- which is essential to our model -- cannot be achieved without

 
% The following two examples explain the single-use and monotonicity restrictions. 
% \begin{example}\label{ex:single-use}
%     To see the need for the single-use restriction, consider the following abstract register tree transducer, which is not single-use. The input alphabet has two letters of arities zero and one. There is one register $r$ of arity zero, and the update function is the following tree homomorphism \mypic{33}
% The output tree has one node labelled by  $r$. This function recognised by this transducer maps an input tree (which is a sequence of unary nodes followed by a leaf)  to a complete binary tree of same depth. In particular this, function has exponential size increase, and therefore it cannot be a first-order tree-to-tree transduction, since the latter have linear size increase\footnote{single-use updates imply linear size outputs, but the converse implication is also true: a corollary of~\cite[Theorem 7.1]{engelfriet_macro_2003} is that  if a  register tree transducer has linear size increase but not necessarily single-use updates, then it is equivalent to a register tree transducer   which has single-use updates.}.
% \end{example}

% \begin{example}\label{ex:monotone}
%     To see the need for the monotonicity restriction, consider the following abstract register tree transducer, which violates monotonicity. The input alphabet is the same as in Example~\ref{ex:single-use}, i.e.~there are two letters of arities zero and one. There are two registers $r,s$ of arity zero, and the update function is the following tree homomorphism \mypic{35}
%     The output tree has one node labelled by  $r$. 
% Because the two registers get swapped in every step (which is a violation of monotonicity), the output  tree will have a unary (blue)  node if and only if the input tree has an even number of nodes. It follows that the recognised function is not a first-order tree-to-tree transducer, because first-order logic cannot compute parity. 
% \end{example}

The main result of this section is that first-order register transducers are expressively complete for first-order tree-to-tree transductions. 

\begin{theorem}\label{thm:stt}
    Every first-order tree-to-tree transduction is recognised by a first-order  register transducer. 
\end{theorem}

The proof, which is in Appendix~\ref{sec:stt-appendix}, uses the composition method for logic,  like similar proofs for~\cite[Theorem 4.6]{alur2017streaming} and~\cite[Theorem 14]{bloem_comparison_2000}. 
The converse  inclusion in the  theorem is also true. This is can be shown directly without much difficulty, following the same lines as in~\cite[Section 5]{bloem_comparison_2000}. The converse inclusion also follows from   other results in this paper: (a) we show in the following sections that every function computed by the transducer is derivable; and (b)  derivable functions are first-order tree-to-tree transductions by the easy implication in Theorem~\ref{thm:main}.

\paragraph*{Proof strategy for Sections~\ref{sec:fo-translation}--\ref{sec:one-register}.} By Theorem~\ref{thm:stt}, to prove derivability of  every first-order tree-to-tree transduction, it suffices to prove derivability for first-order register transducers. In a first-order register transducer,  the computation  has two steps: a first-order relabelling, followed by evaluation of the register updates. The first step is handled in Section~\ref{sec:fo-translation}, and the second step is handled in Sections~\ref{sec:stt-derivable} and~\ref{sec:one-register}.




% Thanks to the above lemma, to prove Proposition~\ref{prop:forat} it is enough to show that the prime functions and combinators can derive the (characteristic functions of) the child, until and since queries mentioned in the lemma. This is proved in the appendix. 

% \begin{lemma}\label{lem:decomposition-of-fo-transductions} Every first-order tree function $f$  can be decomposed as
%     \begin{align*}
%         \xymatrix@C=2cm{\trees \Sigma \ar[d]_f \ar[r]^g& \trees \overbrace{(\Delta \product \cdots \product \Delta)}^{\text{$k$ times}} \ar[d]^{\mathsf{unfold}} \\   \trees \Gamma  & \trees \Delta \ar[l]^h}
%     \end{align*}
%     where 
%     \begin{itemize}
%         \item $\Delta$ is a finite ranked alphabet and $k \in \set{1,2,\ldots}$;
%         \item $g$ is an fo-translation;
%         \item $h$ is recognised by a first-order register transducer which has one register, and that register has arity one.
%     \end{itemize}
% \end{lemma}

% By the above lemma, it remains to deal with first-order translations and register tree transducers with one unary register. We do this in Sections~\ref{sec:fo-translation} and~\ref{sec:one-register}, respectively.
