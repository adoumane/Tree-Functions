\section{The proof strategy}
\label{sec:strategy}
In this section, we present the proof strategy for the top-down inclusion in Theorem~\ref{thm:main}:
\begin{align*}
    \text{first-order tree-to-tree interpretations} \qquad \subseteq \qquad \text{first-order tree functions}
\end{align*}
The general idea is to convert a first-order tree transduction into an automaton model (streaming tree transducers), and then decomposition results for automata and logic on trees to arrive at the atomic tree functions. We begin by describing the automaton nodel. 

\subsection{Streaming tree transducers}
The automaton model is based on streaming tree transducers~\cite{alur2017streaming}, although are syntax is based more on notions from universal algebra such as terms and substituion.
% We use an algebraic syntax for streaming tree transducers. 
% \emph{Linear polynomials.} Suppose that $\Sigma$ is a finite ranked alphabet.  For a ranked set $X$, define a \emph{polynomial with variables $X$ over $\Sigma$} to be any term in $\trees (\Sigma + X)$. Given such a polynomial $p$, and an arity preserving valuation $v : X \to \trees \Sigma$, we define $p(v)\in \trees \Sigma$ to be the term obtained from $p$ by replacing each variable with its value under $v$. The arity of $p(v)$ is the same as the arity of $p$.  A polynomial is called \emph{linear}, or non-duplicating, if each variable from $X$ is used at most once.

% \begin{definition}
%     For a ranked set $\Sigma$, define $\aalg_\Sigma$ to be the following multisorted algebra. 
%     \begin{itemize}
%         \item The sorts are arities, i.e.~numbers in $\set{0,1,2,\ldots}$, and elements of sort $n$ are $n$-ary terms in $\trees  \Sigma$;
%         \item The operations are the polynomial operations. 
%     \end{itemize}
% \end{definition}




\paragraph*{Register valuations and updates.}   Let $\Gamma, R$ be finite ranked sets, which are called the \emph{output alphabet} and \emph{register names} respectively.  
A \emph{register valuation} (over $\Gamma$ and $R$) is defined to be any arity-preserving function
\begin{align*}
    v : R \to \trees \Gamma
\end{align*}
For  $k \in \set{0,1,\ldots}$, a \emph{$k$-ary register update} (over $\Gamma$ and $R$) is defined to be any an arity-preserving function
\begin{align*}
    u : R \to \trees ( \Gamma + \underbrace{R + \cdots +R}_{\text{$k$ times}})
\end{align*}
A $k$-ary register update can be applied to a $k$-tuple of register valuations, yielding a single register valuation, as illustrated in the following picture:
\begin{center}
    picture here
\end{center}
In other words, if we fix the an output alphabet $\Gamma$ and a set of registers $R$, then we can talk about an algebra, where the unvierse is the register valuations, and the operations are the register updates.

\paragraph*{Streaming tree transducers.} A register update is called \emph{copyless} if each copy of a register name is used at most once. More formally, for each $i \in \set{1,\ldots,k}$ and each register name $r \in R$, there is at most one $s \in R$ such that the $i$-th copy of $r$ appears in $u(s)$, and furthermore this copy appears at most once in $u(s)$. 


\begin{definition}[Streaming Tree Transducer]\label{def:stt}
    The syntax of a  \emph{streaming tree transducer} consists of 
\begin{itemize}
    \item Input and output alphabets $\Sigma$ and $\Gamma$, which are finite ranked sets;
    \item A finite set of states $Q$, not ranked.
    \item A finite set of regsiters $R$, ranked.
    \item For each input letter $a \in \Sigma$, of arity $n$, a transition function
 \begin{align*}
     Q^n \to Q \times \text{($n$-ary register updates over $\Gamma$ and $R$)}
 \end{align*}
 \item An ouptut function
 \begin{align*}
     Q \to \treesz(\Gamma + R)
 \end{align*}
\end{itemize}
\end{definition}

The semantics of a streaming tree transducer are defined as follows. For a tree over the input alphabet, one defines a pair (state, register valuation)
by induction on the size of the tree, using the transition function, in the natural way. Suppose that $t$  is a tree over the input alphabet, and let $(q,v)$ be the associated states and register  valuations. To obtain the output for the tree $t$, one applies the output function to $q$, yielding a tree over $\Gamma + R$, and then substitutes the register names according to the register valuation $v$. 


\begin{definition}[First-order Definable Streaming Tree Transducer]\ 
    A streaming tree-to-tree transducer is called \emph{first-order definable} if for every state $q$, the set of trees which get evaluated to state $q$ is a first-order definable tree language.
\end{definition}    
\begin{theorem}\label{thm:sst}
    The following classes of tree-to-tree transductions are the same:
    \begin{itemize}
        \item  \mso tree-to-tree transductions;
        \item streaming tree transducers.
    \end{itemize}
    First-order  tree-to-tree transductions are equal to first-order definable streaming tree transducers.
\end{theorem}

The result for \mso, adjusting for a different syntax,  was proved in~\cite[Theorem 4.6]{alur2017streaming}. The result for fo uses the same ideas. A proof is in the appendix.
\begin{center}
    do this appendix 
\end{center}

\subsection{The proof strategy}
Having defined streaming tree transducers, we can present the proof strategy for the bottom-up inclusion in Theorem~\ref{thm:main}. By Theorem~\ref{thm:sst}, it is enough to show that every first-order streaming tree transducer can be simulated using the atomic functions and combinators. To simulate streaming tree transducers, we use the following persepective on a run of such a transducer: (a)  replace each node in the input tree by an operation in the algebra of register updates; (b) evaluate the resulting term in the algebra of register upates; (c) apply the output function to the resulting register valuation. In the rest of the paper, we show that each of the steps (a), (b) and (c) can be simulated using our atomic funcitons and combinators. 

To formalise the replacement mentioned in step (a), we use the following definition. The definition is more general than needed for step (a), since it allows the new label of a node to depend not just on the node's subtree, but also no the rest of the tree. These sort of two-way dependencies will play a role in Section~\ref{sec:one-register}.


\begin{definition}[Fo translation]
     An fo-translation is a  function
    \begin{align*}
      f :   \treesz \Sigma \to \treesz \Gamma
    \end{align*}
    which is given by a family of unary queries 
    \begin{align*}
        \set{\varphi_a(x)}_{a \in \Gamma}
    \end{align*}
    over input alphabet $\Sigma$ such that for every tree $t \in \treesz \Sigma$ and node $x$ in $t$, there is exactly one $a \in \Gamma$ such that $\varphi_a(x)$ is true, and furthermore the arity of $a$ is the same as the arity (number of children) of $x$.  The output tree $f(t)$ is defined by replacing the label of each node $x$ by the unique $a$ which makes $\varphi_a(x)$ true.
\end{definition}

% \begin{lemma}
%     Every first-order streaming tree transducer is equivalent to one where all registers have arity 1.
% \end{lemma}



\begin{lemma}\label{lem:decomposition-of-fo-transductions} Every first-order tree function $f$  can be decomposed as
    \begin{align*}
        \xymatrix@C=2cm{\treesz \Sigma \ar[d]_f \ar[r]^g& \treesz \overbrace{(\Delta \otimes \cdots \otimes \Delta)}^{\text{$k$ times}} \ar[d]^{\mathsf{unfold}} \\   \treesz \Gamma  & \treesz \Delta \ar[l]^h}
    \end{align*}
    where 
    \begin{itemize}
        \item $\Delta$ is a finite ranked alphabet and $k \in \set{1,2,\ldots}$;
        \item $g$ is an fo-translation;
        \item $h$ is recognised by a first-order streaming transducer which has one register, and that register has arity one.
    \end{itemize}
\end{lemma}

By the above lemma, it remains to deal with first-order translations and streaming tree transducers with one unary register. This we do in Sections~\ref{sec:fo-translation} and~\ref{sec:one-register}, respectively.
