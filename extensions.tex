\section{Monadic second-order transductions}
\label{sec:mso-trans}
So far in this paper, we have studied first-order tree-to-tree transductions. The more studied type of transductions is \mso tree-to-tree transductions, which are defined the same way as first-order tree-to-tree transductions except that the formulas defining the output tree are \mso formulas, i.e.~they are allowed to use quantification over sets of nodes.   

Define an \emph{\mso relabelling} in the same way as the first-order relabellings from Definition~\ref{def:forat}, except that the unary queries can use \mso logic instead of first-order logic.  The following lemma is an immediate corollary of a result by Colcombet~\cite[Corollary 1]{colcombetCombinatorialTheoremTrees2007}, which says that every \mso formula on trees can be replaced by a first-order formula that runs on an \mso relabelling of the input tree. 
\begin{lemma}
    Every \mso tree-to-tree transduction can be decomposed as: (a) an \mso relabelling; followed by (b) a first-order tree-to-tree transduction.
\end{lemma}
An immediate corollary of the above lemma is that, once we extend the derivable functions with all \mso relabellings, then we get exactly the \mso tree-to-tree transductions. 
\begin{theorem}\label{thm:mso-transductions}
    A tree-to-tree function is an \mso tree-to-tree transduction if and only if it (is the tree restriction of a function that)  belongs to the class obtained by extending  Definition~\ref{def:derivable-function} with all \mso relabellings. 
\end{theorem}
In the above result, we simply throw in all \mso relabellings as prime functions. In contrast,  in the first-order case from the main theorem of this paper, we took care to have a small number of primitives. The small number of primitives for first-order logic  was possible thanks in part to the Schlingloff theorem that allowed decomposing first-order relabellings (see Section~\ref{sec:fo-translation}).  In principle, such a decomposition might be possible also for the \mso case, but proving it  would likely require developing new Krohn-Rhodes theory for tree languages, which seems to be a very hard problem that is outside the scope of this paper.



