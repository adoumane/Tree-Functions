
\section{Functions computed by register transducers are derivable}
\label{sec:stt-derivable}
We now  have all of the ingredients needed to derive functions computed by register transducers, as stated in the following proposition.
\begin{proposition}
    \label{prop:many-register} 
For every first-order  tree transducer, the computed function is derivable. 
\end{proposition}
The above proposition  completes the proof of the left-to-right implication in Theorem~\ref{thm:main}, as explained below:
\begin{align*}
\text{first-order tree-to-tree transductions}  \stackrel {\text{Theorem~\ref{thm:stt}}}\subseteq  \text{first-order tree transducers} \stackrel {\text{Proposition~\ref{prop:many-register}}} \subseteq \text{derivable}
\end{align*}

The rest of this section is devoted to proving Proposition~\ref{prop:many-register}. There will be four main ingredients, namely derivability for: first-order tree relabelings (Proposition~\ref{prop:forat}), normalisation of affine $\lambda$-terms (Theorem~\ref{thm:normalise}), and unfolding of the matrix power (Proposition~\ref{prop:monotone-unfold}).
\label{sec:proof-of-prop} 
To put these ingredients together,  we use  terminology inspired by universal algebra, as given in the following definition. 

\begin{definition}
    An \emph{algebra} $\alg$ consists of two sets 
    \begin{align*}
    \overbrace{\algdom \alg}^{\text{unranked}} \qquad \overbrace{\algops \alg}^{\text{ranked}},
    \end{align*}
equipped with a  \emph{shallow product}\footnote{
    A more standard approach would be to use a family of operations indexed by the signature
    \begin{align*}
    \set{f^\alg : (\algdom \alg)^{\arity f} \to \algdom \alg}_{f \in \algops \alg}.
    \end{align*}
   Our approach is equivalent, i.e.~defining a shallow product is the same as defining a family of operations.
}, which is function of type
\begin{align*}
 \shallowterm {(\algops \alg)}{\underbrace{(\algdom \alg)}_{\substack{\text{viewed as a ranked}\\ \text{set, with all elements}\\ \text{having arity zero}}}} \rto \redpar {\algdom \alg}.
\end{align*}
    For an algebra $\alg$, define its \emph{product} to be the function of type
    \begin{align*}
        \treepar {\algops \alg} \to \algdom \alg
    \end{align*}
    defined in the natural way.
    %  An algebra is called \emph{derivable} if its product  is derivable. 
\end{definition}


One example of an algebra is when the signature is some ranked set $\rSigma$, the domain is  $\trees \rSigma$, and the interpretation is defined in the natural way. In this case, the product operation is the identity. A variant of this algebra is when the signature is replaced by $\tmonad \rSigma$; in this case the product operation becomes flattening. 

Let us begin the proof of Proposition~\ref{prop:many-register}.
Fix a register transducer with $k$ registers. For the rest of this section, when talking about register valuations and register updates, we mean the registers of the fixed register transducer.
Another example of an algebra was implicit in the definition of register transducers: the domain is the register valuations and the signature is the  register updates. To prove  Proposition~\ref{prop:many-register}, we will show that the algebra of register updates embeds -- via the matrix power -- in an algebra, call it $\alg$, whose product operation is derivable as a special case of normalisation of affine $\lambda$-terms. 
 The idea behind the algebra  $\alg$ is to use  $\lambda$-terms to represent register contents, according to the following picture:
    \mypic{59}  
    Let  $\nmax$  be the maximal arity of registers in the fixed register transducer, and let  $\rGamma$ be its  output alphabet. For a term $t \in \tmonad \rGamma$ of arity at most $\nmax$, its representation as a  $\lambda$-term, which we denote by $t^\lambda$,  uses variables from the following finite set (we use the notation $\otype^i \to \otype$ defined in Example~\ref{ex:affine-not-enough}):
\begin{align*}
X  \quad \eqdef \quad \set{\typevar {x_a} {\otype^i \to \otype} : a \in \rGamma \text{ of arity $i$}} \cup \set{\typevar {x_i} \otype : i \in \set{1,2,\ldots,\nmax }}.
\end{align*}
Furthermore, $t^\lambda$ is affine and it can be typed using simple types  from  the following set finite set :
\begin{align*}
    \Tt \quad \eqdef \quad \set{\otype^i \to \otype : i \in \set{0,\ldots,\nmax}}
\end{align*}
This means that if $t$ has arity at most $\nmax$ -- which is true for any term that can appear in a register of our fixed register transducer -- then its $\lambda$-term representation $t^\lambda$ satisfies constraints as in Theorem~\ref{thm:normalise}.  On its own, $t^\lambda$ is already in normal form, so there is no need to normalise it, but we intend to compose  such $\lambda$-terms, leading to terms that are not in normal form (but which still use variables from $X$ and types from $\Tt$). 
This motivates the  following definition of an algebra, call it $\alg$. Its domain and signature are given by 
\begin{align*}
\overbrace{\set \bot + \trees \lamrank X}^{\algdom \alg}  \qquad \overbrace{ \tmonad \lamrank X}^{\algops \alg} \qquad \text{where }\lamrank X = {\overbrace{\set{x : x \in X}}^{\text{arity 0}} \cup \overbrace{\set{\lambda x : x \in X}}^{\text{arity 1}} \cup  \overbrace{\set @}^{\text{arity 2}}}
\end{align*}
In other words, the domain is $\lambda$-terms plus an error element, and the signature is $\lambda$-terms  with ports. The   product operation is 
\begin{align*}
t^\alg(t_1,\ldots,t_n) = \begin{cases}
    \text{normal form of $t(t_1,\ldots,t_n)$} & \text{if $t(t_1,\ldots,t_n)$ is affine and can be typed using $\Tt$}\\
    \bot & \text{otherwise}
\end{cases}
\end{align*}
In particular, if one of the inputs $t_1,\ldots,t_n$ is $\bot$, then the output is also $\bot$.  The algebra $\alg$ is designed so that we can apply Theorem~\ref{thm:normalise} about normalising $\lambda$-terms to see that the  product operation in the  algebra $\alg$ is derivable. Next, by observing that the matrix power is a form of syntactic sugar for manipulating registers in a register automaton, we see that there is an arity preserving 
    \begin{align*}
    \ranked h : \regups \to \mati k {(\algops \alg)}
    \end{align*}
    which has only monotone elements in its image, and which makes the following diagram commute
    \begin{align*}
    \xymatrix@C=2cm{
        \treepar {\regups} \ar[r]^-{\trees \ranked h}\ar[d]_{\text{product}} &
         \treepar {\mati k {(\algops \alg)}}\ar[r]^{\text{Corollary~\ref{cor:matrix-power}}}  & 
         \treepar {\algops \alg} \ar[d]^{\text{product}} \\
        \regvalss \ar[rr]_{(t_1,\ldots,t_k) \mapsto t^\lambda_1}&  & \algdom {\alg^{[k]}}
    }
    \end{align*} 
Putting this all together, we see that there is a derivable function which inputs a tree of register updates, and outputs the value of the first register, represented as a $\lambda$-term. The last remaining observation is that function $t^\lambda \mapsto t$, i.e.~the inverse of the $\lambda$-term representation, is derivable. This inverse can be achieved, for example, by composing a first-order tree relabelling with a tree homomorphism. This completes the proof of Proposition~\ref{prop:many-register}, and therefore also of Theorem~\ref{thm:main}. 