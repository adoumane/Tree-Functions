\section{First-order translations}\label{sec:fo-translation}




\newcommand{\nextmod}{\mathsf X}
\newcommand{\untilmod}{\mathsf U}
\newcommand{\sincemod}{\mathsf S}


\begin{definition}[2-ctl]
    Define 2-ctl to be the least set of unary queries which contains queries of the form ``node $x$ has label $a$'', and which is closed under the following connectives
    \begin{enumerate}
        \item \emph{Boolean.} Boolean combinations of unary queries, including negation;
        \item \emph{Next.} If $\varphi$ is a unary query and $i \in \Nat$, then  $\nextmod_i \varphi$ is also a unary query, which is true in nodes whose $i$-th child satisfies $\varphi$.
         \item \emph{Until.} If $\varphi,\psi$ are  unary queries, then  $\varphi \untilmod \psi$ is also a unary query, which is true in a node $x$ if there exists some $y > x$ such that: (a) $\psi$ is true in $y$; (b) $\varphi$ is true in all nodes $z$ with $x < z < y$;
         \item \emph{Since.} If $\varphi,\psi$ are  unary queries, then  $\varphi \sincemod \psi$ is also a unary query, which is true in a node $x$ if there exists some $y < x$ such that: (a) $\psi$ is true in $y$; (b) $\varphi$ is true in all nodes $z$ with $y < z < x$.
    \end{enumerate}
\end{definition}
\begin{theorem}~\cite[Theorem 2.6]{schlingloff1992expressive} \label{thm:ctl} The unary queries in 2-ctl are exactly the unary queries that are definable in first-order logic.
\end{theorem}

\begin{definition}[Characteristic function of a unary query]\label{def:characteristic}
    Let $\varphi(x)$ be a formula of first-order logic, over alphabet $\Sigma$. Define the \emph{characteristic function} of $\varphi(x)$ to be the function
            \begin{align*}
                f : \treesz \Sigma \to \treesz (2 \otimes \Sigma)
            \end{align*}
            which adds 0 or 1 to  the label of each node depending on whether the node satisfies $\varphi(x)$.
\end{definition}


\begin{lemma}\label{lem:fo-translation} Every fo translation is a first-order tree function.
\end{lemma}
\begin{pr}
    Every fo translation can be decomposed as 
    \begin{align*}
        f \circ f_1 \circ \cdots \circ f_n
    \end{align*}
    where each $f_i$ is the characteristic function of some unary first-order query as in Definition~\ref{def:characteristic}, while $f$ is a relabelling, i.e.~a function
    \begin{align*}
        f : \treesz \Sigma \to \treesz \Gamma
    \end{align*}
    that replaces each label $a \in \Sigma$ by some fixed label  $f(a) \in \Gamma$. Relabellings are clearly first-order tree functions, and first-order tree functions are closed under composition. It remains to deal with the characteristic functions. By Theorem~\ref{thm:ctl}, the characteristic function of every first-order unary query is a composition of characteristic functions of queries of the form:
    \begin{align*}
        \nextmod_i \Gamma \quad \Gamma \untilmod \Delta \quad \Gamma \sincemod \Delta \qquad \text{for finite ranked sets $\Gamma,\Delta \subseteq \Sigma$ and $i \in \set{1,2,\ldots}$} 
    \end{align*}
    where $\nextmod_i \Gamma$ is defined as 
    \begin{align*}
        \nextmod_i (\bigvee_{a \in \Gamma} a) 
    \end{align*}
    and likewise for $\Gamma \untilmod \Delta $ and $ \Gamma \sincemod \Delta$. Therefore it is enough to  deal with these, which we do below, in Lemmas~\ref{lem:nextmod},~\ref{lem:untilmod} and~\ref{lem:sincemod}.
    \end{pr}


    \begin{lemma}\label{lem:nextmod}
        For every finite ranked sets $\Gamma \subseteq \Sigma$ and $i \in \set{1,2,\ldots}$ the characteristic function of $\nextmod_i \Gamma$ is a first-order tree function.
    \end{lemma}
\begin{pr}
    
\end{pr}

\begin{lemma}\label{lem:untilmod}
    For every finite ranked sets $\Gamma, \Delta \subseteq \Sigma$  the characteristic function of $\Gamma \untilmod \Delta$ is a first-order tree function.
\end{lemma}
\begin{pr}

\end{pr}


\begin{lemma}\label{lem:sincemod}
    For every finite ranked sets $\Gamma, \Delta \subseteq \Sigma$  the characteristic function of $\Gamma \sincemod \Delta$ is a first-order tree function.
\end{lemma}
\begin{pr}

\end{pr}

    % \begin{definition}[Fo-rational functions]\ 
    %     \begin{enumerate}
    %         \item \emph{Characteristic functions.} Let $\varphi(x)$ be a formula of first-order logic, over vocabulary $\sigma$. Define the \emph{characteristic function} of $\varphi(x)$ to be the function
    %         \begin{align*}
    %             f : \treesz \sigma \to \treesz (2 \otimes \sigma)
    %         \end{align*}
    %         which adds 0 or 1 to  the label of each node depending on whether the node satisfies $\varphi(x)$.
    %         \item \emph{Fo-rational.} An \emph{fo-rational} function is any finite composition of charactersitic functions as defined in the previous item, and tree-to-tree homomorphisms. 
    %     \end{enumerate}
         
    % \end{definition}
