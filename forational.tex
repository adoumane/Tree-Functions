\section{First-order translations}\label{sec:fo-translation}

The goal of this seciton is to prove the following lemma.
\begin{lemma}\label{lem:fo-translation} Every fo translation is a first-order tree function.
\end{lemma}

 The key result is a tree variant of Kamp's Theorem. The tree variant, which was proved by Schlingloff in~\cite{schlingloff1992expressive}, gives a temporal logic similar to {\sc ltl} that is expressively complete for first-order logic on trees. We begin by describing this logic.


\newcommand{\nextmod}{\mathsf X}
\newcommand{\untilmod}{\mathsf U}
\newcommand{\sincemod}{\mathsf S}


\begin{definition}[2-ctl]
    Define 2-ctl to be the least set of unary queries which contains queries of the form ``node $x$ has label $a$'', and which is closed under the following connectives
    \begin{enumerate}
        \item \emph{Boolean.} Boolean combinations of unary queries, including negation;
        \item \emph{Next.} If $\varphi$ is a unary query and $i \in \Nat$, then  $\nextmod_i \varphi$ is also a unary query, which is true in nodes whose $i$-th child satisfies $\varphi$.
         \item \emph{Until.} If $\varphi,\psi$ are  unary queries, then  $\varphi \untilmod \psi$ is also a unary query, which is true in a node $x$ if there exists some $y \geq x$ such that: (a) $\psi$ is true in $y$; (b) $\varphi$ is true in all nodes $z$ with $x \leq z \leq y$;
         \item \emph{Since.} If $\varphi,\psi$ are  unary queries, then  $\varphi \sincemod \psi$ is also a unary query, which is true in a node $x$ if there exists some $y \leq x$ such that: (a) $\psi$ is true in $y$; (b) $\varphi$ is true in all nodes $z$ with $y \leq z \leq x$.
    \end{enumerate}
\end{definition}

The following result was proved in~\cite[Theorem 2.6]{schlingloff1992expressive}, although the proof in that paper looks too simple to me to be correct.

\begin{theorem} \label{thm:ctl} The unary queries in 2-ctl are exactly the unary queries that are definable in first-order logic.
\end{theorem}

The Since modality in 2ctl, which implies that the truth value of a formula in a node can depend on the node's ancestors, is crucial to the  expressive completeness result above. A one-way variant, having only next and until is weaker. Without since one cannot express the first-order definable property ``one some root-to-leaf path, the labels alternate between $a$ and $b$'', which can be proved using a similar technique as in~\cite[Theorem 1]{bojanczyk2008common}. 
\begin{definition}[Characteristic function of a unary query]\label{def:characteristic}
    Let $\varphi(x)$ be a formula of first-order logic, over alphabet $\Sigma$. Define the \emph{characteristic function} of $\varphi(x)$ to be the function
            \begin{align*}
                f : \treesz \Sigma \to \treesz (\Sigma_{\sf{Bool}})
            \end{align*}
            which adds 0 or 1 to  the label of each node depending on whether the node satisfies $\varphi(x)$.
\end{definition}


It is not hard to see that every fo translation can be decomposed as 
    \begin{align*}
        f \circ f_1 \circ \cdots \circ f_n
    \end{align*}
    where each $f_i$ is the characteristic function of some unary first-order query as in Definition~\ref{def:characteristic}, while $f$ is a relabelling, i.e.~a function
    \begin{align*}
        f : \treesz \Sigma \to \treesz \Gamma
    \end{align*}
    that replaces each label $a \in \Sigma$ by some fixed label  $f(a) \in \Gamma$. Relabellings are clearly first-order tree functions, and first-order tree functions are closed under composition. Therefore, to prove Lemma~\ref{lem:fo-translation}, it remains to deal with the characteristic functions. By Theorem~\ref{thm:ctl}, the characteristic function of every first-order unary query is a composition of characteristic functions of queries of the form:
    \begin{align*}
        \nextmod_i \Gamma \quad \Gamma \untilmod \Delta \quad \Gamma \sincemod \Delta \qquad \text{for finite ranked sets $\Gamma,\Delta \subseteq \Sigma$ and $i \in \set{1,2,\ldots}$} 
    \end{align*}
    where $\nextmod_i \Gamma$ is defined as 
    \begin{align*}
        \nextmod_i (\bigvee_{a \in \Gamma} a) 
    \end{align*}
    and likewise for $\Gamma \untilmod \Delta $ and $ \Gamma \sincemod \Delta$. Therefore it is enough to  deal with these, which we do below, in Lemmas~\ref{lem:nextmod},~\ref{lem:untilmod} and~\ref{lem:sincemod}.



    \begin{lemma}\label{lem:nextmod}
        For every finite ranked sets $\Gamma \subseteq \Sigma$ and $i \in \set{1,2,\ldots}$ the characteristic function of $\nextmod_i \Gamma$ is a first-order tree function.
    \end{lemma}
\begin{pr}
    To show that $\chi_{\nextmod_i \Gamma}:\trees \Sigma\to \trees{(\Sigma_{\sf{Bool}})}$,  the characteristic function of $\nextmod_i \Gamma$, is a first order tree function, we start applying the sibling function $\sf{Sib}:\trees\Sigma\to \trees \Sigma^{\sf{Sib}}$. Consider the function $f$
    \begin{align*}
    f:\Sigma^{\sf{Sib}} &\to \Sigma_{\sf{Bool}}\\
    (a,l)& \mapsto (a,1) \qquad\text{if } l[i]\in\Gamma, \\
    & \mapsto (a,0) \qquad\text{otherwise.}   \end{align*}
The function $f$ is clearly first order since it is arity preserving with finite domain.    
    
  We get $\chi_{\nextmod_i \Gamma}$ by liftinf $f$ to trees and composing it with $\sf{Sib}$. 
\end{pr}

\begin{lemma}\label{lem:untilmod}
    For every finite ranked sets $\Gamma, \Delta \subseteq \Sigma$  the characteristic function of $\Gamma \untilmod \Delta$ is a first-order tree function.
\end{lemma}
\begin{pr}
Let $\chi$ be the characteristic function of $\Gamma \untilmod \Delta$.
We start by decomposing our tree into blocks, depending on whether their node labels are in $\Gamma\cup\Delta$ or not. We will show that the restriction of $\chi$ to $\trees (\Gamma\cup\Delta)$, denoted $\chi_1$; and the restriction of $\chi$ to $\trees (\Sigma\setminus(\Gamma\cup\Delta))$, denoted $\chi_2$, are first-order tree functions. The function $\chi$ is obtained by applying $\chi_1$ or $\chi_2$ to each block, depending on its type, then by applying a flattening.
This decomposition is relevant because the value of a node depends only on the node labels of its block. %(it is constant equal to $0$ in the case of $\Sigma\setminus(\Gamma\cup\Delta)$ blocks).

The function $\chi_2:\trees (\Sigma\setminus(\Gamma\cup\Delta))\to \trees(\Sigma_{\sf{Bool}})$ is clearly a first-order tree function, as it assigns uniformly $0$ to every node label. 

To see that $\chi_1:\trees (\Gamma\cup\Delta)\to \trees(\Sigma_{\sf{Bool}})$ 
is a first-order function, we start by applying the function $\mathsf{Desc}_\Delta$ from
Example~\ref{}, which adds $1$ to the node labels if they have a descendent in $\Delta$ and adds $0$ otherwise. Then we decompose the obtained tree into blocks, depending on whether their node labels contain $0$ or not. For the block of nodes having $0$ in their label, their value w.r.t. $\Gamma\untilmod \Delta$ is indeed $0$, since they do not have a descendent in $\Delta$. To these blocks we apply the identity function. For the blocks that received $1$, we assign $1$ to the nodes whose labels are in  $\Gamma$ and $0$ to the others (whose labels are in $\Delta\setminus\Gamma$). Indeed, inside these blocks, a node $n$ labeled from $\Gamma$ has by construction a descendent in $\Delta$, moreover the path connecting $n$ to such descendent lies in the same block as $n$. If we consider the nearest such descendent $m$, the path between $n$ and $m$  is labeled in $\Gamma$ (since we are in blocks having only $\Gamma\cup\Delta$ labels). Thus $n$ satisfies $\Gamma\untilmod\Delta$. Nodes labeled in $\Delta\setminus\Gamma$ have obviously value $0$.      
\end{pr}


\begin{lemma}\label{lem:sincemod}
    For every finite ranked sets $\Gamma, \Delta \subseteq \Sigma$  the characteristic function of $\Gamma \sincemod \Delta$ is a first-order tree function.
\end{lemma}
\begin{pr}
The same proof as above, one only needs to replace the use of $\mathsf{Desc}_\Delta$ by that of $\mathsf{Anc}_\Delta$.
\end{pr}

    % \begin{definition}[Fo-rational functions]\ 
    %     \begin{enumerate}
    %         \item \emph{Characteristic functions.} Let $\varphi(x)$ be a formula of first-order logic, over vocabulary $\sigma$. Define the \emph{characteristic function} of $\varphi(x)$ to be the function
    %         \begin{align*}
    %             f : \treesz \sigma \to \treesz (2 \otimes \sigma)
    %         \end{align*}
    %         which adds 0 or 1 to  the label of each node depending on whether the node satisfies $\varphi(x)$.
    %         \item \emph{Fo-rational.} An \emph{fo-rational} function is any finite composition of charactersitic functions as defined in the previous item, and tree-to-tree homomorphisms. 
    %     \end{enumerate}
         
    % \end{definition}
