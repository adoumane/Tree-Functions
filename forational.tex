\section{First-order  relabellings}\label{sec:fo-translation}
In this section we prove derivability of the first computation step used in first-order register transducers.
\begin{proposition} \label{prop:forat}    
    Every first-order relabelling is derivable.
\end{proposition}
To prove the proposition, we use a decomposition of first-order relabellings into simpler functions, in the style of the Krohn-Rhodes theorem. 
%  a characterisation of first-order logic on trees in terms of temporal logic, which was proved by Schlingloff~\cite[Theorem 4.5]{schlingloff1992expressive}. The operators in this temporal logic (which are tree variants of the since and until operators of {\sc ltl}) can be viewed as very simple first-order relabellings; by the completeness result of Schlingloff it is enough to show that these operators are derivable.  
We use the name \emph{unary query} for a first-order formula with one free variable over the vocabulary of trees. This assumes some implicit alphabet $\rSigma$.
For a  unary query,  define its  \emph{characteristic function},  of type
\begin{align*}
 \trees \rSigma \to \trees (\ranked{\rSigma + \rSigma}),
\end{align*}
to be the function which replaces the label of each node by its first or second copy, depending on whether the node is selected by the query. This is a special case of a first-order relabelling. The key to Proposition~\ref{prop:forat} is the following  lemma, which decomposes first-order relabellings  into characteristic functions of certain basic unary queries.

\begin{lemma}\label{lem:schlingloff} Every first-order relabelling can be obtained by composing the following functions:
    \begin{enumerate}
        \item \label{it:relabelling} \emph{Letter-to-letter homomorphisms}. For  every finite $\rGamma,\rSigma$ and $\ranked {f : \rSigma \to \rGamma}$, its tree lifting $\trees \ranked f : \trees \rSigma \to \trees \rGamma$.
        \item \label{it:temporal-operators} For every finite  $\rSigma$ and its subsets $\rDelta, \rGamma \subseteq \rSigma$, the characteristic functions of the following unary queries over alphabet $\rSigma$:
        \begin{enumerate}
            \item \label{it:child} \emph{Child:} $x$ is an $i$-th child, for $i \in \set{1,2,\ldots}$
            \begin{align*}
            \child i (x); 
            \end{align*}
             \item \label{it:until} \emph{Until:}  $x$ has a descendant $y$ with label in $\rDelta$, such that all nodes strictly between $x$ and $y$ have label in $\rGamma$
             \begin{align*} 
                  \exists y\ y > x \land \rDelta(y) \land  \forall z \ (x < z < y \Rightarrow \rGamma(z));
             \end{align*} 
             \item \label{it:since}\emph{Since:} $x$ has an ancestor $y$ with label in $\rDelta$, such that all nodes strictly between $x$ and $y$ have label in $\rGamma$
             \begin{align*}
                  \exists y\ y < x \land \rDelta(y) \land  \forall z \ (y < z < x \Rightarrow \rGamma(z)).
             \end{align*} 
        \end{enumerate}
    \end{enumerate}
    
\end{lemma}

The  lemma uses a theorem of   Schlingloff~\cite[Theorem 2.6]{schlingloff1992expressive}, which says  that all first-order definable tree properties can be defined using a temporal logic with operators similar to the ones used in items~\ref{it:temporal-operators} of the lemma. Note that the temporal logic is a two-way logic, because  \emph{until} depends on the descendants of the node $x$, while \emph{since} depends on the ancestors. In fact, there is no temporal logic which characterises first-order logic, uses only descendants, and has finitely many operators~\cite[Theorem 5.5]{bojanczykWreathProductsForest2012}. 
The exact reduction to Schlingloff's theorem is  in Appendix~\ref{sec:AppendixForat}.

It remains to show that all of the functions from Lemma~\ref{lem:schlingloff} are derivable. 
The letter-to-letter homomorphisms from item~\ref{it:relabelling} are   a special case of homomorphisms discussed in Example~\ref{ex:filter}, and hence derivable. In Appendix~\ref{sec:AppendixForat}, we show that the functions from item~\ref{it:temporal-operators} are also derivable. In the proof, a key role is played by the factorisation functions discussed in Section~\ref{sec:factorisation-functions}. 
% i lemmas~\ref{lem:nextmod}, \ref {lem:untilmod} and \ref{lem:sincemod}. 