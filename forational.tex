\section{First-order tree relabellings}\label{sec:fo-translation}
The rest of the paper is devoted to the left-to-right implication in Theorem~\ref{thm:main}, which says that every first-order tree-to-tree transduction is derivable. As discussed in the proof strategy from Section~\ref{sec:strategy}, we begin with a special case of first-order tree-to-tree transductions, namely first-order relabellings. 

Define a \emph{unary   query over $\rSigma$} to be a first-order formula $\varphi(x)$ with one free variable, which uses the  vocabulary of models associated to  trees over $\rSigma$, as per Definition~\ref{def:tree-model}. Given a tree over $\rSigma$, a unary  query selects a subset of nodes. An example of a unary query is ``$x$ has at least four ancestors''. 

\begin{definition}[First-order tree relabelling] \label{def:forat}  A \emph{first-order tree relabelling} is given by two finite ranked sets, called the \emph{input and output alphabets}, and a family 
    \begin{align*}
    \set{\varphi_a(x)}_{a \in \text{ output alphabet}}
    \end{align*}
    of unary queries over the input alphabet such that:
    \begin{enumerate}
        \item[(*)] for every tree over the input alphabet and node in that tree, there is a unique output letter $a$ such that $\varphi_a(x)$ selects the node; furthermore, the arity of $a$ is the same as the arity of (the label of) the node. 
    \end{enumerate}
The semantics of a  first-order tree relabelling is a function from trees over the input alphabet to trees over the output alphabet, which changes the label of every node in the input tree to the unique letter described in  (*). 
      \end{definition}


A first-order tree relabeling is a very special case of a first-order tree-to-tree transduction, where only the labeling of the input tree is changed, while the universe as well as the child and descendant relations are not affected. 
The goal of this section is to prove the following proposition.
\begin{proposition} \label{prop:forat}    
    First-order tree relabelings are derivable.
\end{proposition}   
The key step in the proof is  Lemma~\ref{lem:schlingloff} below, which decomposes  first-order tree labelings into  simpler functions. 
For a unary query  over $\rSigma$,  its \emph{characteristic function} is the function of type
\begin{align*}
 \trees \rSigma \to \trees (\rSigma + \rSigma)
\end{align*}
which replaces the label of each node by its first or second copy, depending on whether the node is selected by the query. The following lemma shows that every first-order tree relabeling can be obtained by composing local label changes, together with characteristic functions of certain simple unary queries. 

\begin{lemma}\label{lem:schlingloff} The first-order tree relabelings are equal to the smallest class of functions which is closed under composition and which  contains the following functions:
    \begin{enumerate}
        \item \label{it:relabeling} \emph{Relabeling}. To every finite $\rGamma,\rSigma$ and $\ranked {f : \rSigma \to \rGamma}$, the tree lifting $\trees \ranked f : \trees \rSigma \to \trees \rGamma$.
        \item \label{it:child} \emph{Child.} For every $\rSigma$ and  $i \in \set{1,2,\ldots}$, the characteristic function of the  unary query 
        \begin{align*}
            \underbrace{\child i (x).}_{\text{$x$ is an $i$-th child}}
        \end{align*}
         \item \label{it:until} \emph{Until.} For every finite $\rGamma, \rDelta \subseteq \rSigma$,  the characteristic function of the unary query
         \begin{align*}
              \underbrace{\exists y\ y > x \land \rDelta(y) \land  \forall z \ (x < z < y \Rightarrow \rGamma(z)).}_{\substack{\text{$x$ has a descendant $y$ with label in $\rDelta$, such that}\\ \text{all nodes strictly between $x$ and $y$ have label in $\rGamma$}}} 
         \end{align*} 
         \item \label{it:since}\emph{Since.} For every $\rGamma, \rDelta \subseteq \rSigma$,    the characteristic function of the unary query
         \begin{align*}
              \underbrace{\exists y\ y < x \land \rDelta(y) \land  \forall z \ (y < z < x \Rightarrow \rGamma(z)).}_{\substack{\text{$x$ has a descendant $y$ with label in $\rDelta$, such that}\\ \text{all nodes strictly between $x$ and $y$ have label in $\rGamma$}}}  
         \end{align*} 
    \end{enumerate}
    In items \ref{it:child} -- \ref{it:since}, the alphabet is $\rSigma$, so the characteristic functions have type $\trees \rSigma \to \trees(\rSigma + \rSigma)$. 
\end{lemma}

This lemma follows easily from a result of Schlingloff~\cite[Theorem 2.6]{schlingloff1992expressive}, which says that all first-order definable tree properties can be defined using a temporal logic that has operators similar to the ones used in items~\ref{it:child} -- \ref{it:since} of the lemma. The exact reduction to Schlingloff's theorem is given in Appendix~\ref{sec:AppendixForat}.


In light of Lemma~\ref{lem:schlingloff},  Proposition~\ref{prop:forat} will follow once we  show that all functions  used in the lemma are derivable. The relabelings from item~\ref{it:relabeling} are immediate: we simply lift to terms (and therefore also trees) the function $\ranked f$, which is derivable by virtue of having a finite domain. The remaining functions are also derivable, as shown in Appendix~\ref{sec:relabeling}, lemmas~\ref{lem:nextmod}, \ref {lem:untilmod} and \ref{lem:sincemod}. 