\section{First-order tree relabellings}\label{sec:fo-translation}
The rest of the paper is devoted to the left-to-right implication in Theorem~\ref{thm:main}, which says that every first-order tree-to-tree transduction is derivable. As discussed in the proof strategy from Section~\ref{sec:strategy}, we begin with a special case of first-order tree-to-tree transductions, namely first-order relabellings. 


\begin{definition}[First-order tree relabelling] \label{def:forat} For a ranked set $\rSigma$, define a \emph{unary tree query over $\rSigma$} to be a first-order formula with one free variable, over the vocabulary of trees over $\rSigma$. A unary tree query selects nodes in trees over $\rSigma$.  Let $\rSigma, \rGamma$ be finite ranked sets. A \emph{first-order tree relabelling} is given by two finite ranked sets, called the \emph{input and output alphabets}, and a family of unary queries $\varphi_a(x)$ over the input alphabet, one for each letter of the output alphabet, such that:
    \begin{enumerate}
        \item[(*)] for every tree over the input alphabet, every node is selected by exactly one of the queries $\varphi_a(x)$; furthermore, the arity of $a$ is the same as the arity of (the label of) the node. 
    \end{enumerate}
A first-order tree relabelling yields a function
\begin{align*}
f : \trees \rSigma \to \trees \rGamma,
\end{align*}
which changes the label of every node in the input tree to the unique letter described in  (*). 
      \end{definition}
% In this section, we discuss first-order rational tree functions. These are compositions of two types of tree transformations: (a) characteristic functions of first-order query, which annotate nodes of the input tree with first-order definable information; (b) and tree homomorphisms, which replace letters in the input tree  by terms. 


% \begin{definition}[First-order rational tree functions] \label{def:forat}\ 
%     \begin{itemize}
%         \item \emph{Characteristic functions.}  Let $\rSigma$ be a finite ranked set, and let $\varphi(x)$ be a first-order formula with one free variable over the vocabulary associated to tree models over alphabet $\rSigma$. The \emph{characteristic function of $\varphi$} 
%         \begin{align*}
%             \mathrm{char}_\varphi : \trees \rSigma \to \trees (\rSigma + \rSigma)
%         \end{align*}
%         is defined to be the function which inputs a tree, and returns a tree with the same nodes, but where the labels are changed so that the first copy of $\rSigma$ is used for nodes that satisfy  $\varphi(x)$, and the second copy is used for the other nodes.
%         \item \emph{Homomorphism.} A \emph{tree homomorphism} is defined to be any function
%         \begin{align*}
%             f : \trees \rSigma \to \trees \rGamma
%         \end{align*} 
%         which applies some arity preserving function $\ranked f : \rSigma \to \tmonad \rGamma$ to each node in the input tree. 
%         \item \emph{First-order rational tree function}. A first-order rational tree function is defined to be any composition $f_1 \circ \cdots \circ f_n$, where each $f_i$ is either a characteristic function or a tree homomorphism\footnote{One can show that, without loss of generality, we can assume the composition $f_1 \circ \cdots \circ f_n$ to be such that $f_1$ is a tree homomorphism and the remaining functions are  characteristic functions. We do not need this result, so we do not prove it.
%         }.
%     \end{itemize}
%       \end{definition}

% \begin{lemma}
%     Every first-order streaming tree transducer is equivalent to one where all registers have arity 1.
% \end{lemma}

First-order tree relabellings will be used as a basic pre-processing step in the automaton model that will be defined in Section~\ref{sec:stt}.
The main  result of Section~\ref{sec:fo-translation} is the following proposition. 
\begin{proposition} \label{prop:forat}    
    Every first-order tree relabelling is derivable.
\end{proposition}       
Since the combinators include function composition, it is enough to show the above proposition for the case when $f$ is either a characteristic function, or a tree homomorphism. 

For tree homomorphisms, we use the composition of the two functions
\begin{align*}
    \ranked{
    \xymatrix{
        \tmonad \rSigma \ar[r]^{\tmonad \ranked f} & \tmonad \tmonad \rGamma \ar[r]^{\flatt_\rGamma} & \tmonad \rGamma,
    }}
\end{align*}
where $\ranked f : \rSigma \rto \tmonad \rGamma$ is an atomic function because it has a finite domain.  

The interesting case in Proposition~\ref{prop:forat} is when $f$ is a characteristic function, associated to some first-order formula with one free variable. Here, the key step is a lemma based on Schlingloff~\cite[Theorem 2.6]{schlingloff1992expressive}, which says that  over trees, every formula of first-order logic can be decomposed into simple formulas.  The lemma can be seen as a tree version of Kamp's theorem on the expressive completeness of linear temporal logic.


\begin{lemma}\label{lem:schlingloff} Every characteristic function (of a first-order query with one free variable on trees) can be decomposed as a composition of characteristic functions of the following three kinds of  query:
    \begin{enumerate}
        \item \emph{Child.} For every $\rSigma$ and  $i \in \set{1,2,\ldots}$, a query $\varphi(x)$ defined by
        \begin{align*}
            \underline t \models \varphi(x) \quad \text{iff} \quad \text{$x$ is an $i$-th child}\qquad \text{for every $t \in \trees \rSigma$}
        \end{align*}
         \item \emph{Until.} For every $\rGamma, \rDelta \subseteq \rSigma$, a query $\varphi(x)$ defined by
         \begin{align*}
             \underline t \models \varphi(x) \quad \text{iff} \quad \underbrace{\exists y\ y > x \land \rDelta(y) \land  \forall z \ (x < z < y \Rightarrow \rGamma(z))}_{\substack{\text{$x$ has a descendant $y$ with label in $\rDelta$, such that}\\ \text{all nodes strictly between $x$ and $y$ have label in $\rGamma$}}}  \qquad \text{for every $t \in  \trees \rSigma$}
         \end{align*} 
         \item \emph{Since.} For every $\rGamma, \rDelta \subseteq \rSigma$,  a query $\varphi(x)$ defined by
         \begin{align*}
             \underline t \models \varphi(x) \quad \text{iff} \quad \underbrace{\exists y\ y < x \land \rDelta(y) \land  \forall z \ (y < z < x \Rightarrow \rGamma(z))}_{\substack{\text{$x$ has a descendant $y$ with label in $\rDelta$, such that}\\ \text{all nodes strictly between $x$ and $y$ have label in $\rGamma$}}}  \qquad \text{for every $t \in  \trees \rSigma$}
         \end{align*} 
    \end{enumerate}
\end{lemma}

In light of the above lemma, in order to prove Proposition~\ref{prop:forat}, it is enough to show how to derive the three types of characteristic functions in the lemma. 