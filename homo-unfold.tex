
\subsection{Term unfolding for homogeneous inputs}
\label{sec:homo-unfold}
For a monotone function 
\begin{align*}
\alpha: \set{1,\ldots,k} \to \set{1,\ldots,k}
\end{align*}
we say that a term $ t \in \tmonad \mati k \rSigma$ is $\alpha$-homogeneous if all internal branches have twist $\alpha$. This section is devoted to proving the following lemma. 

\begin{lemma}\label{lem:homo-twist}
    Let $k \in \set{1,2,\ldots}$ and let $\alpha : \set{1,\ldots,k} \to \set{1,\ldots,k}$ be a monotone function. There is a derivable operation 
    \begin{align*}
        \ranked{f : \tmonad \mati k \rSigma \to \mati k {(\tmonad \Sigma)} }
        \end{align*}      
which coincides with term unfolding for all inputs which are $\alpha$-homogeneous.
\todo{I am not sure that this is true. More precisely, I do not see how to get this precise codomain. Maybe should it be $\mati {k'} {(\tmonad \Sigma)}$?}
\end{lemma}
\begin{proof}
We will show this lemma by induction on $k$. For that, we need to strengthen the invariant of the induction: we will show that the unfolding of $\alpha$-homogenous terms in $\ranked{\tmonad \reduce l\rSigma^k}$ is derivable. Note that Lemma~\ref{lem:homo-twist} is the particular case of this statement where $k=l$. To state this strengthening more precisely, we need to generalize the definition of unfolding and $\alpha$-homogenuity to $\ranked{\tmonad \reduce l\rSigma^k}$ terms.

For a port $i$ in an element $a/f$ of $\ranked{\reduce l \rSigma^k}$ define its \emph{twist} to be the partial function from $\set{1,\ldots,l}$ to $\set{1,\ldots,k}$
which is the composition of 
\begin{align*}
\xymatrix{
    \set{1,\ldots,l} \ar[r]^-{j \mapsto {(j,i)}} & \set{1,\ldots,l} \times \set{1,\ldots,n} \ar[r]^-{f^{-1}} & \set{1,\ldots,\arity a} \ar[r] & \set{1,\ldots,k}
},
\end{align*}
where the last function maps each port of the tuple $a$ to the coordinate of the tuple that created that port. 

We say that a term $t\in \ranked{\tmonad \reduce l \Sigma^k}$ is \emph{coherent} if the twist of every internal branche has $\set{1,\ldots,k}$ as domain. The unfold of $\ranked{\tmonad \reduce l \Sigma^k}$ coherent terms is the operation 
\begin{align*}
\ranked{\tmonad \reduce l \Sigma^k \to \reduce l (\tmonad \Sigma)^k}
\end{align*}
defined as follows
\begin{center}
TODO: {Define precisely unfold?}
\end{center}

For a monotone function 
\begin{align*}
\alpha: \set{1,\ldots,k} \to \set{1,\ldots,k}
\end{align*}
we say that a term $ t \in\ranked{\tmonad\reduce l \Sigma^k}$ is $\alpha$-homogeneous if  it is coherent and if all internal branches have as twist the function $\beta$ such that
\begin{align*}
\beta|_{\set{1,\dots,k}}=\alpha.
\end{align*}
Now we are ready to state the generalisation of our lemma.
\begin{lemma}
 Let $k, l \in \set{1,2,\ldots}$ and let $\alpha : \set{1,\ldots,k} \to \set{1,\ldots,k}$ be a monotone function. There is a computable $l'\geq l$ and a derivable operation 
    \begin{align*}
        \ranked{f : \tmonad \reduce l \rSigma^k \to \reduce {l'} {(\tmonad \Sigma)}^k }
        \end{align*}      
which coincides with term unfolding for all inputs which are $\alpha$-homogeneous.
\end{lemma}
As said earlier, we proceed by induction on $k$. 
When $k=1$, the unfolding can be derived as the composition of tghe following functions
\begin{align*}
\ranked{\tmonad \reduce l \Sigma \longrightarrow \shallowterm {\reduce l\tmonad \Sigma}{\tmonad \reduce l\Sigma} \longrightarrow \reduce 1\reduce l \tmonad \Sigma \longrightarrow \reduce l\tmonad \Sigma }
\end{align*}
To treat the inductive case, we indroduce a tool that will be useful to analyse the function $\alpha$. For a function $$\alpha: \set{1,\ldots,k} \to \set{1,\ldots,k}$$ define  \emph{its graph} as the directed graph whose set of vertices is $\set{1,\ldots,k}$, and which contains an edge $i\rightarrow j$ if $\alpha(i)=j$. Note that the out-degree of the nodes is $1.$

In the proof of the inductive case, we distinghuish two cases. The first is when the graph of $\alpha$ is not weakly connected.

The graph of alpha is weakly connected. 


\end{proof}
