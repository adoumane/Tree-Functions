
\subsection{Term unfolding for homogeneous inputs}
\label{sec:homo-unfold}
For a monotone function 
\begin{align*}
\alpha: \set{1,\ldots,k} \to \set{1,\ldots,k}
\end{align*}
we say that a term $ t \in \tmonad \mati k \rSigma$ is $\alpha$-homogeneous if all internal branches have twist $\alpha$. This section is devoted to proving the following lemma. 

\begin{lemma}\label{lem:homo-twist}
    Let $k \in \set{1,2,\ldots}$ and let $\alpha : \set{1,\ldots,k} \to \set{1,\ldots,k}$ be a monotone function. There is a derivable operation 
    \begin{align*}
        \ranked{f : \tmonad \mati k \rSigma \to \mati k {(\tmonad \Sigma)} }
        \end{align*}      
which coincides with term unfolding for all inputs which are $\alpha$-homogeneous.
\end{lemma}
\begin{proof}
We will show this lemma by induction on $k$. For that, we need to strengthen the invariant of the induction: we will show that the unfolding of $\alpha$-homogenous terms in $\ranked{\tmonad \reduce l\rSigma^k}$ is derivable. Note that Lemma~\ref{lem:homo-twist} is the particular case of this statement where $k=l$. To state this strengthening more precisely, we need to generalize the definition of unfolding and $\alpha$-homogenuity to $\ranked{\tmonad \reduce l\rSigma^k}$ terms.



If $k, l\in \set{2,3,\dots}$ such that $k\leq l$, we can embed the elements of $\ranked{\reduce l \rSigma^k}$ in 
$\ranked{\mati l {(\rSigma+\set{0})}}$ by adding $l-k$ copies of the nullary element $0$ to the tensor product, as illustrated below:
\begin{center}

\end{center}  
Hence, we can talk about the twist of an element of $\ranked{\reduce l \Sigma^k}$, and we can embed every element of $\ranked{\tmonad \reduce l \Sigma^k}$  into $\ranked{\tmonad \mati l \Sigma}$.

We define the unfolding of a term $t$ in $\ranked{\tmonad \reduce l \Sigma^k}$ as its unfolding when seen as an element of $\ranked{\tmonad \mati l \Sigma}$. Note that the unfolding of a element of $\ranked{\tmonad \reduce l \Sigma^k}$ is an element of $\ranked{\reduce l(\tmonad (\Sigma+\set{0}))^k}$.

We will be interested in a particular case of terms: the \emph{coherent} ones. We say that a term $t\in \ranked{\tmonad \reduce l \Sigma^k}$ is \emph{coherent} if the twist of every internal branche of $t$ seen as an element of $\ranked{\tmonad \mati l \Sigma}$ has $\set{1,\ldots,k}$ as domain. Note that the unfolding of a coherent element of $\ranked{\tmonad \reduce l \Sigma^k}$ is an element of $\ranked{\reduce l(\tmonad \Sigma)^k}$.


For a monotone function 
\begin{align*}
\alpha: \set{1,\ldots,k} \to \set{1,\ldots,k}
\end{align*}
we say that a term $ t \in\ranked{\tmonad\reduce l \Sigma^k}$ is $\alpha$-homogeneous if  it is coherent and if all internal branches have as twist the function $\beta$ such that
\begin{align*}
\beta|_{\set{1,\dots,k}}=\alpha.
\end{align*}
Now we are ready to state the generalisation of our lemma.
\begin{lemma}
 Let $k, l \in \set{1,2,\ldots}$ and let $\alpha : \set{1,\ldots,k} \to \set{1,\ldots,k}$ be a monotone function. There is a  derivable operation 
    \begin{align*}
        \ranked{f : \tmonad \reduce l \rSigma^k \to \reduce {l} {(\tmonad \Sigma)}^k }
        \end{align*}      
which coincides with term unfolding for all inputs which are $\alpha$-homogeneous.
\end{lemma}
As said earlier, we proceed by induction on $k$. When $k=1$, unfolding coincides with the basic function
\begin{align*}
\ranked{\mathsf{unfold} : \tmonad \reduce l \rSigma \to \shallowterm{\reduce {l} {\tmonad \Sigma}} {\tmonad \reduce l \rSigma}  }
\end{align*}
We only need to adjust the output type, that is to get rid of the $\ranked{\tmonad \reduce l \rSigma}$ after the ``$\cdot$'' of the shallow term. Fisrt, we lift the basic function 
\begin{align*}
\ranked{\composeterm: \tmonad \reduce l \rSigma \to 
        1 + \shallowterm {\reduce l \rSigma} {\tmonad \reduce l \rSigma}}          
\end{align*} 
and the identity function 
\begin{align*}
\ranked{ \reduce {l} {\tmonad \Sigma}\to \reduce {l} {\tmonad \Sigma}}          
\end{align*} 
to the the shallow combinator. \todo{Je ne sais pas comment finir.}

Let us treat the inductive case. For that, we indroduce a tool that will be useful to analyse the function $\alpha$. For a function $$\alpha: \set{1,\ldots,k} \to \set{1,\ldots,k}$$ define  \emph{its graph} as the directed graph whose set of vertices is $\set{1,\ldots,k}$, and which contains an edge $i\rightarrow j$ if $\alpha(i)=j$. Note that the out-degree of the nodes is $1.$

In the proof of the inductive case, we distinghuish two cases. The first is when the graph of $\alpha$ is not weakly connected. In this case, by monotonicity of $\alpha$, we can find $m\in [1,k[$ such that $\alpha(\set{1,m})\subseteq\set{1,m}$ and $\alpha(\set{m+1,k})\subseteq\set{m+1,k}$. The idea is then to create two copies of the original tree: in the first one we keep only the first $m$ elements of the tensor product of each node, and in the second one we keep the last $k-m$ copies. Then we unfold these terms by applying the induction hypothesis, $m$ and $k-m$ being strictly smaller than $k$, and  finally we gather them to obtain the  unfolding of the original term. \todo{Adapt the name of the combinators.}
The following derivation allows us to derive the function  
\begin{align*}
\ranked{f_1:\tmonad \reduce l \Sigma^k \to \reduce l (\tmonad \Sigma)^m}
\end{align*}
which creates a copy of the original term, keeping only the first $m$ elements of the tensor product, then applies the induction hypothesis to the obtained term.
\begin{align*}
\begin{prooftree}
\Infer{0}{\ranked{\Sigma^k\to \Sigma^m\otimes \Sigma^{k-m}}}
\Hypo{\text{\small Projection}}
\Infer{1}[]{\ranked{\Sigma^m\otimes \Sigma^{k-m} \to \reduce 1 \Sigma^m}}
\Infer{2}[$(\circ)$]{\ranked{\Sigma^k \to \reduce 1 \Sigma^m}}
\Infer{1}[$(\reduce l)$]{\ranked{\reduce l \Sigma^k \to \reduce l \reduce 1 \Sigma^m}}
\Infer{0}[]{\ranked{\reduce l \reduce 1 \Sigma^m \to \reduce l \Sigma^m}}
\Infer{2}[$(\circ)$]{\ranked{\reduce l \Sigma^k \to \reduce l \Sigma^m}}
\Infer{1}[$(\tmonad)$]{\ranked{\tmonad \reduce l \Sigma^k \to \tmonad \reduce l \Sigma^m}}
\Hypo{\text{\small Induction Hypothesis}}
\Infer{1}[]{\ranked{\tmonad \reduce l \Sigma^m \to \reduce l (\tmonad \Sigma)^m}}
\Infer{2}[$(\circ)$]{\ranked{\tmonad \reduce l \Sigma^k \to \reduce l (\tmonad \Sigma)^m}}
\end{prooftree}
\end{align*}
Similarly, we can derive the function
\begin{align*}
\ranked{f_2:\tmonad \reduce l \Sigma^k \to \reduce l (\tmonad \Sigma)^{k-m}}
\end{align*}
which creates a copy of the original term, keeping only the last $k-m$ elements of the tensor product, then applies the induction hypothesis to the obtained term.
Finally, we combine $f_1$ and $f_2$ as follows. First we duplicate the original tree using the following combinator:
\begin{align*}
\ranked{\tmonad \reduce l \Sigma^k\to \reduce 2 (\tmonad \reduce l \Sigma^k\otimes \tmonad \reduce l \Sigma^k)}
\end{align*}
Then we compose it with the following function, which applies $f_1$ to the first copy and $f_2$ to the second one
\begin{align*}
\begin{prooftree}
\Infer{0}[]{\ranked{f_1:\tmonad \reduce l \Sigma^k\to\reduce l (\tmonad \Sigma)^m}}
\Infer{0}[]{\ranked{ \tmonad \reduce l \Sigma^k\to \reduce l(\tmonad \Sigma)^{k-m}}}
\Infer{2}[$(\ranked{\otimes})$]{\ranked{\tmonad \reduce l \Sigma^k\otimes \tmonad \reduce l \Sigma^k\to\reduce l(\tmonad  \Sigma)^m\otimes \reduce l(\tmonad  \Sigma)^{k-m}}}
\Infer{1}[$(\reduce 2)$]{\ranked{\reduce 2 (\tmonad \reduce l \Sigma^k\otimes \tmonad \reduce l \Sigma^k)\to \reduce 2 ( \reduce l (\tmonad  \Sigma)^m\otimes \reduce l (\tmonad \reduce\Sigma)^{k-m})}}
\end{prooftree}
\end{align*}
Finally, we compose the result with the following fucntion 
\begin{align*}
\begin{prooftree}
\Infer{0}[]{\ranked{f_1:\tmonad \reduce l \Sigma^k\to\reduce l (\tmonad \Sigma)^m}}
\Infer{0}[]{\ranked{ \tmonad \reduce l \Sigma^k\to \reduce l(\tmonad \Sigma)^{k-m}}}
\Infer{2}[$(\ranked{\otimes})$]{\ranked{\tmonad \reduce l \Sigma^k\otimes \tmonad \reduce l \Sigma^k\to\reduce l(\tmonad  \Sigma)^m\otimes \reduce l(\tmonad  \Sigma)^{k-m}}}
\Infer{1}[$(\reduce 2)$]{\ranked{ \reduce 2 ( \reduce l (\tmonad  \Sigma)^m\otimes \reduce l (\tmonad \reduce\Sigma)^{k-m})\to \reduce l (\tmonad \Sigma)^k}}
\end{prooftree}
\end{align*}
Now consider the case where the graph of $\alpha$ is weakly connected. 


\end{proof}
%We proceed by induction on $k$. When $k=1$, the unfolding coincides with the basic function 
%\begin{align*}
%\ranked{\distrtf : \tmonad \reduce 1 \Sigma \to {\reduce 1 \tmonad \Sigma}}
%\end{align*}