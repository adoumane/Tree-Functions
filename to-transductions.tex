\newcommand{\Root}[1]{\mathsf{root}_{#1}}
\newcommand{\Port}[1]{\mathsf{port}_{#1}}
\newcommand{\Interface}[1]{\mathsf{Interface}_{#1}}
\section{Atomic functions and combinators to first-order transductions}
\label{sec:to-transductions}
In this section, we prove the right-to-left implication in Theorem~\ref{thm:main}, namely that if $\rSigma,\rGamma$ are finite ranked sets, then for every derivable function $\ranked {f : \tmonad \rSigma \to \tmonad \rGamma}$, its restriction to trees $f : \trees \rSigma \to \trees \rGamma$
is a first-order tree-to-tree transduction. The proof is a simple induction on the derivation of  $\ranked f$. In the induction, we must deal with functions that manipulate types that are more complex than trees, e.g.~tensor pairs of terms of coproducts.  To operate on such types with first-order transductions, we need to show how such types can be modelled as relational structures.


% We do this by assigning to each element $a$ in a type $\rSigma$ an associated model $\underline a$. 
% \newcommand{\structa}{\mathfrak A}
% When defining the models below, we use two types of disjoint union. 
% \begin{itemize}
%     \item For two relational structures $\structa_1,\structa_2$ over relational vocabularies $\ranked{\tau_1},\ranked{\tau_2}$, define $\structa_1 + \structa_2$ to be the  relational structure over vocabulary $\ranked{\tau_1 + \tau_2}$ defined by taking the disjoint union of the structures, and interpreting the relations of $\structa_i$ using the copy $\ranked{\tau_i}$ of the vocabulary.
%     \item For a family of $\set{\structa_i}_{i \in I}$ of relational structures over a common vocabulary $\ranked \tau$, define $\coprod_i \structa_i$ to be the structure over $\ranked \tau$ disjoint union of the structures, with all components using the common vocabulary.
% \end{itemize}


\begin{definition}[Associated models for terms, pairs, co-pairs and tensors.] \label{def:type-model} To each type  $\rSigma$ be a vocabulary, called the \emph{vocabulary of $\rSigma$}, and a map 
    \begin{align*}
        a \in \rSigma \qquad \mapsto \qquad \underbrace{\underline a \in \text{models over the  vocabulary of  $\rSigma$}}_{\text{associated model of $a$}}.
    \end{align*}
    Furthermore, for each $a \in \rSigma$ we  distinguish a  sequence (whose length is the arity of $a$) of distinguished elements in $\underline a$, which are called the ports $\underline a$.   The definitions are by induction on the structure of $\rSigma$, as given below.
    \begin{itemize}
        \item \emph{Finite ranked sets.} Elements of a ranked set   $\rSigma =  \set{a_1,\ldots,a_k}$ are modelled a using  vocabulary which has unary relations $A_1,\ldots,A_k$ and $P_1,\ldots,P_m$ where $m$ is the maximal arity of elements in $\rSigma$. 
        For $a \in \rSigma$ of arity $n$, the  universe of $\underline a$ is $\set{0,1,\ldots,n}$, with the ports being $1,\ldots,n$. 
            The  relation $P_i$  is interpreted as $\set i$ when $i \in \set{1,\ldots,n}$ and as the empty set otherwise. The relation $a_i$ is interpreted as $\set 0$ when $a = a_i$ and as the empty set otherwise. 
        \item \emph{Coproduct.}  Elements of the coproduct $\ranked{\Sigma_1 + \Sigma_2}$ are modelled using the disjoint union of the vocabularies of $\ranked{\Sigma_1}$ and $\ranked{\Sigma_2}$. 
            If an element of the coproduct comes from $\ranked{\Sigma_1}$, then its associated model is defined as for the type $\ranked{\Sigma_1}$, with  the remaining relations from the vocabulary of   $\ranked{\voctype \Sigma_2}$ interpreted   as empty sets. The definition is analogous for  elements from $\ranked{\Sigma_2}$. 
        \item \emph{Tensor product.}   Tensor pairs in   $\ranked{\Sigma_1 \otimes \Sigma_2}$ are modelled
        using the disjoint union of the vocabularies of $\ranked{\Sigma_1}$ and $\ranked{\Sigma_2}$. 
            For  $\tensorpair{a_1,a_2}$, the associated model is    the disjoint union $\underline{a_1} + \underline {a_2}$, with the relations of $\underline {a_1}$ using the    $\voctype{\ranked{\Sigma_1}}$  part of the vocabulary, and the relations of $\underline {a_2}$ using the the    $\voctype{\ranked{\Sigma_1}}$  part of the vocabulary. 
            If $n_1$ is the arity of $a_1$, then the first $n_1$ ports are inherited from  $\underline {a_1}$ and the remaining ports are inherited from  $\underline {a_2}$.
        \item \emph{Cartesian product.}   Cartesian pairs in  $\ranked{\Sigma_1 \otimes \Sigma_2}$ are modelled  using the disjoint union of the vocabularies of $\ranked{\Sigma_1}$ and $\ranked{\Sigma_2}$, plus an extra binary relation $R$.  
            The model associated to a Cartesian pair   $ (a_1,a_2)$,  is      the disjoint union $\underline{a_1} +  \underline {a_2}$, in the same sense as in the previous item for $\otimes$,  with the binary relation $R$ interpreted as 
                \begin{align*}
                    \set{(\text{$i$-th port of $\underline {a_1}$}, \text{$i$-th port of $\underline a_2$}): i \in \set{1,\ldots,\text{arity of $(a_1,a_2)$}}} 
                \end{align*}
                The ports are inherited from   $\underline {a_1}$.
        \item \emph{Folding.}   For $k \in \set{1,2,\ldots}$, elements of   $\reduce k \rSigma$ are modelled using the  vocabulary of $\rSigma$ plus two extra binary relations $<$ and $R$. If $a \in \rSigma$ has arity $nk$, then the model associated to $a/f$ -- which has arity $n$ --   is obtained from  $\underline{a}$ by adding a copy of the model
                \begin{align*}
                (\set{1,\ldots,n}, <),
                \end{align*}
        whose elements are used as the ports, and interpreting the binary relation $R$ as
        \begin{align*}
        \set{(\text{$i$-th port of $\underline a$},f(i)) : i \in \set{1,\ldots,nk}}
        \end{align*}
                
                
        \item \emph{Terms.}  Terms in $\tmonad \rSigma$ are modelled using vocabulary of $\rSigma$ extended with two fresh binary relations $\anceord$ and $\portord$. 
          Let $t \in \tmonad \rSigma$. Consider the disjoint union of models
            \begin{align}\label{eq:non-port}
                 \coprod_{x \in \text{non-port nodes in $t$}} \underline{a(x)},
            \end{align}
         where  $\underline a(x)$ is the model over vocabulary $\voctype \rSigma$ that  is defined by induction assumption.   In the above  disjoint union, the same vocabulary, namely the vocabulary of $\rSigma$,  is used  for all parts of the disjoint union. Next, consider  the model
            \begin{align}\label{eq:ports}
            (\set{1,\ldots,n}, \portord)
            \end{align}
            where $\portord$ is the natural ordering on $\set{1,\ldots,n}$. 
            The model of $t$ is defined by taking the disjoint union of the models in~\eqref{eq:non-port} and~\eqref{eq:ports}, and defining the binary relation $\anceord$ as follows.
            \begin{center}
                (fill in the definition of $\anceord$)
            \end{center}
            % It  consists of all pairs $(u,v)$ such that $u$ is the $i$-th port of $\underline{a(x)}$  some non-port node $x$, and $v$ belongs to $\underline{a(y)}$ for some node (possibly a port) $y$ which is a descendant (not necessarily proper) of the $i$-th child of $x$. The binary relation $\portletter$ is the total order 
            The ports are taken from the structure~\eqref{eq:ports}.
    \end{itemize}
\end{definition}

The above  definition creates a certain ambiguity for trees, because if $t$ is a tree over a finite ranked set $\rSigma$, then $\underline t$ can be understood in two ways: as per  Definition~\ref{def:tree-model} for trees, or as per Definition~\ref{def:type-model} when $t$ is viewed as a special case of a term $t \in \tmonad \rSigma$. Since we only use first-order transductions to transform relational structures,  this ambiguity is not a problem, because one can easily define first-order transductions which map one definition of $\underline t$ to the other.

The following lemma immediately yields the right-to-left implication in Theorem~\ref{thm:main}. It proof, which is a straightforward case analysis of all atomic functions and combinators, is in the appendix. 




  
\begin{proposition}\label{prop:to-logic} If $f : \rSigma \to \rGamma$ is derivable, then there is a first-order transduction $g$ 
    which makes the following diagram commute
    \begin{align*}
        \xymatrix@C=3cm{
            \rSigma \ar[d]_{a \mapsto \underline a}\ar[r]^f &  \rGamma \ar[d]^{a \mapsto \underline a} \\
            \text{models over the vocabulary of  $ \rSigma$} \ar[r]_g &  \text{models over the vocabulary of  $ \rGamma$}.
        } 
    \end{align*}
    
\end{proposition}



