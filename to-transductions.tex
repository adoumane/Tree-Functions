\newcommand{\Root}[1]{\mathsf{root}_{#1}}
\newcommand{\Port}[1]{\mathsf{port}_{#1}}
\newcommand{\Interface}[1]{\mathsf{Interface}_{#1}}
\section{To transductions}
\label{sec:to-transductions}
In this section, we prove the bottom-up inclusion in Theorem~\ref{thm:main}:
\begin{align*}
    \text{first-order tree-to-tree interpretations} \qquad \supseteq \qquad \text{first-order tree functions}
\end{align*}
Since first-order tree functions are defined as tree restrictions of term-to-term functions, we need to work with first-order transductions that transforms terms (and elements of other types, such as pairs of terms, terms of terms, etc.). We begin by describing how elements of a type are encoded as models. 
\\

We define by induction on the type $\Sigma$, the vocabulary $\hat \Sigma$ and the encoding 
 \begin{align*}
     t \in \Sigma \quad \mapsto \quad \underline t \in \text{models over $\hat \Sigma$}
 \end{align*}
 which represents elements $t$ of $\Sigma$ as relational structures over $\hat \Sigma$.
 \\
 
 For every type $\Sigma$, the vocabulary $\hat \Sigma$ will contain at least the following set of predicates, called $\sf{Interface}_\Sigma$, which includes:
 \begin{itemize}
 \item a unary predicate $\Root{\Sigma}$;
 \item a binary predicate $\leq_\Sigma$ ordering the nodes of the structure;
  \item a unary predicate $\Port{\Sigma}$ selecting the "free variables" of a term; 
 \item a binary predicate $\prec_\Sigma$ ordering these ports from right to left.  
\end{itemize}  

For a $\hat\Sigma$ structure $t$, we call the \emph{root of $t$} the node satisfying the predicate $\Root{\Sigma}$, and we call {the $i^{th}$ port} the node satisfying $\Port\Sigma$, which is in the $i^{th}$ position wrt. $\prec_\Sigma$. 
\\

The rest of the predicates of $\hat \Sigma$ is inherited by induction from its sub-types, as described in the following.\\

\noindent{\bf Atomic types.} 
   \begin{itemize}
   \item Let $F$ be a finite ranked set.  We set $\hat F$ to be $\Interface{F}$ together with a unary predicate $a$ for each symbol $a$ of $F$. 
            
            If $a$ is an $n$-ary element of $F$, we define its corresponding structure $\underline a$ as follows:
            \begin{itemize}
            \item Its univers contains $n+1$ nodes named $0,1\dots,n$;
            \item node $0$ satisfies the predicate $a$ and $\Root F$;
            \item For every $i\in[1,n]$, node $i$ satisfies $\Port F$  and $0\leq_F i$; and for every $i, j \in[1,n]$ such that $i< j$, we have that $i\prec_F j$.
\end{itemize}             
            \begin{center}
            Picture ?
            \end{center}
            \item For the atomic type $\llone$, we set $\hat \llone$ to be $\Interface \llone$. The structure $\underline a$ of an element $a$ of $\llone$ is exactly as above (exept that we ignore the predicate $a$).
        \end{itemize}
    {\bf Type constructors.}
    
    If $\Sigma \odot \Gamma$ is a type where $\odot \in \{\times, \otimes,+\}$, we set $\widehat{\Sigma\odot\Gamma}$ to be $\hat{\Sigma} \uplus \hat{\Gamma} \uplus \Interface {\Sigma\odot\Gamma}$. Let us describe the encoding of an element of $\Sigma \odot \Gamma$ as a structure over $\widehat{\Sigma\odot\Gamma}$.
     \begin{itemize}
\item Let $(t,u)$ be an $n$-ary element of $\Sigma \times \Gamma$. Thus $t$ and $u$ are elements from $\Sigma, \Gamma$, respectively, whose arities are both $n$. We set:
        \begin{enumerate}
       \item the universe of $\underline{(t,u)}$ is the disjoint union of the universes of $\underline{t}$ and $\underline{u}$ together with $n+1$ new nodes named $0,1,\dots, n$;
       \item the interpretation of the predicates from $\hat \Sigma$ and $\hat \Gamma$ are inherited from $\underline{t}$ and $\underline {u}$ respectively;
       \item node $0$ satisfies $\Root{\Sigma\times\Gamma}$ and for every $i,j \in[1,n]$ such that $i<j$, we set $i \prec_{\Gamma\times\Sigma} j$;
       \item the relation $\leq_{\Gamma\times\Sigma}$ contains   $\leq_{\Gamma}$ and $\leq_{\Sigma}$ together with:
\begin{align*}
0\leq_{\Gamma\times\Sigma} r \quad\text{ if } \quad \Root{\Sigma}(r) \quad\text{ or }\quad \Root{\Gamma}(r);\qquad\\
\forall i\in[1,n],\quad p \leq_{\Gamma\times\Sigma} i \text{ where } 
p \text{ is the } i^\text{th} \text{ port of } t \text{ (resp.  $u$)}.
\end{align*}  
        \end{enumerate}
        \begin{center}
        Picture
        \end{center}        
\item Let $(t,u)$ be an $n$-ary element of $\Sigma\otimes \Gamma$. Thus $t$ is an element of $\Sigma$ of arity $n_1$ and $u$ is an element of $\Gamma$ of arity $n_2$ such that $n_1+n_2=n$. The universe of $\underline{(t,u)}$ and the interpretation of the predicates from $\hat \Sigma$ and $\hat \Gamma$ are exctly as in items 1 and 2 above. As in item 3, node $0$ satisfies $\Root{\Sigma\otimes \Gamma}$ and $\prec_{\Sigma\otimes \Gamma}$
       is the order on natural numbers. Item 4 become:
       \begin{enumerate}
       \item[4'.] The relation $\leq_{\Gamma\otimes\Sigma}$ contains   $\leq_{\Gamma}$ and $\leq_{\Sigma}$ together with:
       \begin{align*}
0\leq_{\Gamma\otimes\Sigma} r &\quad\text{ if } \quad \Root{\Sigma}(r) \quad\text{ or }\quad \Root{\Gamma}(r);\qquad\\
\forall i\in[1,n_1] & \quad p \leq_{\Gamma\otimes\Sigma} i \text{ where } 
p \text{ is the } i^\text{th} \text{ port of } t;\\
\forall i\in[1,n_2]&\quad p \leq_{\Gamma\otimes\Sigma} n_1+i \text{ where } 
p \text{ is the } i^\text{th} \text{ port of } u.
\end{align*} 
\end{enumerate}        
 \item Let $t$ be an $n$-ary element of $\Sigma+\Gamma$. Then, it is either an element of $\Sigma$ or an element of $\Gamma$. Suppose wlog. that $t$ is an element of $\Sigma$. We set $\underline{t}_\Sigma$ to be the interpretation of $t$ seen as an element of $\Sigma$. The universe of $\underline{t}$ is the universe of $\underline{t}_\Sigma$ together with $n+1$ new nodes named $0,1,\dots,n$. The interpretation of the predicates from $\hat \Sigma$ are inherited from $\underline{t}_\Sigma$ as usual. Node $0$ satisfies $\Root{\Sigma+\Gamma}$, the nodes $[1,n]$ satisfy $\Port{\Sigma+\Gamma}$, and they are ordered by $\prec_{\Sigma+\Gamma}$ respecting the order of natural numbers. The relation $\leq_{\Sigma+\Gamma}$ extends $\leq_\Sigma$ with the following pairs: $0\leq r$ if $\Root{\Sigma}(r)$, and $p\leq_{\Sigma+\Gamma} i$ if $p$ is the $i^{th}$ port of $\underline{t}_\Sigma$.
        
        \item The vocabulary $\widehat {\trees \Sigma}$, corresponding to the type $\trees \Sigma$, is $\Interface{\Sigma}\uplus\{\sqsubseteq\}$. 
    
 Let $t$ be an $n$-ary tree of $\trees \Sigma$. Let $V_\Sigma$ be  the set of its $\Sigma$-nodes and $V_\portletter$ be the set of its $\portletter$-nodes. 
\begin{enumerate}
\item The universe of $\underline{t}$ is $\{\underline{v}| v\in V_\Sigma\}\uplus V_\portletter\uplus\{0\}$.
\item The interpretation of the predicates from $\hat \Sigma$ are inherited from the structures $\underline{v}$, where $v\in V_\Sigma$.
\item Node $0$ satisfies $\Root{\trees\Sigma}$. The nodes $V_\portletter$ satisfy $\Port{\trees\Sigma}$, and they are ordered by $\prec_{\trees\Sigma}$ respecting their order in $t$. 
\item The relation $\leq_{\trees\Sigma}$ extends $\leq_\Sigma$ with the following pairs:      
\begin{itemize}
\item $0\leq_{\trees\Sigma} r$ if $r$ is the root of $\underline{v}$ and $v$ is the root of $t$.
\item $ p\leq_{\trees\Sigma} r$ if there is $v,w \in V_\Sigma$ such that:
\begin{itemize}
\item $v$ is the $i^{th}$ son of $w$; 
\item $r$ is the root of $\underline{w}$ and $p$ is the $i^{th}$ port of $\underline{v}$.
\end{itemize}
\item $p\leq_{\trees\Sigma} w$ if there is $v\in V_\Sigma$ and $w\in V_\portletter$ such that:
\begin{itemize}
\item $w$ is the $i^{th}$ son of $v$ in $t$;
\item $p$ is the $i^{th}$ port of $\underline{v}$.
\end{itemize}
\end{itemize}
\end{enumerate}
\begin{center}
Picture
\end{center}
 \end{itemize}  

\begin{definition} Let $\Sigma, \Gamma\in \Tt$. We say that a function $f:\Sigma\to \Gamma$ is \emph{definable by an FO-transduction} if there is some FO-transduction $\varphi$ which makes the following diagram commutes:
\[\begin{array}{lll}
\Sigma& \stackrel{t\mapsto \underline{t}}{\xrightarrow{\hspace*{3cm}}} & \text{structures over }\hat \Sigma \\
f&&\varphi\\
\Gamma& \stackrel{t\mapsto \underline{t}}{\xrightarrow{\hspace*{3cm}}}  & \text{structures over }\hat \Gamma 
\end{array}\]
\end{definition}

\begin{proposition}
Let $\Gamma, \Sigma \in \Tt$. If $f:\Sigma\to\Gamma$ is a first-order term function, then it is definable by an FO-transduction. 
\end{proposition}

\begin{pr}
The proof is by induction, following the defintion of first-order term functions. The atomic functions $\llone$, constant functions, projection, co-projection and distributions are easily definable by FO-transductions.  
We prove now that are also definable by FO-ransductions. 
\\

\begin{enumerate}
\item $\sf{Flat}$. Given a tree $t$ of type $\trees\trees\Sigma$, the fucntion $\sf{Flat}(t)$ can be implemented using an FO-transduction as follows: the copying constant is 1, the universe formula is 
\[\varphi_1(x) = \neg (\Root{\trees\Sigma}(x) \vee \Port{\trees\Sigma}(x) )  \]
For every relation symbol $R\in\hat \Sigma$ , we conserve the same interpretation:
\[\varphi_R^{1,1}(x,y)=R(x,y) \text{ if $R$ is binary }
\quad \text{ and } \quad\varphi_R^{1}(x)=R(x) \text{ if $R$ is unary }\] 
For the other symbols of $\widehat\trees\Sigma$, that is $\Root{\trees\Sigma},\Port{\trees\Sigma}, \prec_{\trees\Sigma}, \leq_{\trees\Sigma}$, we "shift by one $\trees$" their interpretation, more precisely, if $R\in\{\Root{}, \Port{}, \prec, \leq\}$, we set:
\[ \varphi^1_{R_{\trees\Sigma}}(x) = R_{\trees\trees\Sigma}(x) \text{ if $R_{\trees\Sigma}$ is unary } \]
\[\varphi^{1,1}_{R_{\trees\Sigma}}(x,y) = R_{\trees\trees\Sigma}(x,y) \text{ if it is binary. }\] 
\item $\sf{Block}$. We show how to simulation the function $\sf{Block}:\trees (\Sigma + \Gamma) \to \trees (\trees \Sigma + \trees \Gamma)$ by an FO-transduction. 
\end{enumerate} 

 \end{pr}