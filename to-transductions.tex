\newcommand{\Root}[1]{\mathsf{root}_{#1}}
\newcommand{\Port}[1]{\mathsf{port}_{#1}}
\newcommand{\Interface}[1]{\mathsf{Interface}_{#1}}
\section{To transductions}
\label{sec:to-transductions}
In this section, we prove the bottom-up inclusion in Theorem~\ref{thm:main}:
\begin{align*}
    \text{first-order tree-to-tree interpretations} \qquad \supseteq \qquad \text{first-order tree functions}
\end{align*}
Since first-order tree functions are defined as tree restrictions of term-to-term functions, we need to work with first-order transductions that transforms terms (and elements of other types, such as pairs of terms, terms of terms, etc.). We begin by describing how elements of a type are encoded as models. 
\\

We define by induction on the type $\Sigma$, the vocabulary $\underline \Sigma$ and the encoding 
 \begin{align*}
     t \in \Sigma \quad \mapsto \quad \underline t \in \text{models over $\underline \Sigma$}
 \end{align*}
 which represents elements $t$ of $\Sigma$ as relational structures over $\underline \Sigma$.
 \\
 
 For every type $\Sigma$, the vocabulary $\underline \Sigma$ will contain at least the following set of predicates, denoted $\mathcal{I}_\Sigma$ and called \emph{interface vocabulary}, which includes:
 \begin{itemize}
 \item a unary predicate $\Root{\Sigma}$;
% \item a binary predicate $\leq_\Sigma$ ordering the nodes of the structure;
  \item a unary predicate $\Port{\Sigma}$ selecting the "free variables" of a term; 
 \item a binary predicate $\prec_\Sigma$ ordering these ports from right to left.  
\end{itemize}  

For a $\underline\Sigma$ structure $t$, we call the \emph{root of $t$} the node satisfying the predicate $\Root{\Sigma}$, and we call {the $i^{th}$ port} the node satisfying $\Port\Sigma$, which is in the $i^{th}$ position wrt. $\prec_\Sigma$. 
\\

%The rest of the predicates of $\underline \Sigma$ will be defined by induction. But first, let us provide two constructions which will allow us to combine structures to get more complex ones:
%\begin{enumerate}
%\item {\bf Disjoint union.} Let $t_i, i\in I$ be a collection of structures over the vocabularies $V_i, i\in I$ respectively. We set $\Pi_{i\in I} t_i$
%to be the structure defined as follows:
%\begin{itemize}
%\item Its universe is the disjoint union of the universes of $t_i, i\in I$.
%\item Its vocabulary is the disjoint union of the vocabularies $V_i, i\in I$ together with the unary predicates $C_i, i\in I$.
%\item Say how to interpret the symbols. 
%\end{itemize}  
%\item {\bf Disjoint sum.}
%Let $t_i, i\in I$ be a collection of structures over a vocabulary $V$ respectively. We set $\amalg_{i\in I} t_i$
%to be the structure defined as follows:
%\begin{itemize}
%\item Its universe is the disjoint union of the universes of $t_i, i\in I$.
%\item Its vocabulary is the disjoint union of the vocabulary $V$ together with the binary predicate $\sim$. 
%\item Say how to interpret the symbols.
%\end{itemize}  
%\end{enumerate}  

We construct the vocabulary $\underline{\Sigma}$ and the structures $\underline{t}$ for every $t\in\Sigma$ by induction as follows:
 

\noindent{\bf Atomic types.} 
   \begin{itemize}
   \item Let $F$ be a finite ranked set.  We set $\underline F$ to be $\mathcal{I}_{F}$ together with a unary predicate $a$ for each symbol $a$ of $F$. 
            
            \noindent If $a$ is an $n$-ary element of $F$, we define its corresponding structure $\underline a$ as follows:
            \begin{enumerate}
            \item Its univers contains $n+1$ nodes named $0,1\dots,n$;
            \item node $0$ satisfies the predicate $a$ and $\Root F$;
            \item For every $i\in[1,n]$, node $i$ satisfies $\Port F$; and for every $i, j \in[1,n]$ such that $i< j$, we have that $i\prec_F j$.
            \end{enumerate}             
            \begin{center}
            Picture ?
            \end{center}
            \item For the atomic type $\llone$, we set $\underline \llone$ to be $\mathcal{I}_ \llone$. The structure $\underline a$ of an element $a$ of $\llone$ is exactly as above (exept that we ignore the predicate $a$).
        \end{itemize}
    {\bf Type constructors.}
    
     \begin{itemize}
\item {\bf Cartesian product.} The vocabulary $\underline{\Sigma\times\Gamma}$ is the disjoint union of $\mathcal{I}_{\Sigma\times\Gamma}$, $\underline{\Sigma}$, $\underline{\Gamma}$ and two unary predicates $L_{\Sigma\times\Gamma}$ and $R_{\Sigma\times\Gamma}$.

 Let $(t,u)$ be an $n$-ary element of $\Sigma \times \Gamma$. We set:
        \begin{enumerate}
       \item the universe of $\underline{(t,u)}$ is the disjoint union of the universes of $\underline{t}$ and $\underline{u}$ together with $n+1$ new nodes named $0,1,\dots, n$;
       \item the interpretation of the predicates from $\underline \Sigma$ and $\underline \Gamma$ are inherited from $\underline{t}$ and $\underline {u}$ respectively;
       \item node $0$ satisfies $\Root{\Sigma\times\Gamma}$ and for every $i,j \in[1,n]$ such that $i<j$, we set $i \prec_{\Gamma\times\Sigma} j$;
        \item a node satisfies $L_{\Sigma\times\Gamma}$ (resp. $R_{\Sigma\times\Gamma}$) if it comes from $\underline{t}$ (resp.  $\underline{u}$).  
        \end{enumerate}
 \item {\bf Disjoint union.} The vocabulary $\underline{\Sigma+\Gamma}$ is the disjoint union of $\mathcal{I}_{\Sigma+\Gamma}$, $\underline{\Sigma}$, $\underline{\Gamma}$ and two unary predicates $L_{\Sigma+\Gamma}$ and $R_{\Sigma+\Gamma}$.

Let $t$ be an $n$-ary element of $\Sigma+\Gamma$. Then, it is either an element of $\Sigma$ or an element of $\Gamma$. Suppose wlog. that $t$ is an element of $\Sigma$. We set $\underline{t}_\Sigma$ to be the interpretation of $t$ seen as an element of $\Sigma$. The universe of $\underline{t}$ is the universe of $\underline{t}_\Sigma$ together with $n+1$ new nodes named $0,1,\dots,n$. The interpretation of the predicates from $\underline \Sigma$ are inherited from $\underline{t}_\Sigma$ as usual. Node $0$ satisfies $\Root{\Sigma+\Gamma}$, the nodes $[1,n]$ satisfy $\Port{\Sigma+\Gamma}$, and they are ordered by $\prec_{\Sigma+\Gamma}$ respecting the order of natural numbers. All the nodes satisfy $L_{\Sigma+\Gamma}$ (they would all satisfy $R_{\Sigma+\Gamma}$ if $t$ was an element of $\Gamma$).
        
        \item {\bf Terms.} The vocabulary $\underline {\trees \Sigma}$ is $\mathcal{I}_{\trees\Sigma}+\underline{\Sigma}+\{\sf{reach}_{\trees \Sigma}, \sf{Sib}_{\trees \Sigma}, \sim_{\trees \Sigma}\}$ where . 
    
 Let $t$ be an $n$-ary tree of $\trees \Sigma$. Let $V_\Sigma$ be  the set of its $\Sigma$-nodes and $V_\portletter$ be the set of its $\portletter$-nodes. 
\begin{enumerate}
\item The universe of $\underline{t}$ is $\{\underline{v}| v\in V_\Sigma\}+ V_\portletter+\{0\}$.
\item The interpretation of the predicates from $\underline \Sigma$ are inherited from the structures $\underline{v}$, where $v\in V_\Sigma$.
\item Node $0$ satisfies $\Root{\trees\Sigma}$. The nodes $V_\portletter$ satisfy $\Port{\trees\Sigma}$, and they are ordered by $\prec_{\trees\Sigma}$ respecting their order in $t$. 
\item The relation $\leq_{\trees\Sigma}$ 
\end{enumerate}
\begin{center}
Picture
\end{center}
 \end{itemize}  

\begin{definition} Let $\Sigma, \Gamma\in \Tt$. We say that a function $f:\Sigma\to \Gamma$ is \emph{definable by an FO-transduction} if there is some FO-transduction $\varphi$ which makes the following diagram commutes:
\[\begin{array}{lll}
\Sigma& \stackrel{t\mapsto \underline{t}}{\xrightarrow{\hspace*{3cm}}} & \text{structures over }\underline \Sigma \\
f&&\varphi\\
\Gamma& \stackrel{t\mapsto \underline{t}}{\xrightarrow{\hspace*{3cm}}}  & \text{structures over }\underline \Gamma 
\end{array}\]
\end{definition}

\begin{proposition}
Let $\Gamma, \Sigma \in \Tt$. If $f:\Sigma\to\Gamma$ is a first-order term function, then it is definable by an FO-transduction. 
\end{proposition}

\begin{pr}
The proof is by induction, following the defintion of first-order term functions. The atomic functions $\llone$, constant functions, projection, co-projection and distributions are easily definable by FO-transductions.  
We prove now that are also definable by FO-ransductions. 
\\

In the following we adopt the following shortcuts:
\begin{align*}
\sf{FirstPort}_\Sigma (x)& := \Port\Sigma(x)\wedge \neg (\exists y. \Port\Sigma(y)\wedge y\prec_\Sigma x)\\
y=\sf{succ}_< (x) &:=  y<x \wedge \neg (\exists z. y<z \wedge z<x)
\end{align*} 
\begin{enumerate}
\item $\sf{Flat}$. Given a tree $t$ of type $\trees\trees\Sigma$, the fucntion $\sf{Flat}(t)$ can be implemented using an FO-transduction as follows: the copying constant is 1, the universe formula is 
\[\varphi_1(x) = \neg (\Root{\trees\Sigma}(x) \vee \Port{\trees\Sigma}(x) )  \]
For every relation symbol $R\in\underline \Sigma$ , we conserve the same interpretation:
\[\varphi_R^{1,1}(x,y)=R(x,y) \text{ if $R$ is binary }
\quad \text{ and } \quad\varphi_R^{1}(x)=R(x) \text{ if $R$ is unary }\] 
For the other symbols of $\underline\trees\Sigma$, that is $\Root{\trees\Sigma},\Port{\trees\Sigma}, \prec_{\trees\Sigma}, \leq_{\trees\Sigma}$, we "shift by one $\trees$" their interpretation, more precisely, if $R\in\{\Root{}, \Port{}, \prec, \leq\}$, we set:
\[ \varphi^1_{R_{\trees\Sigma}}(x) = R_{\trees\trees\Sigma}(x) \text{ if $R_{\trees\Sigma}$ is unary } \]
\[\varphi^{1,1}_{R_{\trees\Sigma}}(x,y) = R_{\trees\trees\Sigma}(x,y) \text{ if it is binary. }\] 
\item $\sf{Block}$. We show how to simulation the function $\sf{Block}:\trees (\Sigma + \Gamma) \to \trees (\trees \Sigma + \trees \Gamma)$ by an FO-transduction. 
\item $\sf{Comb}$. The function $\sf{Comb}:\trees \llone\to\trees\llone$ is implemented by an FO-transduction as follows. Th copying constant is 4. 
In the first copy we keep the root and the ports, and in the other 3 copies we keep all the ports exept the first one:
\begin{align*}
\varphi_1(x)&=\Root{\trees\llone}(x) \vee \Port{\trees\llone}(x)\\
\varphi_i(x)&=\Port{\trees\llone}(x)\wedge \neg \sf{FirstPort}_{\trees\llone}(x)& i=2,3,4
\end{align*}

\begin{align*}
\varphi^1_{\Root{\trees\llone}}(x)=\Root{\trees\llone}(x) \qquad \varphi^1_{\Port{\trees\llone}}(x)=\Port{\trees\llone}(x)
\qquad\varphi^{1,1}_{\prec_{\trees\llone}}(x,y)=\prec_{\trees\llone}(x,y) 
\end{align*}

\begin{align*}
\varphi^{2,3}_{\leq_\llone}(x,y) = \varphi^{2,4}_{\leq_\llone}(x,y) = (x=y)\\
\varphi^{4,3}_{\prec_\llone}(x,y) =(x=y)
\end{align*}

\begin{align*}
\varphi^{2,3}_{<_{\trees\llone}}(x,y)= \varphi^{2,4}_{<_{\trees\llone}}(x,y)= ((x=y) \vee (x\prec_{\trees\llone} y))\\
\varphi^{4,2}_{<_{\trees\llone}}(x,y)=\varphi^{4,3}_{<_{\trees\llone}}(x,y)=\varphi^{2,2}_{<_{\trees\llone}}(x,y)= \varphi^{4,4}_{<_{\trees\llone}}(x,y)= (x\prec_{\trees\llone} y)\\
\varphi^{1,1}_{<_{\trees\llone}}(x,y)=\varphi^{1,2}_{<_{\trees\llone}}(x,y)= \Root{\trees\llone}(x)\\
\varphi^{2,1}_{<_{\trees\llone}}(x,y)=\varphi^{3,1}_{<_{\trees\llone}}(x,y)=(x=y) \vee (y\prec_{\trees\llone} x)\qquad 
\varphi^{4,1}_{<_{\trees\llone}}(x,y)=(y\prec_{\trees\llone} x)
\end{align*}
\begin{center}
Add pictures and explanations.
\end{center}
\item $\sf{Unfold}$. Let us now consider the function $\sf{Unfold}:\trees(\Sigma\otimes \Gamma)\to \trees\Sigma\otimes\trees\Gamma$. The copying constant is 2. 
\begin{align*}
\varphi_1(x)=\neg [\Root{\Sigma\otimes\Gamma}(x)\vee (\Port{\Sigma\otimes\Gamma}(x)\wedge \exists y. y=succ_{<_{\trees(\Sigma\otimes\Gamma)}}(x) \wedge \Root{\Sigma\otimes\Gamma}(y))]\\
\varphi_2(x)=\exists r, y. \Root{\trees(\Sigma\otimes\Gamma)}(r) \wedge y=succ_{<_{\trees(\Sigma\otimes\Gamma)}}(r) \wedge x= succ_{<_{\trees(\Sigma\otimes\Gamma)}}(x) 
\end{align*}
In the first copy, all the relations of $\underline{\Sigma}$ and $\underline{\Gamma}$ are interpreted in the same way. 
Let us discuss the relation of $\trees\Sigma$ and $\trees\Gamma$.
\[\varphi^{1,1}_{\trees\Sigma}(x,y)=\] 
\end{enumerate} 

 \end{pr}