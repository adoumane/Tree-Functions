\section{Factorisation forests}

\newcommand{\branches}{\mathsf{B}}

\paragraph{Branches.}
For a ranked set $\rSigma$, define its \emph{branches} to be the unranked set (hence the black font):
\begin{align*}
\branches \rSigma \quad \eqdef \quad \set{(a,i) : a \in \rSigma, i \in \set{1,\ldots,\arity a}}.
\end{align*}
For $a \in  \rSigma$, a branch in $a$ is defined to be any branch obtained by choosing some port of $a$. 
We are mainly interested in the case of branches in terms, which can be visualised as follows:
\mypic{81}
 
For a term, we classify its edges as internal (linking a non-port node with a non-port child) and external (linking a non-port node with a child port). Each edge in a term $t \in \tmonad \rSigma$ corresponds to a branch over $\rSigma$, namely the branch which leads to the edge. Any branch obtained this way is called a \emph{subbranch} of $t$. Here is a picture of subbranches in the case of a term of terms:
\mypic{80} 

\paragraph{Factorisation forests} The idea behind factorisation forests is to split a term into a nested factorisation, which is a term of terms of terms, and so on up to a certain depth.  
Define a \emph{nested factorisation} of depth $k \in \set{1,2,\ldots}$ over alphabet $\rSigma$ to be an element of $\tmonadn k \rSigma$ which is defined by
\begin{align*}
\tmonadn 0 \rSigma = \rSigma  \quad \text{and} \quad \tmonadn {k+1}\rSigma = \tmonad \tmonadn k \rSigma.
\end{align*}
Nested factorisations can be flattened to terms by using an  operation $\flatn k : \tmonadn k \rSigma \rto \tmonad \rSigma $ defined by 
\begin{align*}
     \flatn 1 = \text{\ranked{identity}} \quad \text{and} \quad  \flatn {k+1} \eqdef \flatt  \circ \tmonad \redpar { \flatn k}.
\end{align*}
An equivalent definition of $\flatn {k+1}$ would be $\flatn k \circ \tmonadn {k-1} \flatt$, the equivalence of these definitions corresponds to the fact that $\tmonad$ is a monad.



Branches in  terms $\branches \tmonad \rSigma$  form a semigroup, which we denote by $\branches \rSigma$. 
The idea behind factorisation forests, as expressed in Definition~\ref{def:hom-for} below, is to factorise a term into a term of terms of terms (etc.) so that the depth of nesting is bounded, and at each level all branches behave regularly with respect to some semigroup homomorphism. 

\begin{definition}[Homogeneous factorisations]\label{def:hom-for}
    Let $h : \branches \tmonad \rSigma \to S$ be a semigroup homomorphism.
    \begin{itemize}
\item     We say that $t \in \tmonad \tmonad \rSigma$ is \emph{$h$-homogeneous} if it is either a shallow term (which means that all internal edges originate from the root) or all internal subbranches of $t$ have the same value under $h$.
\item We say that  $t \in \tmonad \rSigma$ is \emph{hereditarily $h$-homogeneous} if it is the unit of some letter;
\item We say that  $t \in \tmonadn k  \rSigma$ is \emph{hereditarily $h$-homogeneous}, for $k \ge 2$, if both:
\begin{enumerate}
    \item  after applying $\tmonad \flatn {k-1}$,   the resulting term in $\tmonad \tmonad \rSigma$ is $h$-homogeneous; and 
    \item every node has a label in $\tmonadn {k-1} \rSigma$ that is hereditarily $h$-homogeneous. 
\end{enumerate}
    \end{itemize}
\end{definition}

The main result of this section is the following version of the factorisation forest theorem. It differs from the original Factorisation Forest Theorem of Imre Simon in the following ways: (a) we consider trees instead of strings; (b) we use aperiodic finite semigroups instead of arbitrary finite semigroups; and (c) the factorisation in the conclusion of the theorem can be computed by a derivable function.  A tree generalisation of the Factorisation Forest Theorem was already proved by Colcombet~\cite[Theorem 1 and Section 3.3]{colcombetCombinatorialTheoremTrees2007}, but Colcombet's result is proved for monadic second-order logic, and therefore it does not satisfy condition (c). 
\begin{theorem}[Factorisation Forest Theorem]\label{thm:factfor}
    Let $\rSigma$ be a finite ranked set and let $h : \branches \rSigma \to M$ be a monoid homomorphism into a finite aperiodic monoid $M$. There is some $k \in \set{1,2,\ldots}$ and a derivable function
    \begin{align*}
        \ranked {f : \tmonad \rSigma \to \tmonad^k \rSigma}  
    \end{align*}
such that $\flatn k \circ \ranked f$ is the identity on $\tmonad \rSigma$, and  all outputs of  $\ranked f$ are hereditarily $h$-homogeneous.
\end{theorem}



\newcommand{\hint}{\bar h}
\newcommand{\hintplus}{\bar h^+}
\newcommand{\branchesplus}{\mathsf B^+}
For $t \in \tmonad  \tmonad \rSigma$ define 
\begin{align*}
    \hint(t) = \set{h(\pi) : \text{$\pi$ is an internal branch of $t$}}
\end{align*}
The theorem is proved by induction on two parameters: the size of $\hint(t)$, and the size of the semigroup generated by $\hint(t)$, ordered lexicographically with the second parameter being more important. This induction is stated in the following lemma, where the theorem is the special case when $A=B=S$. 
\begin{lemma} Let $h : \branchesplus \tmonad \rSigma \to S$ be a semigroup homomorphism into a finite aperiodic semigroup, and let  $A,B \subseteq S$. There is some $k$ and  a derivable function
    \begin{align*}
        \ranked {f_{A,B}: \tmonad \rSigma \to \tmonad^k \rSigma}  
    \end{align*}
    such that $\flatn k \circ \ranked f$ is the identity on $\tmonad \rSigma$, and   outputs of  $\ranked {f_{A,B}}$ are hereditarily $h$-homogeneous for inputs such that:
    \begin{align*}
    B = \hint(t) \qquad A = \text{semigroup generated by $\hint(t)$.}
    \end{align*}
        
\end{lemma}
The theorem is the special case of the a
\begin{enumerate}
    \item the size of the set 
    \begin{align*}
    \set{h(t)}
    \end{align*}
\end{enumerate}

\begin{itemize}
    \item There is some $x_0 \in $ such that 
    \begin{align}\label{eq:smaller-semigroup}
         \set{ a a_0 :  a \in \hintplus(t)} \subsetneq  \hintplus(t)
    \end{align}
    Choose some $x$ with the above property, and define $T$ to be the above set. It is not hard to see that $T$ is a proper subsemigroup of $S_t$.  Define a \emph{sensitive edge} in $t$ to be any internal edge where the corresponding subbranch has value $x_0$ under  $h$. Let $t' \in \tmonadn 3 \rSigma$ be the nested factorisation that results from $t$ by factorising along sensitive edges. Here is a picture:
    \begin{center}
        picture
    \end{center}
    Define the \emph{outer factorisation} $s \in \tmonad \tmonad \rSigma$ to be the result of applying $\tmonad \flatt$ to $t'$. By definition, every inner subbranch in the outer factorisation ends with $x_0$, and therefore the semigroup generated $\hint(s)$ is contained in 
     by  For a node  $x$ in $t'$ define the \emph{inner factorisation} $s_x \in \tmonad \tmonad \rSigma$ to be the label of node $x$ in $t''$. 
    By definition, the inner factorisations do not have internal subbranches labelled by $a_0$, and therefore 
    \begin{align*}
    \hint(s_x) \subsetneq \hint(t) \qquad \text{for every inner factorisation $s_x$}.
    \end{align*}
    We can therefore use the induction assumption to get nested factorisations for the inner factorisations. 
    For every inner factorisation, its inner subbranches have 
    \begin{itemize}
        \item For every node $x$ in $t'$ its label $t'(x) \in \tmonad \tmonad \rSigma$ satisfies 
        \begin{align*}
        \beta(t'(x)) \subseteq \beta(t) - \set{a_0}.
        \end{align*}
        \item The term $\tmonad \flat (t') \in $ satisfies
        \begin{align*}
            \alpha((\tmonad \flatt)(t')) \subseteq 
        \end{align*}
    \end{itemize}
    
\end{itemize}

For $t \in \tmonad \tmonad \rSigma$ consider $G_t$ the semigroup
