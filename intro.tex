
\section{Introduction}

The purpose of this paper is to decompose tree transformations into simple building blocks. An important inspiration  is the Krohn-Rhodes theorem~\cite[p.~454]{Krohn1965}, which says that every string-to-string function recognised by a Mealy machine can be decomposed into certain prime functions. 

\paragraph*{Regular functions.} The transformations studied in this paper are the so called  regular functions.

In~\cite[Theorem 13]{engelfrietMSODefinableString2001}, Engelfriet and Hoogeboom proved that deterministic two-way transducers recognise the same string-to-string functions as \mso transductions. Because of this and other properties -- such as closure under composition~\cite[Theorem 1]{chytilSerialComposition2Way1977} and decidable equivalence~\cite[Theorem 1]{gurariEquivalenceProblemDeterministic1982} --  this class of functions is now called the \emph{regular string-to-string functions}. Other  equivalent descriptions of the regular functions include: string transducers of Alur and {\v C}ern{\'y}~\cite{alurExpressivenessStreamingString2010}, and several models based on combinators~\cite{alur2014regular,daveGastinKrishna18, bojanczykRegularFirstOrderList2018}. 
 
% One corollary of the description in~\cite{bojanczykRegularFirstOrderList2018} is that the regular  string-to-string functions are the smallest class of string-to-string functions which is closed under composition, contains functions recognised by one-way deterministic automata, and the following two operations:
% \begin{eqnarray*}
% w_1 |  \cdots | w_n &\qquad \mapsto \qquad& w_1 w_1 |  \cdots | w_n w_n, \\
% w_1 |  \cdots | w_n &\qquad \mapsto \qquad& \text{reverse}(w_1) |  \cdots | \text{reverse}(w_n),
% \end{eqnarray*}
% where $w_1,\ldots,w_n \in \set{a,b}^*$ and $|$ is a separator symbol. The deterministic one-way automata themselves can be further decomposed using the Krohn-Rhodes theorem; thus leading a decomposition into very simple prime functions.

There are also regular functions for trees, which can be defined using any of the following equivalent models: \mso tree-to-tree transductions~\cite[Section 3]{bloem_comparison_2000}, single use attributed tree grammars~\cite{bloem_comparison_2000}, macro tree transducers of linear size increase~\cite[Theorem 7.1]{engelfriet_macro_2003}, and streaming tree transducers~\cite[Theorem 4.6]{alur2017streaming}. 

The goal of this paper is to prove a decomposition result for regular tree-to-tree functions. As in the Krohn-Rhodes theorem, we want to show that every such function can be obtained by combining certain prime functions.  

\paragraph*{First-order transductions. } Although \mso transductions are the more popular model, we work mainly with the less expressive model of first-order transductions. Why?

As we explain in Section~\ref{sec:mso-trans}, every \mso tree-to-tree transduction can be decomposed as: (a) first, a relabelling defined in \mso, which does not change the tree structure; followed by (b) a first-order tree-to-tree transduction. In this sense, as far as transformations of the tree structure are concerned,  first-order and \mso transductions have the same expressive power. Another argument for the importance of first-order tree-to-tree transductions is the connection with the $\lambda$-calculus. As we explain in Section~\ref{sec:stt-derivable}, first-order tree-to-tree transductions are expressive enough to capture evaluation of $\lambda$-terms (which are linear, i.e.~every variable is used once), and such evaluation turns out to be one of the core computational steps implicit in a tree-to-tree transduction. 

Another advantage of first-order logic on trees is that it has a better decomposition theory, in the sense of decomposing formulas into combinations of simpler ones~\cite{haferthomas,bojanczykDecidablePropertiesTree2004,esik-weil1}. 
% This theory includes characterisations in terms of the temporal logic {\sc ctl*} \cite[Main Theorem]{haferthomas}, cascade products of tree automata~\cite[Theorem 2.5.7]{bojanczykDecidablePropertiesTree2004}, block products~\cite[Corollary 3.11]{esik2010algebraic} and wreath products~\cite[Theorem 3.1]{bojanczykWreathProductsForest2012}. 
For our paper, the most useful decomposition is a remarkable theorem of Schlingloff, which shows that first-order logic on trees is equivalent to a certain two-way variant  {\sc ctl}~\cite[Theorem 4.5]{schlingloff1992expressive}. In contrast, there are no such results for \mso. 
% One reason is that \mso on trees is too hard if we are interested in Krohn-Rhodes decompositions.  Already for the simplest tree formalisms, such as tree languages or letter-to-letter transformations, there is no known  Krohn-Rhodes theory. One would expect a Krohn-Rhodes theorem  for trees to yield an effective characterisation of first-order logic -- as it does for words -- but finding such a characterisation remains a major open problem~\cite[Section 3]{bojanczyk2015automata}. We do not attempt to solve this problem here. In contrast, first-order logic on trees does have decomposition theorems, which we can use: including

% Another, more positive, reason is that in terms of restructuring trees, first-order logic is not that far from \mso.  By~\cite[Corollary 1]{colcombetCombinatorialTheoremTrees2007},  every \mso tree-to-tree transduction can be decomposed into an \mso transduction that does not change the tree structure, followed by a first-order transduction. Furthermore, as we show in this paper, first-order logic is sufficient for certain fundamental transformations, such as evaluating $\lambda$-terms. 


Summing up, we believe that first-order  tree transformations are an expressive model, with a strong theory, and deserve to leave the shadow of their better known \mso cousin.


\paragraph*{Structured datatypes.} We present our main result in a formalism which can be viewed as a  combinator $\lambda$-calculus, with structured datatypes such as pairs or co-pairs.  The motivation behind this programming language is that, in the  Krohn-Rhodes decomposition, the   intermediate functions  operate on strings that are full of endmarkers and separators.   To avoid such low-level, annotation the programming language from~\cite{bojanczykRegularFirstOrderList2018}, uses  datatype constructors such as pairs or disjoint unions.  Thanks to such datatypes, one can use established operations such as map, head or tail that are used in functional programming languages. 
In this paper, we apply the same method to trees.
For trees,  the choice of datatypes  becomes harder. The difficulty is in splitting the input into smaller pieces. A piece of a string is also a string, but this is no longer true for trees, where the pieces have dangling edges (or variables). As a result, more complicated  datatypes are needed.



This is a long paper. Given the limited space, we have decided to prioritise  explaining the definitions and approach that we use, with pictures and examples. As a result, almost all of the proofs are in the appendix. 



