\section{Evaluation of simply typed linear terms}
\label{sec:one-register}

In this section, we consider an important example of derivable term transformations, namely evaluating (i.e.~computing $\beta$-normal forms of)  simply typed  $\lambda$-terms that are linear in the sense that each bound variable is used exactly once in its scope. Apart from its independent interest, evaluation of $\lambda$-terms will  be used in the proof our main theorem. 



\newcommand{\otype}{o}
We assume that the reader is familiar with the basic notions of the simply typed $\lambda$-calculus; more detailed definitions can be found in~\cite{sorensen_lectures_2006}. Define  \emph{simple types} to be expressions  generated from an atomic type $\otype$ using a binary arrow constructor, as in the following examples:
    \begin{align*}
        \otype \qquad \otype \to \otype \qquad (\otype \to \otype) \to (\otype \to \otype) \qquad \cdots 
    \end{align*}
Let $X$ be a set of variables, each one with an associated simple type.  A $\lambda$-term over these variables can be seen as a tree over the ranked alphabet
\begin{align*}
      \overbrace{\set{x : x \in X}}^{\text{arity 0}} \cup \overbrace{\set{\lambda x : x \in X}}^{\text{arity 1}} \cup  \overbrace{\set @}^{\text{arity 2}}
\end{align*}
where @ is the application symbol that is used to apply one term to another.
We say that a $\lambda$-term is \emph{well-typed} if one can associate  to it  a simple type according to the usual typing rules of simply typed $\lambda$-calculus\footnote{
    Here we assume that the variables are typed, but the types for the remaining $\lambda$-terms need to be inferred. We could adopt a different approach, more thoroughly in the style of Church, where all term constructors (application and $\lambda$-abstraction) come decorated with the type of the resulting term. We do not do this to make  notation lighter, and also because one show -- see the appendix -- that first-order logic is enough to reconstruct the type of a term once the types of the variables are known. 
}, see~\cite[Definition 3.2.1]{sorensen_lectures_2006}. Because the variables are typed, a  $\lambda$-term can have either a unique type, or it is not well-typed.  Here is an example of a well-typed $\lambda$-term: 
\mypic{45}
We use the standard notion of $\beta$-reduction for $\lambda$-terms, see~\cite[Definition 1.2.1]{sorensen_lectures_2006}. 
Because of normalisation and confluence for the simply typed $\lambda$-calculus, every well-typed $\lambda$-term has a unique normal form, i.e~a $\lambda$-term to which it $\beta$-reduces (in zero or more steps), and which cannot be further $\beta$-reduced.


\begin{example}\label{ex:exponential}
    Because of iterated duplication, the normal form of a $\lambda$-term can be exponential. Assume that we have two variables $\typevar x  \otype$ and $\typevar y {\otype \to \otype \to \otype}$ and consider the $\lambda$-terms defined by:
    \begin{align*}
        M_0 \eqdef \typevar x \otype \qquad M_{n+1} = (\lambda \typevar x  \otype . \typevar y {\otype \to \otype \to \otype}  \typevar x  \otype \typevar x  \otype)M_n.
    \end{align*}
    The $\lambda$-term $M_n$ is well-typed and of type $\otype$. It has size linear in $n$, but its normal form has size exponential in $n$. 
    % as shown in the following picture, which uses variables $x : \otype$ and  $z : \otype \to \otype \to \otype$.
    % \mypic{44}
\end{example}
As witnessed by the above example, normal forms can be exponential size (or worse, see~\cite[Section 3.6]{sorensen_lectures_2006}), and therefore they cannot be computed  using derivable functions or first-order transductions, because the latter have linear size outputs. To avoid this problem, we  limit attention to linear $\lambda$-terms: a $\lambda$-term is called \emph{linear} if every bound variable is used exactly once in its scope. Here is an example: 
\mypic{43}
For linear $\lambda$-terms, each step of $\beta$-reduction reduces the number of nodes by exactly 3, and therefore the normal form is linear in the size of (in fact, smaller or equal to) the original term. 

Linearity alone is not enough to normalise terms with first-order transductions. Another obstacle is terms that use types of unbounded complexity, as illustrated in the following example. 

\begin{example}
Consider the following $\lambda$-terms, which have types of unbounded size:     \begin{align*}
        M_0 \eqdef \typevar u {\otype \to \otype}  \qquad M_{n+1} \eqdef \overbrace{\lambda \typevar x {\otype}. \lambda \typevar y {\otype}.   M_n (\typevar z {\otype \to \otype \to \otype}  \typevar x {\otype} \typevar y {\otype}).}^{\text{type} \overbrace{\otype \to \otype \to \cdots \to \otype}^{\text{$n+1$ arrows}}}
    \end{align*}
    Apart from having large types (although still of rank 1, as described in~\cite[Exercise 3.6.7]{sorensen_lectures_2006}), these  $\lambda$-terms are simple: they use four variables (although the bound variables are reused), and are linear, because each bound variable is used exactly once in its scope. 
    To $M_n$, apply  $m$ arguments of type $\otype$:
    \begin{align}\label{eq:complicated-term}
    M_n \overbrace{\typevar v \otype \ \typevar v {\otype} \cdots \typevar v{\otype} }^{\text{$m$ times}}.
    \end{align}
    We claim that the above $\lambda$-term cannot be normalised using a first-order transduction, or even a monadic second-order transduction. In order to normalise, a transduction would need to be able to compare the numbers $n$ and $m$ as follows:  if $m < n$  the normal form contains $\lambda$, if $m=n$  the normal form does not contain $\lambda$, and if $m > n$ then the normal form is undefined because the $\lambda$-term is not well-typed.  Whether or not a $\lambda$-term (seen as a tree over a finite alphabet) contains $\lambda$ is a first-order definable property, and first-order definable properties are preserved under inverse images of first-order transductions. Therefore, if normalisation would be a first-order transduction,
then there would be a first-order formula which would be true for terms of the form~\eqref{eq:complicated-term} with $m>n$ and which would be false for terms of the form~\eqref{eq:complicated-term} with $m=n$. Such a formula cannot exist (even if we allow monadic second-order logic), which can be shown using a pumping argument or Ehrenfeucht-Fra\"iss\'e games. 
\end{example}

The above example explains why we need to bound the size of types that occur in subterms, motivating the following definition:  we say that a $\lambda$-term uses only types from a finite set of simple types $\typeset$ if all  subterms have types in $\typeset$.  Once we assume that $\lambda$-terms are linear, use a fixed finite set of bound variables, and use only types from a fixed finite set of simple types, then normalisation can be done by a derivable function, and therefore also by a first-order tree-to-tree transduction: 

\begin{theorem}\label{thm:normalise} Let $X$ be a finite set of simply typed variables, and let $\typeset$ be a finite set of simple types.
    The following tree-to-tree function is derivable (i.e.~it is the tree restriction of a derivable function on terms):
    \begin{align*}
        M \in \text{$\lambda$-terms over variables $X$} \qquad \mapsto \qquad \begin{cases}
            \text{normal form of $M$} & \text{if $M$ is linear, well-typed,}\\
            & \text{and uses only types from $\typeset$;}\\
            M & \text{otherwise}.
        \end{cases}
    \end{align*}
\end{theorem}

% Actually, we prove a stronger result, namely that the above function is derivable. In the end, of course, we will prove that all first-order tree-to-tree transductions are derivable, but our proof of this fact will use the derivability of normalisation as stated in Theorem~\ref{thm:normalise}.



% Recall that elements of $\tmonad \rSigma$ are defined to be trees over $\rSigma + \varnames$, where 
% \begin{align*}
%     \varnames = \set{x_1,x_2,\ldots}
% \end{align*}
% is a fixed set of  variable names that are used for ports. In this section, we view $\varnames$ as a simply typed set, where all variables $x_1,x_2,\ldots$ have the same simple type $\otype$.  Under this convention, we can view
% \begin{align*}
%     \tmonad (\overbrace{\set{x : x \in X}}^{\text{arity 0}} \cup \overbrace{\set{\lambda x : x \in X}}^{\text{arity 1}} \cup  \overbrace{\set @}^{\text{arity 2}})
% \end{align*}
% as a set of $\lambda$-terms. This is exactly the set of those $\lambda$-terms over variables $X + \varnames$ where: (a) variables from $\varnames$ are never bound; and (b) there is some $n$ such that each of the variables $x_1,\ldots,x_n \in \varnames$  appears as a free variable exactly once  and the remaining variables in $\varnames$ do not appear at all. If $S$ is a finite set of simple types, then define  $\linterm S X$ to be those terms in .. which are linear and $S$-typed. 

% Computing the normal form is beyond the scope of first-order transductions, the principal reason being that first-order transductions have linear size outputs, while normalisation can incur a blowup that is exponential or larger,
%  In Section~\ref{sec:lambda}, we will show that normalisation can be computed by a first-order transduction, assuming that: (a) the input terms are linear, which means that each bound variable is used exactly once in its scope; (b) we place an upper bound on the complexity of types used in subterms. 

% as discussed in Example~\ref{ex:labmda-terms}.  We show that the tree-to-tree function which inputs a  term and outputs its normal form can be derived, assuming that bound variables are used exactly one and there is a bound on the number of distinct terms types that can appear in the term. 

% Let $X$ be a typed set, i.e.~a set of variables with associated simple types. As in  Example~\ref{ex:labmda-terms},  we view $\lambda$-terms with variables of $X$ as trees over an ranked alphabet $\lamrank X$. 

% \begin{lemma}
%     For every typed set $X$ and every finite set $S$ of simple types, the tree language 
%     \begin{align*}
%         \set{ M \in \trees \lamrank X : \text{$M$ is well-typed and all subterms have type in $S$}}
%     \end{align*}
%     is first-order definable 
% \end{lemma}



% We say that a function $\ranked f : \linterm S X \rto \linterm S Y$ is \emph{derivable} if it can be extended to a derivable function $\ranked g : \tmonad \lamrank X \rto \tmonad \lamrank X$. The main result of this section is that normalisation is derivable, for every fixed finite $X$ and $S$. 
% \begin{proposition}\label{prop:one-register} 
%     For every typed set $X$ and every finite set $S$ of simple types, the function 
%     \begin{align*}
%         M \in  \linterm S X \qquad \mapsto \qquad \text{normal form of $M$} \in  \linterm S X
%     \end{align*}
%     is derivable.
% \end{proposition}