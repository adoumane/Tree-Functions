\subsection{Factorisation forests}
\label{sec:factfor}
This section is devoted to stating and proving a tree version of the Factorisation Forest Theorem of Imre Simon.  Our result differs from the original Factorisation Forest Theorem in the following ways: (a) we consider trees instead of strings; (b) we use aperiodic finite monoids instead of arbitrary finite monoids; and (c) the factorisation in the conclusion of the theorem can be computed by a derivable function.  A tree generalisation of the Factorisation Forest Theorem was already proved by Colcombet~\cite[Theorem 1 and Section 3.3]{colcombetCombinatorialTheoremTrees2007}, but Colcombet's result is proved for monadic second-order logic, and therefore it does not satisfy condition (c). 



\paragraph{Factorisation forests} The idea behind factorisation forests is to split a term into a nested factorisation, which is a term of terms of terms, and so on up to a certain depth.  
Define a \emph{nested factorisation} of depth $k \in \set{1,2,\ldots}$ over alphabet $\rSigma$ to be an element of $\tmonadn k \rSigma$ which is defined by
\begin{align*}
\tmonadn 0 \rSigma = \rSigma  \quad \text{and} \quad \tmonadn {k+1}\rSigma = \tmonad \tmonadn k \rSigma.
\end{align*}
Nested factorisations can be flattened to terms by using an  operation $\flatn k : \tmonadn k \rSigma \rto \tmonad \rSigma $ defined by 
\begin{align*}
     \flatn 1 = \text{\ranked{identity}} \quad \text{and} \quad  \flatn {k+1} \eqdef \flatt  \circ \tmonad \redpar { \flatn k}.
\end{align*}
An equivalent definition of $\flatn {k+1}$ would be $\flatn k \circ \tmonadn {k-1} \flatt$, the equivalence of these definitions corresponds to the fact that $\tmonad$ is a monad.


\paragraph{Branches and subbranches}
Define a \emph{branch} in a ranked set to be an element  of the ranked set together with a distinguished port. 
We draw branches like  this:
\mypic{82}
For a term, we classify its edges as internal (linking a non-port node with a non-port child) and external (linking a non-port node with a child port). Each edge in a term $t \in \tmonad \rSigma$ corresponds to a branch over $\rSigma$, namely the branch which leads to the edge. Any branch obtained this way is called a \emph{subbranch} of $t$. Here is a picture of subbranches in the case of a term of terms:
\mypic{80} 
Branches in  terms $\branches \tmonad \rSigma$  form a monoid. 
The idea behind factorisation forests, as expressed in Definition~\ref{def:hom-for} below, is to factorise a term into a term of terms of terms (etc.) so that the depth of nesting is bounded, and at each level all branches behave regularly with respect to some monoid homomorphism. 

\begin{definition}[Homogeneous factorisations]\label{def:hom-for}
    Let $h : \branches \tmonad \rSigma \to S$ be a monoid homomorphism.
    \begin{itemize}
\item     We say that a factorisation $t \in \tmonad \tmonad \rSigma$ is \emph{homogeneous with respect to $h$} if it is either a shallow term (which means that all internal edges originate from the root) or all internal subbranches of $t$ have the same value under $h$.
\item We say that a nested factorisation  $t \in \tmonadn k  \rSigma$ is \emph{hereditarily homogeneous with respect to $h$} if either $k=1$ and $t$ is the unit of a letter, or $k \ge 2$ and both:
\begin{enumerate}
    \item  after applying $\tmonad \flatn {k-1}$,   the resulting term in $\tmonad \tmonad \rSigma$ is $h$-homogeneous; and 
    \item every node has a label in $\tmonadn {k-1} \rSigma$ that is hereditarily $h$-homogeneous. 
\end{enumerate}
    \end{itemize}
\end{definition}


Recall that a finite monoid is aperiodic if it has only trivial subgroups. An equivalent definition is that every element $m$ of the monoid satisfies 
\begin{align*}
  \exists n \in \set{1,2,\ldots}\   m^n = m^{n+1}.
\end{align*} 
A famous theorem of Schutzenberger~\cite{DBLP:journals/iandc/Schutzenberger65b} says that the languages  of words recognised by homomorphisms into aperiodic monoids are exactly the star-free languages. Since star-free languages of words are exactly those that can be defined in first-order logic, it is not surprising that we consider aperiodic monoids.

\begin{example}\label{ex:partial-monoton-functions}
   Let $k \in \set{1,\ldots}$ and  consider the monoid of partial functions 
    \begin{align*}
    \set{1,\ldots,k} \to \set{1,\ldots,k}.
    \end{align*}
    This monoid is not aperiodic, because the permutation which swaps $1$ and $2$ generates a group of order 2. Consider now the restriction of this monoid to partial functions which are monotone (this is a monoid, because such functions are closed under composition). This monoid is aperiodic, because if $f$ is a partial function, then for  every $i \in \set{1,2,\ldots,k}$ the sequence
    \begin{align*}
    f^1(k),f^2(k),f^3(k),\ldots
    \end{align*}
    reaches a fixpoint (or becomes undefined) in at most $k$ steps.
\end{example}
We are now ready to state our version of the Factorisation Forest Theorem. 
\begin{theorem}[Factorisation Forest Theorem]\label{thm:factfor}
    Let $\rSigma$ be a  ranked set and let $h : \branches \tmonad \rSigma \to M$ be a monoid homomorphism into a finite aperiodic monoid $M$. There is some $k \in \set{1,2,\ldots}$ and a  function
    \begin{align*}
        \ranked {f : \tmonad \rSigma \to \tmonad^k \rSigma}  
    \end{align*}
such that $\flatn k \circ \ranked f$ is the identity on $\tmonad \rSigma$, and  all outputs of  $\ranked f$ are hereditarily $h$-homogeneous.
\end{theorem}



\newcommand{\hint}{\bar h}
\newcommand{\hintplus}{\bar h^+}
\newcommand{\branchesplus}{\mathsf B^+}
Later in this section, we also give a sufficient condition which ensures that $\ranked f$ as in the theorem is derivable, by analysing in more detail the proof the theorem. 
We prove the theorem by induction on a parameter which is stated in the following lemma. The lemma immediately implies  the theorem,   in the case when $A$ is the entire monoid.
\begin{lemma}  Let $\rSigma$ be a  ranked set and let $h : \branches \tmonad \rSigma \to M$ be a monoid homomorphism into a finite aperiodic monoid $M$. For every    $A \subseteq M$ there exists  $k \in \set{1,2,\ldots}$ and  a  function
    \begin{align*}
        \ranked {f_{A}: \tmonad \rSigma \to \tmonad^k \rSigma}  
    \end{align*}
    such that $\flatn k \circ \ranked {f_A}$ is the identity on $\tmonad \rSigma$, and   outputs of  $\ranked {f_{A}}$ are hereditarily $h$-homogeneous for inputs with 
    \begin{align*}
        A = \hint(t) \eqdef \set{h(\pi) : \text{$\pi$ is an internal branch of $t$}} 
    \end{align*} 
\end{lemma}
Define $\hintplus(t)$ be the semigroup generated by $\hint(t)$. 
The lemma is 
proved by induction on two parameters: the size of  the semigroup $\hintplus(t)$, and the size of $\hint(t)$, ordered lexicographically with the second parameter being more important. The induction base is when $\hint(t)$ contains only one element of the monoid $x$. In this case, 
\begin{align*}
\ranked{\tmonad \unit : \tmonad \Sigma \to \tmonad \tmonad \Sigma}
\end{align*}
gives a hereditarily homogeneous output, since the inner factors are units, and the internal subbranches have the same value, namely $x$. We are left with the induction step, where we do the following case analysis.

\begin{itemize}
    \item There is some $x_0 \in \hint(t)$ such that 
    \begin{align}\label{eq:smaller-semigroup}
         \set{ a a_0 :  a \in \hintplus(t)} \subsetneq  \hintplus(t)
    \end{align}
    Choose some $x_0$. Observe that the left side of the above inclusion is a semigroup; this semigroup will be used later on when applying the induction assumption. Define a \emph{sensitive edge} in $t$ to be any internal edge where the corresponding subbranch has value $x_0$ under  $h$. Let $s \in \tmonadn 2 \rSigma$ be the nested factorisation that results from $t$ by factorising along edges that are sensitive, but their parent is not sensitive, as explained in the following picture:
    \begin{center}
        (todo)
    \end{center}
    For a node  $x$ in $s$ define the \emph{inner term} $s_x \in \tmonad  \rSigma$ to be the label of node $x$ in $s$. We now consider separately the outer term $s$ and the inner terms.
    \begin{itemize}
        \item \emph{The outer term.} By definition, every internal subbranch of $s$ ends with $x_0$, which will allow us to apply the induction assumption for a new alphabet $\tmonad \rSigma$. 
     \begin{center}
            (todo)
        \end{center}
        
    \item \emph{The inner terms.} Consider now an inner term $s_x$. If we follow a  branch in $s_x$ from root to port, we first  have subbranches with value other than $x_0$, and then we have a sequence of branches that have value $x_0$. Consider a factorisation of $s_x$ which is obtained by cutting along the first (closest to the root) subbranches with value $x_0$, as in the following picture:
    \begin{center}
        (todo)
    \end{center}
    The result is a shallow term, call it 
    \begin{align*}
    s'_x \in \shallowterm{\tmonad \rSigma}{\tmonad \rSigma}.
    \end{align*}
    The term in the root of $s'_x$ does not use $x_0$ for internal subbranches, and therefore we can apply the induction assumption to it. The terms in the children are homogeneous, because they only use $x_0$. This induces a hereditarily homogeneous factorisation for $s_x$, which has the shallow term on the outermost level. 
    \end{itemize}
    
    \item The remaining case is when every $x \in \hint(t)$ satisfies 
     \begin{align*}
        \set{ a a_0 :  a \in \hintplus(t)} \subsetneq  \hintplus(t)
   \end{align*}
   
   \begin{center}
       (todo)
   \end{center}
\end{itemize}

For $t \in \tmonad \tmonad \rSigma$ consider $G_t$ the semigroup
