\section{First-order transductions to  atomic functions and combinators}
\label{sec:strategy}
The rest of this paper is devoted to the harder left-to-right implication in Theorem~\ref{thm:main}, which says that every  first-order tree-to-tree transduction
can be derived from the atomic functions using the combinators. This proof has several steps:

\newcommand{\announce}[2]{
\begin{center}
    {\bf #1.} #2
\end{center}
}
\begin{enumerate}
    \item \emph{Section~\ref{sec:fo-translation}: first-order rational tree functions.} We begin the proof with a special case of first-order transductions, which we call \emph{first-order rational tree functions.} In a first-order rational tree functions, each node is replaced by a term, with the choice of term depending on first-order definable properties of the node. Example of first-order rational tree function include  ``remove every node with a unary label'', or ``if a leaf $x$ has at least one ancestor with label $a$, then replace the label of $x$ with the letter $c$''. First-order rational tree functions do not cover all first-order transductions, for example the pre-order function is not a rational tree function, but they are an important subcase.
    The main result of Section~\ref{sec:fo-translation} is:
    \announce
    {Proposition~\ref{prop:forat}}
    {Every first-order rational tree function can be derived using combinators for the atomic functions.}
    The main tool used in the proof of Proposition~\ref{prop:forat} is a theorem of Schlingloff~\cite[Theorem 2.6]{schlingloff1992expressive}, which can be seen as tree version of the Kamp theorem about {\sc ltl} being  expressively complete for first-order logic.  
    \item \emph{Section~\ref{sec:stt}: streaming tree transducers. }In Section~\ref{sec:stt}, we present an automaton model that is equivalent to  first-order tree-to-tree transduction. This  is a bottom-up tree automaton, which uses registers to store parts of the output tree. The model is based on~\cite{alur2017streaming}, but it is  appropriately restricted so that it matches first-order logic, as opposed to monadic second-order logic which was used in~\cite{alur2017streaming}. The main result of Section~\ref{sec:stt} is:
    \announce
    {Proposition~\ref{prop:stt}}
    {Every first-order rational tree function can be computed by a first-order streaming tree transducer.}
    
     \item \emph{Section~\ref{sec:one-register}: streaming tree transducers with one register.} 
     
     \announce
    {Proposition~\ref{prop:one-register}}
    {If a first-order streaming tree transducer has one register, then the function it computes can be derived using the combinators from the atomic functions.}
    
     
     The control structure (which states are used in which nodes of the input tree) of a first-order streaming tree transducer can be computed 
    \item \emph{Section~\ref{sec:matrix-power}: reduction to one register.} By the results of Section~\ref{sec:stt}, every first-order tree transduction can be computed by a first-order streaming tree transducer. In Section~\ref{sec:}
    
\end{enumerate}
