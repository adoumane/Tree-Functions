\section{Proof strategy}
\label{sec:strategy}
In this section, we present an overview of the top-down implication in the  proof of Theorem~\ref{thm:main}, which says that every first-order tree-to-tree transduction is (the restriction to trees of) a derivable function. The proof has five steps, which are spread across Sections~\ref{sec:fo-translation}--\ref{sec:stt-derivable}, and summarised below.

\newcommand{\announce}[2]{
\begin{center}
    {\bf #1.} #2
\end{center}
}
%\begin{enumerate}
%\item



  %  \item
  \paragraph*{Section~\ref{sec:fo-translation}: first-order tree relabellings.} We begin the proof of the left-to-right  implication  with a special case of first-order tree-to-tree transductions, which we call \emph{first-order  tree relabellings.} In such a  function, every node of the input tree  is given a new label, depending on first-order definable properties of the node. An example is:  ``if a leaf $x$ in the input tree has at least one ancestor with label $a$, then replace the label of $x$ with the letter $c$''. First-order tree relabellings  do not cover all first-order transductions, because they do not change the shape of the input tree, but they are an important subcase.
    The main result of Section~\ref{sec:fo-translation} is:
    \announce
    {Proposition~\ref{prop:forat}}
    {Every first-order tree relabelling is derivable.}
    The main tool used in the proof of Proposition~\ref{prop:forat} is a theorem of Schlingloff~\cite[Theorem 2.6]{schlingloff1992expressive}, which can be seen as tree version of Kamp's theorem about {\sc ltl} being  expressively complete for first-order logic.  
    
    

 \paragraph*{Section~\ref{sec:stt}: register transducers.} To prove that  first-order tree-to-tree transductions are derivable, it will be more convenient to use an automaton model. In Section~\ref{sec:stt}, we present  this  automaton model, which we call \emph{register transducers}\footnote{These are the only kind of register transducer used in this paper. A more complete name would be first-order tree-to-tree register transducers.}. This  is a bottom-up tree automaton, which uses registers to store parts of the output tree. The model is based on streaming tree transducers from~\cite{alur2017streaming}, but it is  appropriately restricted so that it matches first-order logic, as opposed to monadic second-order logic which was used in~\cite{alur2017streaming}. The main result of Section~\ref{sec:stt} is:
    \announce
    {Theorem~\ref{thm:stt}}
    {Every first-order  tree-to-tree transduction can be computed by a  register transducer.}
    The proof of the above theorem, like for similar results in~\cite{alur2017streaming} or~\cite[Theorem 23]{engelfrietMSODefinableString2001}, is based on the composition method in logic. 
    The converse implication is also true, and follows from Proposition~\ref{prop:many-register} and the bottom-up implication in Theorem~\ref{thm:main}.
    
     \paragraph*{Section~\ref{sec:one-register}: register transducers with one register.} By Theorem~\ref{thm:stt}, in order to prove the left-to-right implication in Theorem~\ref{thm:main}, it is enough to show that register transducers can only compute  derivable  functions. As a first step, we consider register transducers with only one register. When there is only one register, the computed function turns out to be a composition of a first-order tree relabelling, as treated  Proposition~\ref{prop:forat}, followed by the evaluation (i.e.~computation of the $\beta$-normal form) of    a simply typed $\lambda$-term which is affine (i.e.~uses each variable at most once). The main result of Section~\ref{sec:one-register} is that evaluation of  affine $\lambda$-terms is  derivable, as stated in the following theorem:
    \announce
    {Theorem~\ref{thm:normalise}}
    {The following function is derivable
    \begin{align*}
    \text{affine $\lambda$-term} \qquad \mapsto \qquad \text{its $\beta$-normal form,}
    \end{align*}
    if we restrict inputs so that they use  a fixed number of variables, and all subterms have types from a fixed finite set of simple types. }
    

\paragraph*{Section~\ref{sec:stt-derivable}: deriving a register transducer.}  In Section~\ref{sec:stt-derivable}, we complete the proof of Theorem~\ref{thm:main}:
    \announce
    {Proposition~\ref{prop:many-register}}
    {Every function computed by a register transducer is derivable.}
    The above result, together with Theorem~\ref{thm:stt}, completes the proof of the left-to-right implication in Theorem~\ref{thm:main}. The proof of Proposition~\ref{prop:many-register} is the observation that register automata with many registers can be reduced to register automata with one register, using the unfolding operation. For register transducers with one register only, the only computation that is done is iterated substitution, which is derivable thanks to Theorem~\ref{thm:normalise}.
