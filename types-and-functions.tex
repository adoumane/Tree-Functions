\section{First-order term functions}
The goal of this paper is to characterise the first-order tree-to-tree functions in terms of some atomic functions and combinators which produce new functions from previously defined ones. A typical combinator is composition of functions.

\subsection{Terms}
For our purposes, the datatype of trees will be insufficient, because  we will need a generalisation of trees where some leaves  are ports   where other trees can be substituted. We use the name \emph{terms} for this generalisation. Fix a set of variables
\begin{align*}
    \termvars = \set{x_1,x_2,\ldots}
\end{align*}
which will be assumed to be disjoint with all other sets.  We view the variables as a ranked set, where each variable has arity zero, i.e.~it can only be used in leaves. 

\begin{definition}[Terms]\label{def:terms}
    An $n$-ary term over a ranked set $\rSigma$ is defined to be a tree over the ranked alphabet $\rSigma + \termvars$ where each of the  variables $x_1,\ldots,x_n$ appears exactly once,  and the other variables do not appear at all.    We write $\tmonad \rSigma$ for the ranked set of terms over $\rSigma$. 
\end{definition}
 
We adopt the following convention for drawing terms. The term is drawn inside a box, and the dangling edges which leave the box at the bottom are its variables, ordered left-to-right, as in the following picture: \mypic{7}
% If we disallow crossing edges in the pictures, then the above drawing convention can only represent terms where the variables are ordered $x_1,\ldots,x_n$ from left to right. 
 The restriction that each variable appears exactly one, sometimes called \emph{linearity}, is connected to the fact that first-order transductions have linear size increase. Because the set of terms is itself a ranked set,  we can create terms of terms, which will be important for the constructions we use.  

To see the need for terms, suppose that we want to group the nodes of a tree into connected parts, called \emph{factors} (a factor in a tree is defined to be a set of nodes that is connected by the parent-child relation). The factors are not trees, because they have dangling edges, but they can be viewed as terms, as explained in the following picture:
\mypic{15}
        
A consequence of using terms is that we will work in the category of ranked sets; i.e.~the types we use will be ranked sets and the functions we consider are arity-preserving functions between ranked sets. We adopt the notation that arity-preserving functions are written in red, like this: $\ranked{f : \Sigma \to \Gamma}$. 


\subsection{Types}
The  main goal of this paper is to  show how every first-order tree-to-tree function can be obtained by starting with certain atomic functions, and composing the using certain combinators. The atomic functions and combinators will be based on arity preserving functions on ranked sets. In this section, we describe the kinds of ranked sets that will be allowed as domains and co-domains of our functions. The general idea is that we start with finite ranked sets, and close these under the following type constructors: coproducts, two kinds of product (Cartesian and tensor), taking terms.
%  and the matrix power. 
% The  matrix power -- which is possibly the least natural type constructor -- will be motivated and discussed in more detail later on.


\begin{definition}[Types] \label{def:types} The following ranked sets are called \emph{types}:
    \begin{enumerate}
        \item  \emph{Finite ranked sets.} Every finite  ranked set is a type. (A finite ranked set is one that  has finitely many elements altogether, in particular only finitely many arities are represented.)
        \item \emph{Coproduct.} If $\ranked{\Sigma_1}$ and $\ranked{\Sigma_2}$ are types, then  their \emph{coproduct} $\ranked{\Sigma_1 + \Sigma_2}$
        is also a type. An $n$-ary element of the coproduct is a pair $(i,a)$ where $i \in \set{1,2}$ and $a$ is an $n$-ary element of  $\ranked{\Sigma_i}$. 
        \item \emph{Cartesian product.} If $\ranked{\Sigma_1}$ and $\ranked{\Sigma_2}$ are types, then  their \emph{Cartesian product}
        $ \ranked{\Sigma_1 \times \Sigma_2}$
        is also a type. An $n$-ary element of the Cartesian product is a pair $(a_1,a_2)$ such that $a_i \in \ranked{\Sigma_i}$ and 
        \begin{align*}
            n = \arity{a_1} = \arity{a_2}.
        \end{align*}
        \item \emph{Tensor product.} If $\ranked{\Sigma_1}$ and $\ranked{\Sigma_2}$ are types, then  their \emph{tensor product}
        $ \ranked{\Sigma_1 \otimes \Sigma_2}$
        is also a type. An $n$-ary element of the Cartesian product is a pair $\tensorpair{a_1,a_2}$ such that $a_i \in \ranked{\Sigma_i}$ and 
        \begin{align*}
            n = \arity{a_1} + \arity{a_2}.
        \end{align*}
        Note how we use  angular brackets to distinguish tensor $\tensorpair{a_1,a_2}$ pairs from Cartesian pairs $(a_1,a_2)$.
        \item \emph{Terms.} If $\ranked \Sigma$ is a type, then  the ranked sets of terms  $\tmonad \rSigma$ is also a type.
    \end{enumerate}
\end{definition}

\subsection{Atomic functions and combinators}
The atomic functions and combinators are listed in Figure~\ref{fig:fo-term}. 
The first two items in the figure, namely function composition and functions with finite domains should be self-explanatory. There are also no surprises for the remaining items which do not involve the term constructor $\tmonad$:
\begin{center}
    \newcommand{\fotitemsmall}[2]{$ #1$ & $#2$ \\ }
\begin{tabular}{ll}
        \fotitemsmall{
            \ranked{\iota_i : \Sigma_i \to \Sigma_1 + \Sigma_2}
            }
            {
                a \mapsto (i,a)
            }
    \fotitemsmall{
        {\ranked{f_1 \text{or} f_2 :  (\Sigma_1 + \Sigma_2) \to  \Gamma}}
        }
        {
            (i,a) \mapsto {\ranked{f_i}(a)}
        }
        \fotitemsmall{
            \ranked{\pi_i : \Sigma_1 \times \Sigma_2 \to \Sigma_i}
            }
            {
                (a_1,a_2) \mapsto a_i
            }
    \fotitemsmall{
        {\ranked{(f_1,f_2) :  \Sigma \to  \Gamma_1 \times \Gamma_2}}
        }
        {
            a \mapsto (\ranked{f_1}(a_1), \ranked{f_2}(a_2))
        }
        \fotitemsmall{
            \ranked{\distrcart : (\Sigma_1 + \Sigma_2)\times \Gamma \to (\Sigma_1 \times \Gamma) + (\Sigma_2 \times \Gamma)}
            }
            {
                \tensorpair{(i,a),b} \mapsto (i,\tensorpair{a,b})
            }
        \fotitemsmall{
        {\ranked{ \tensorpair{f_1,f_2}  :  \Sigma_1 \otimes \Sigma_2 \to  \Gamma_1 \otimes \Gamma_2}}
        }
        {
            \tensorpair{a_1,a_2} \mapsto \tensorpair {\ranked{f_1}(a_1), \ranked{f_2}(a_2)}
        }
        \fotitemsmall{
            \ranked{\distrtensor : (\Sigma_1 + \Sigma_2)\otimes \Gamma \to (\Sigma_1 \otimes \Gamma) + (\Sigma_2 \otimes \Gamma)}
            }
            {
                \tensorpair{(i,a),b} \mapsto (i,\tensorpair{a,b})
            }
\end{tabular}
\end{center}
The functions and combinators involving terms    will be described in more detail in  Section~\ref{sec:atomic-and-combinators}. For the rest of this paper, whenever we refer to ``atomic functions and combinators'', we mean those from Figure~\ref{fig:vocmodels}.  Since all atomic functions are arity preserving functions between types (as per Definition~\ref{def:types}), and the combinators preserve this property, it follows that all functions that can be derived from the atomic functions using combinators are arity preserving. 

We are now ready to state the main theorem of this paper. 

\begin{theorem}\label{thm:main}
    Let $\rSigma,\rGamma$ be finite ranked sets. A function 
    \begin{align*}
        f : \trees \rSigma \to \trees \rGamma
    \end{align*}
    is a first-order tree-to-tree transduction if and only if it is the restriction to trees of some  function
    \begin{align*}
        \ranked {g : \tmonad \Sigma \to \tmonad \Gamma}
    \end{align*}
    which can be derived from the atomic functions using combinators.
\end{theorem}



\newcommand{\fotitem}[2]{$\displaystyle #1$ & #2 \\ \\ }
\begin{figure}[h]
    \centering 
    \begin{tabular}{ll}
        \fotitem{
            \frac
            { \ranked{f : \Sigma \to \Delta} \quad \ranked{g : \Delta \to \Gamma}}
            {\ranked{g \circ f :  \Sigma \to  \Gamma}}
            }
            {
                Composition of functions.
            }
        \fotitem{
            \ranked{f : \Sigma \to \Gamma}
            }
            {
                Every arity preserving function with finite domain $\rSigma$.
            }
            \fotitem{
                \ranked{\iota_i : \Sigma_i \to \Sigma_1 + \Sigma_2}
                }
                {
                    Coprojection for $i \in \set{1,2}$.
                }
        \fotitem{
            \frac
            { \ranked{f_1 : \Sigma_1 \to \Gamma} \quad \ranked{f_2 : \Sigma_2 \to \Gamma}}
            {\ranked{f_1 \text{or} f_2 :  (\Sigma_1 + \Sigma_2) \to  \Gamma}}
            }
            {
                Co-pairing of functions.
            }
            \fotitem{
                \ranked{\pi_i : \Sigma_1 \times \Sigma_2 \to \Sigma_i}
                }
                {
                    Projection for $i \in \set{1,2}$.
                }
        \fotitem{
            \frac
            { \ranked{f_1 : \Sigma \to \Gamma_1} \quad \ranked{f_2 : \Sigma \to \Gamma_2}}
            {\ranked{(f_1,f_2) :  \Sigma \to  \Gamma_1 \times \Gamma_2}}
            }
            {
                Pairing of functions.
            }
            \fotitem{
                \ranked{\distrcart : (\Sigma_1 + \Sigma_2)\times \Gamma \to (\Sigma_1 \times \Gamma) + (\Sigma_2 \times \Gamma)}
                }
                {
                    Cartesian product distributes across coproduct.
                }
            \fotitem{
            \frac
            { \ranked{f_1 : \Sigma_1 \to \Gamma_1} \quad \ranked{f_2 : \Sigma_2 \to \Gamma_2}}
            {\ranked{ \tensorpair{f_1,f_2}  :  \Sigma_1 \otimes \Sigma_2 \to  \Gamma_1 \otimes \Gamma_2}}
            }
            {
                Lifting functions to tensors.
            }
            \fotitem{
                \ranked{\distrtensor : (\Sigma_1 + \Sigma_2)\otimes \Gamma \to (\Sigma_1 \otimes \Gamma) + (\Sigma_2 \otimes \Gamma)}
                }
                {
                    Tensor product distributes across coproduct.
                }
            \fotitem{
            \frac{ \ranked{f : \Sigma \to \Gamma}}{\ranked{\tmonad f : \tmonad \Sigma \to \tmonad \Gamma}}
            }
            {
                Lift a function from the alphabet  to terms.
            }
            \fotitem{
            \ranked{\unit_\Sigma : \Sigma \to \tmonad \Sigma}
            }
            {
                View a letter as a term.
            }
            \fotitem{
                \ranked{\flatt_\Sigma : \tmonad \tmonad \Sigma \to \tmonad \Sigma}
                }
                {
                    Flatten a term of terms into a term.
                }
                \fotitem{
                \ranked{ \composeterm :  
                \set * + \coprod_{a \in \Sigma} \overbrace{\tmonad \rSigma \otimes \cdots \otimes \tmonad \rSigma}^{\text{arity of $a$ times}} \to \tmonad \rSigma }
                }
                {
                        Compose a term from its list of children, for finite $\rSigma$.
                }
            \fotitem{
                \ranked{ \decomposeterm : \tmonad \rSigma \to 
                \set * + \coprod_{a \in \Sigma} \overbrace{\tmonad \rSigma \otimes \cdots \otimes \tmonad \rSigma}^{\text{arity of $a$ times}}}
                }
                {
                        The inverse of $\composeterm$, for finite $\rSigma$.
                }    
    
            \fotitem{
                \ancfact, \decfact  : \ranked{\tmonad(\Sigma_1+\Sigma_2) \to \tmonad(\tmonad \Sigma_1 + \tmonad \Sigma_2)}
            }
                    {
                        Ancestor and descendant factorisations.
                    }
            \fotitem{
                \ranked{\preorder : \tmonad \Sigma \to \tmonad (\rSigma + \set{\grayball, \grayballbin})}
                }
                {
                    Pre-order traversal.
                }    
            % \fotitem{
            %     \ranked{\unfold : \tmonad (\Sigma^{[k]}) \to (\tmonad \Sigma)^{[k]}}
            % }
            %     {
            %             Matrix power distributes across terms.
            %     }
    \end{tabular} 
    
    \caption{First-order term functions}
    \label{fig:fo-term}
\end{figure}





\subsection{The atomic functions and combinators for terms}
\label{sec:atomic-and-combinators}
In this section, we define the atomic functions and combinators from Figure~\ref{fig:fo-term} which involve the term type constructor $\tmonad$. 





\subsubsection{Terms as an algebraic data type} Let $\ranked \Sigma$ be a ranked set. If $a \in \Sigma$ is a letter of arity $n$, and $t_1,\ldots,t_n$ are terms in $\tmonad \rSigma$, then also 
\begin{align*}
    a \tensorpair{t_1,\ldots,t_n}
\end{align*}
is a term, which has $a$ in the root and child subterms $t_1,\ldots,t_n$. We use the tensor brackets $\tensorpair{t_1,\ldots,t_n}$ because the arity of the term is the sum of the arities of $t_1,\ldots,t_n$, as in the tensor product. Another way of constructing a term is to simply have a port $\portletter$ in the root. The above observations explain the function 
\begin{align*}     
    \set * + \coprod_{a \in \Sigma} \overbrace{\tmonad \rSigma \otimes \cdots \otimes \tmonad \rSigma}^{\text{arity of $a$ times}}   \qquad \ranked{\xymatrix{
        \ar[r]^{\composeterm}& 
        %  \ar@<.5ex>[l]^{\composeterm} 
    }} \qquad \tmonad \rSigma
\end{align*}
from Figure~\ref{fig:fo-term}. Note that the coproduct above can only be viewed as a legitimate type when $\rSigma$ is finite; and therefore the function $\composeterm$ is only allowed when $\rSigma$ is finite. The function $\composeterm$ is not injective but not surjective, because its image will contain only the terms that satisfy condition (*) explained in the following picture: 
\mypic{28}
If we assumed, in definition~\ref{def:terms}, that the variables in a term are ordered left-to-right, then $\composeterm$ would be a bijection. The operation 
\begin{align*}     
    \set {\bot,*} + \coprod_{a \in \Sigma} \overbrace{\tmonad \rSigma \otimes \cdots \otimes \tmonad \rSigma}^{\text{arity of $a$ times}}   \qquad \ranked{\xymatrix{
&          \ar[l]^{\decomposeterm} 
    }} \qquad \tmonad \rSigma
\end{align*}
is a one-sided inverse of $\composeterm$, i.e.~it is output is the inverse of $\composeterm$ for terms that satisfy (*) and otherwise it is $\bot$. 


\subsubsection{Terms as a monad.} Three of the operations associated to the type constructor $\tmonad$, namely  
\begin{align*}
        \underbrace{\ranked {\tmonad f : \tmonad \Sigma \to \tmonad \Gamma}}_{\text{$\tmonad$ is a functor}} \qquad  \underbrace{\unit_\rSigma : \rSigma \rto \tmonad \rSigma}_{\text{the unit in the monad}} \qquad  \underbrace{\flatt_\rSigma : \tmonad \tmonad \rSigma \rto \tmonad \rSigma}_{\text{the product operation in the monad}}
\end{align*}
induce a monad structure on terms (in the category of ranked sets). The lifting $\tmonad f$ should be self-explanatory: one simply changes the labels in the term according to $f$. The  unit operation interprets each letter as a term with one non-variable node, with the variables ordered left-to-right,  as illustrated below:
        \mypic{10}
Finally, the  \emph{flattening} operation $\flatt_\rSigma$ 
        maps a term of terms into a term, by using substitution as illustrated in the following picture:  \mypic{9}
% A more formal definition is the flattening of $\portletter$ is $\portletter$, while the flattening of a term of the form 
% \begin{align*}
%     t \tensorpair{t_1,\ldots,t_n} \qquad t \in \tmonad \rSigma, t_1,\ldots,t_n \in \tmonad \tmonad \rSigma    
% \end{align*}
%  is the term obtained by taking $t$, and replacing each port $x_i$ by the flattening of $t_i$.
    
\subsubsection{Factorisations}
    Let $t \in \tmonad \rSigma$ be a term. 
    There are two equivalent definitions of factorisations of $t$. One definition is that a factorisation is an equivalence relation on non-port nodes where every equivalence class is a factor; the other definition is that it is any term $s  \in \tmonad \tmonad \rSigma$ which flattens to $t$. 
    The two definitions are easily seen to be equivalent equivalent, in the sense that there is a one-to-one correspondence between factorisation equivalences and factorisation terms.
    %  which is explained in the following picture:
    % \mypic{14}
    Suppose that $\ranked{\Sigma_1}$ and $\ranked{\Sigma_2}$ are ranked sets. The ancestor and descendant factorisations 
        \begin{align*}
            \overbrace{\ancfact}^{\text{ancestor}}, \overbrace{\decfact}^{\text{descendant}}  : \ranked{\tmonad(\Sigma_1+\Sigma_2) \to \tmonad(\tmonad \Sigma_1 + \tmonad \Sigma_2)}
        \end{align*}
        are defined as follows. Consider an input term
        \begin{align*}
            t \in \ranked{\tmonad(\Sigma_1+\Sigma_2)}.
        \end{align*}
        We say that two non-port nodes have \emph{same type} if both have labels in the same  $\ranked{\Sigma_i}$; otherwise we say that non-port nodes have \emph{opposing type}.  Call two non-port nodes \emph{ancestor equivalent}  if they  same proper ancestors of opposing type. Call two non-port nodes \emph{descendant equivalent}  if they  are ancestor equivalent and they have the same proper descendants of opposing type. Here is a picture, with $\ranked{\Sigma_1}$ being red and $\ranked{\Sigma_2}$ being blue:
        \mypic{11}
        Both ancestor and descendant equivalences are factorisations; and in each case equivalence classes contain only nodes of same type.  The function $\ancfact$ maps a term to (the term of terms corresponding to) its ancestor equivalence relation, likewise we define $\decfact$ for  descendant factorisations.
    
        \subsubsection{Pre-order traversal.} Let  $\rSigma$ be a ranked set. Let \grayball and \grayballbin be two letters, of arities zero and two. Define 
        \begin{align*}
            \ranked{\preorder : \tmonad \Sigma \to \tmonad (\rSigma + \set{\grayball, \grayballbin})}
        \end{align*}
        to be the extension to  terms of the pre-order function discussed in Example~\ref{ex:pre-order}, as explained below:
        \mypic{13}
        

