\section{Derivable functions}
The goal of this paper is to characterise the first-order tree-to-tree functions in terms of a set of  atomic functions together with combinators which produce new functions from previously defined ones. A typical combinator is composition of functions.

\subsection{Terms}
We begin by introducing terms, which are a generalisation of trees, where some parts of the tree are unknown. We will need to use terms instead of trees, because we will be splitting  trees into smaller pieces, and such smaller pieces are going to be terms. This is a difference of trees as opposed to strings: while a piece  of a string is also a string, a piece of a tree is a term.

Fix a set of variables
\begin{align*}
    \termvars = \set{x_1,x_2,\ldots}
\end{align*}
which will be assumed to be disjoint with all other sets.  We view the variables as a ranked set, where each variable has arity zero, i.e.~it can only be used in leaves. 

\begin{definition}[Terms]\label{def:terms}
    An $n$-ary term over a ranked set $\rSigma$ is defined to be a tree over the ranked alphabet $\rSigma + \termvars$ where each of the  variables $x_1,\ldots,x_n$ appears exactly once,  and the other variables do not appear at all.  We call \emph{ports} the nodes labeled by $\termvars$.  We write $\tmonad \rSigma$ for the ranked set of terms over $\rSigma$.  
\end{definition}

In universal algebra, the restriction that each variable of a term appears exactly once is sometimes called \emph{linearity}; since we do not consider non-linear terms, we use the word term without additionally specifying that it is linear.  The linearity restriction is connected to the fact that first-order transductions have linear size increase.

We adopt the following convention for drawing terms. The term is drawn inside a box, and the dangling edges which leave the box at the bottom are its variables, ordered left-to-right, as in the following picture: \mypic{7}
% If we disallow crossing edges in the pictures, then the above drawing convention can only represent terms where the variables are ordered $x_1,\ldots,x_n$ from left to right. 
 Because the set of terms is itself a ranked set,  we can create terms of terms, which will be useful.

To see the need for terms, suppose that we want to group the nodes of a tree into connected parts, called \emph{factors} (a factor in a tree is defined to be a set of nodes that is connected by the parent-child relation). The factors are not trees, because they have dangling edges, but they can be viewed as terms, as explained in the following picture:
\mypic{15}
        
A consequence of using terms is that we will work in the category of ranked sets; i.e.~the types we use will be ranked sets and the functions we consider are arity-preserving functions between ranked sets. We adopt the notation that arity-preserving functions are written in red, like this: $\ranked{f : \Sigma \to \Gamma}$. 


\subsection{Types}
The  main goal of this paper is to  show how every first-order tree-to-tree function can be obtained by starting with certain atomic functions, and composing them using certain combinators. The atomic functions and combinators will be based on arity preserving functions on ranked sets. In this section, we describe the types of ranked sets that will be allowed as domains and co-domains of our functions. The general idea is that we start with finite ranked sets, and close these under the following type constructors: coproducts, two kinds of product (Cartesian and tensor), taking terms, and a type constructor that allows to merge ports.
%  and the matrix power. 
% The  matrix power -- which is possibly the least natural type constructor -- will be motivated and discussed in more detail later on.


\begin{definition}[Types] \label{def:types} The following ranked sets are called \emph{types}:
    \begin{enumerate}
        \item  \emph{Finite ranked sets.} Every finite  ranked set is a type. (A finite ranked set is one that  has finitely many elements altogether, in particular only finitely many arities are represented.)
        \item \emph{Coproduct.} If $\ranked{\Sigma_1}$ and $\ranked{\Sigma_2}$ are types, then  their \emph{coproduct} $\ranked{\Sigma_1 + \Sigma_2}$
        is also a type. An $n$-ary element of the coproduct is a pair $(i,a)$ where $i \in \set{1,2}$ and $a$ is an $n$-ary element of  $\ranked{\Sigma_i}$. 
        \item \emph{Cartesian product.} If $\ranked{\Sigma_1}$ and $\ranked{\Sigma_2}$ are types, then  their \emph{Cartesian product}
        $ \ranked{\Sigma_1 \times \Sigma_2}$
        is also a type. An $n$-ary element of the Cartesian product is a pair $(a_1,a_2)$ such that $a_i \in \ranked{\Sigma_i}$ and 
        \begin{align*}
            n = \arity{a_1} = \arity{a_2}.
        \end{align*}
        We draw Cartesian pairs like this:
        \mypic{51}
        \item \emph{Tensor product.} If $\ranked{\Sigma_1}$ and $\ranked{\Sigma_2}$ are types, then  their \emph{tensor product}
        $ \ranked{\Sigma_1 \otimes \Sigma_2}$
        is also a type. An $n$-ary element of the Cartesian product is a pair $\tensorpair{a_1,a_2}$ such that $a_i \in \ranked{\Sigma_i}$ and 
        \begin{align*}
            n = \arity{a_1} + \arity{a_2}.
        \end{align*}
        Note how we use  angular brackets to distinguish tensor $\tensorpair{a_1,a_2}$ pairs from Cartesian pairs $(a_1,a_2)$.         We draw tensor pairs like this:
        \mypic{52}

        \item \emph{Folding.} If $\rSigma$ is a type and $k \in \set{2,3,\ldots}$ then $\reduce k \rSigma$ is a type. Elements of this set are called $k$-folds of $\rSigma$. A $k$-fold of arity $n$ consists of an element   $a \in \rSigma$ of arity $nk$ and  a function
        \begin{align*}
            f : \set{1,\ldots,nk} \to \set{1,\ldots,n} \qquad \text{every inverse image $f^{-1}(i)$ has size exactly $k$}
        \end{align*}
        We write $a/f$ for the $k$-fold given by $a$ and $f$.  We draw $k$-folds like this:
        \mypic{53}
        \item \emph{Terms.} If $\ranked \Sigma$ is a type, then  the ranked sets of terms  $\tmonad \rSigma$ is also a type.
    \end{enumerate}
\end{definition}

\subsection{Atomic functions and combinators}
We now introduce the main new definition of this paper.

\begin{definition}[Derivable function]
    A function is called \emph{derivable} if it can be constructed using the atomic functions and combinators listed in Figure~\ref{fig:fo-term}. 
\end{definition}



The first two items in the figure, namely function composition and functions with finite domains should be self-explanatory. There are also no surprises for the remaining items which do not involve the term constructor $\tmonad$:
\begin{center}
    \newcommand{\fotitemsmall}[2]{$ #1$ & $#2$ \\ }
\begin{tabular}{ll}
        \fotitemsmall{
            \ranked{\iota_i : \Sigma_i \to \Sigma_1 + \Sigma_2}
            }
            {
                a \mapsto (i,a)
            }
    \fotitemsmall{
        {\ranked{f_1 \text{or} f_2 :  (\Sigma_1 + \Sigma_2) \to  \Gamma}}
        }
        {
            (i,a) \mapsto {\ranked{f_i}(a)}
        }
        \fotitemsmall{
            \ranked{\pi_i : \Sigma_1 \times \Sigma_2 \to \Sigma_i}
            }
            {
                (a_1,a_2) \mapsto a_i
            }
    \fotitemsmall{
        {\ranked{(f_1,f_2) :  \Sigma \to  \Gamma_1 \times \Gamma_2}}
        }
        {
            a \mapsto (\ranked{f_1}(a_1), \ranked{f_2}(a_2))
        }
        \fotitemsmall{
            \ranked{\distrcart : (\Sigma_1 + \Sigma_2)\times \Gamma \to (\Sigma_1 \times \Gamma) + (\Sigma_2 \times \Gamma)}
            }
            {
                ((i,a),b) \mapsto (i,(a,b))
            }
        \fotitemsmall{
        {\ranked{ \tensorpair{f_1,f_2}  :  \Sigma_1 \otimes \Sigma_2 \to  \Gamma_1 \otimes \Gamma_2}}
        }
        {
            \tensorpair{a_1,a_2} \mapsto \tensorpair {\ranked{f_1}(a_1), \ranked{f_2}(a_2)}
        }
        \fotitemsmall{
            \ranked{\distrtensor : (\Sigma_1 + \Sigma_2)\otimes \Gamma \to (\Sigma_1 \otimes \Gamma) + (\Sigma_2 \otimes \Gamma)}
            }
            {
                \tensorpair{(i,a),b} \mapsto (i,\tensorpair{a,b})
            }
        \fotitemsmall{
            \ranked{\reduce k f : \reduce k \Sigma \to \reduce k \Gamma }
            }
            {
                a/g \mapsto \ranked f(a)/g
            }
\end{tabular}
\end{center}
The functions and combinators involving terms    will be described in more detail in  Section~\ref{sec:atomic-and-combinators}.   

For a derivable function, its domain and co-domain are types as in Definition~\ref{def:types}, in particular the domain and co-domain are ranked sets. The atomic functions are arity-preserving and the combinators preserve this property, and therefore all derivable functions are arity-preserving. 


We are now ready to state the main theorem of this paper. 

\begin{theorem}\label{thm:main}
    Let $\rSigma,\rGamma$ be finite ranked sets. A function 
    \begin{align*}
        f : \trees \rSigma \to \trees \rGamma
    \end{align*}
    is a first-order tree-to-tree transduction if and only if it is the restriction to trees of some derivable
    \begin{align*}
        \ranked {f : \tmonad \Sigma \to \tmonad \Gamma}.
    \end{align*}
    
\end{theorem}



\newcommand{\simplefunfig}[4]{
    \begin{tabular}{cc}
        $\ranked{
        \xymatrix@C=1.5cm{
#2 \ar[r]^-{#1}& #3
        }}$
        \\
        {#4}
    \end{tabular}   
 }

 \newcommand{\reversiblefunfig}[4]{
    \begin{tabular}{cc}
        $\ranked{
        \xymatrix@C=1.5cm{
#2 \ar@<.5ex>[r]^-{#1}& #3
\ar@<.5ex>[l]
        }}$
        \\
        {#4}
    \end{tabular}   
 }



 \newcommand{\laterfunfig}[3]{
    \begin{tabular}{cc}
        $\ranked{
        \xymatrix@C=1.5cm{
#1 & #2
        }}$
        \\
#3
    \end{tabular}   
 }
 



\begin{figure}
    \begin{tabular}{cc}
        \simplefunfig
        {\iota_i}
        {\Sigma_i}
        {\Sigma_1 + \Sigma_2}
        {$a \mapsto (a,i)$}
        &
        \simplefunfig
        {\mathrm{forget}}
        {\Sigma+\Sigma}
        {\Sigma}
        {$(a,i) \mapsto a$}
        \\ \\
        \reversiblefunfig
        {\mathrm{swap}}
        {\Sigma \otimes \Gamma}
        {\Gamma \otimes \Sigma}
        {$\tensorpair{a,b} \mapsto \tensorpair{b,a}$}
        &
        \reversiblefunfig
        {\distrtensor}
        {(\Sigma_1 + \Sigma_2)\otimes \Gamma}
        {(\Sigma_1 \otimes \Gamma) + (\Sigma_2 \otimes \Gamma)}
        {$\tensorpair{(a,i),b} \mapsto (\tensorpair{a,b},i)$}
        \\ \\
        \reversiblefunfig
        {\distrtensor}
        {\shallowterm {(\Sigma_1 + \Sigma_2)} \Gamma}
        {(\shallowterm {\Sigma_1} \Gamma) + (\shallowterm {\Sigma_2} \Gamma)}
        {$(a,i)\tensorpair{b_1,\ldots,b_n} \mapsto (a\tensorpair{b_1,\ldots,b_n})$ }
        &
        \reversiblefunfig
        {\distrtensor}
        {\shallowterm {(\Sigma_1 \otimes \Sigma_2)} \Gamma}
        {(\shallowterm {\Sigma_1} \Gamma) \otimes (\shallowterm {\Sigma_2} \Gamma)}
        {
        \begin{tabular}{c}
            $\tensorpair{a_1,a_2}\tensorpair{b_1,\ldots,b_n} \mapsto$ \\
            $\tensorpair{a_1 \tensorpair{b_1,\ldots,b_{n_1}}, a_2\tensorpair{b_{n_1+1},\ldots,b_n} }$  \\
            where $n_1$ is the arity of $a_1$
        \end{tabular}    
        }
        \\ \\
        \reversiblefunfig
        {}
        {\rSigma}
        {\shallowterm 1 {\tmonad \Sigma}}
        {$a \mapsto 1.\tensorpair a$}
        &
        \reversiblefunfig
        {\composeterm}
        {1 + \shallowterm \Sigma {\tmonad \Sigma} }
        { \tmonad \Sigma}
        {
        \begin{tabular}{c}
            Every term is either just a port,\\ or has a root and child subterms.    
        \end{tabular}    
        } 
        \\ \\
        \simplefunfig
        {}
        {\Sigma}
        {\reduce k \Sigma}
        {$a \mapsto a/(i \mapsto (1,i))$ } &
        \simplefunfig
        {}
        {\reduce {k_1} \reduce {k_2} \Sigma}
        {\reduce {k_1 \cdot k_2} \Sigma}
        {$(a/f)/g \mapsto a/(g \circ f)$}
        \\ \\
        \simplefunfig
        {\unit_\Sigma}
        {\Sigma}
        {\tmonad \Sigma}
        {\tablepic{69}}
        & \\ \\
        \simplefunfig
        {\flatt_\Sigma}
        {\tmonad \tmonad \Sigma}
        {\tmonad \Sigma} 
        {\tablepic{73}}
        &
        \simplefunfig
        {\distrtf}
        { \tmonad \reduce 1 \Sigma}
        {\reduce 1 \tmonad \Sigma}
        {\tablepic{74}}
        \\ \\
        \laterfunfig
        {\tmonad(\Sigma_1+\Sigma_2) \ar@<.5ex>[r]^{ \ancfact}
        \ar@<-.5ex>[r]_{\decfact}}
        {\tmonad(\tmonad \Sigma_1 + \tmonad \Sigma_2)} 
        {see Section~\ref{sec:atomic-and-combinators}} 
        &
        \laterfunfig
        {\tmonad \Sigma \ar[r]^-{\preorder}}
        {\reduce 1 \tmonad(\rSigma + 0 + 2)}
        {see Section~\ref{sec:atomic-and-combinators}} 
        \\ \\ 
        \laterfunfig
        {\shallowterm{(\reduce k \Sigma)}{\Gamma^k} \ar[r]^-{\unfold}}
        {(\shallowterm \Sigma \Gamma)^k}
        {see Section~\ref{sec:atomic-and-combinators}} 
    \end{tabular} 
    \caption{    \label{fig:fo-term}The atomic functions. The functions are parametrised by types  $\rSigma, \ranked{\Sigma_1}, \ranked{\Sigma_2}, \rGamma$. In the above, the type $\ranked i$ for $i \in \set{0,1,2}$ represents a  ranked set   with one element of arity $i$. Some of the functions are bijections, as indicated by double arrows, in these cases both the function and its inverse are atomic functions. }
\end{figure}








\subsection{The atomic functions and combinators for terms}
\label{sec:atomic-and-combinators}
In this section, we define the atomic functions and combinators from Figure~\ref{fig:fo-term} which involve the term type constructor $\tmonad$. 



\subsubsection{Terms as an algebraic data type} Let $\ranked \Sigma$ be a ranked set. If $a \in \Sigma$ is a letter of arity $n$, and $t_1,\ldots,t_n$ are terms in $\tmonad \rSigma$, then also 
\begin{align*}
    a \tensorpair{t_1,\ldots,t_n}
\end{align*}
is a term, which has $a$ in the root and child subterms $t_1,\ldots,t_n$. We use the tensor brackets $\tensorpair{t_1,\ldots,t_n}$ because the arity of the term is the sum of the arities of $t_1,\ldots,t_n$, as in the tensor product. Another way of constructing a term is to simply have a port $\portletter$ in the root. The above observations explain the function 
\begin{align*}     
    \set * + \coprod_{a \in \Sigma} \overbrace{\tmonad \rSigma \otimes \cdots \otimes \tmonad \rSigma}^{\text{arity of $a$ times}}   \qquad \ranked{\xymatrix{
        \ar[r]^{\composeterm}& 
        %  \ar@<.5ex>[l]^{\composeterm} 
    }} \qquad \tmonad \rSigma
\end{align*}
from Figure~\ref{fig:fo-term}. Note that the coproduct above can only be viewed as a legitimate type when $\rSigma$ is finite; and therefore the function $\composeterm$ is only allowed when $\rSigma$ is finite. The function $\composeterm$ is injective but not surjective, because its image will contain only the terms that satisfy condition (*) explained in the following picture: 
\mypic{28}
If we assumed, in definition~\ref{def:terms}, that the variables in a term are ordered left-to-right, then $\composeterm$ would be a bijection. The operation 
\begin{align*}     
    \set {\bot,*} + \coprod_{a \in \Sigma} \overbrace{\tmonad \rSigma \otimes \cdots \otimes \tmonad \rSigma}^{\text{arity of $a$ times}}   \qquad \ranked{\xymatrix{
&          \ar[l]^{\decomposeterm} 
    }} \qquad \tmonad \rSigma
\end{align*}
is a one-sided inverse of $\composeterm$, i.e.~it is output is the inverse of $\composeterm$ for terms that satisfy (*) and otherwise it is $\bot$. 


\subsubsection{Terms as a monad.} Three of the operations associated to the type constructor $\tmonad$, namely  
\begin{align*}
        \underbrace{\ranked {\tmonad f : \tmonad \Sigma \to \tmonad \Gamma}}_{\text{$\tmonad$ is a functor}} \qquad  \underbrace{\unit_\rSigma : \rSigma \rto \tmonad \rSigma}_{\text{the unit in the monad}} \qquad  \underbrace{\flatt_\rSigma : \tmonad \tmonad \rSigma \rto \tmonad \rSigma}_{\text{the product operation in the monad}}
\end{align*}
induce a monad structure on terms (in the category of ranked sets). The lifting $\tmonad f$ should be self-explanatory: one simply changes the labels in the term according to $f$. The  unit operation interprets each letter as a term with one non-variable node, with the variables ordered left-to-right,  as illustrated below:
        \mypic{10}
Finally, the  \emph{flattening} operation $\flatt_\rSigma$ 
        maps a term of terms into a term, by using substitution as illustrated in the following picture:  \mypic{9}
% A more formal definition is the flattening of $\portletter$ is $\portletter$, while the flattening of a term of the form 
% \begin{align*}
%     t \tensorpair{t_1,\ldots,t_n} \qquad t \in \tmonad \rSigma, t_1,\ldots,t_n \in \tmonad \tmonad \rSigma    
% \end{align*}
%  is the term obtained by taking $t$, and replacing each port $x_i$ by the flattening of $t_i$.
    
\subsubsection{Factorisations}
    Let $t \in \tmonad \rSigma$ be a term. 
    There are two equivalent definitions of factorisations of $t$. One definition is that a factorisation is an equivalence relation on non-port nodes where every equivalence class is a factor; the other definition is that it is any term $s  \in \tmonad \tmonad \rSigma$ which flattens to $t$. 
    The two definitions are easily seen to be equivalent, in the sense that there is a one-to-one correspondence between factorisation equivalences and factorisation terms.
    %  which is explained in the following picture:
    % \mypic{14}
    Suppose that $\ranked{\Sigma_1}$ and $\ranked{\Sigma_2}$ are ranked sets. The ancestor and descendant factorisations 
        \begin{align*}
            \overbrace{\ancfact}^{\text{ancestor}}, \overbrace{\decfact}^{\text{descendant}}  : \ranked{\tmonad(\Sigma_1+\Sigma_2) \to \tmonad(\tmonad \Sigma_1 + \tmonad \Sigma_2)}
        \end{align*}
        are defined as follows. Consider an input term
        \begin{align*}
            t \in \ranked{\tmonad(\Sigma_1+\Sigma_2)}.
        \end{align*}
        We say that two non-port nodes have \emph{same type} if both have labels in the same  $\ranked{\Sigma_i}$; otherwise we say that non-port nodes have \emph{opposing type}.  Call two non-port nodes \emph{ancestor equivalent}  if they have the same proper ancestors of opposing type. Call two non-port nodes \emph{descendant equivalent}  if they  are ancestor equivalent and they have the same proper descendants of opposing type. Here is a picture, with $\ranked{\Sigma_1}$ being red and $\ranked{\Sigma_2}$ being blue: \todo{les images ne collent pas avec la def.}
        \mypic{11}
        Both ancestor and descendant equivalences are factorisations; and in each case equivalence classes contain only nodes of same type.  The function $\ancfact$ maps a term to (the term of terms corresponding to) its ancestor equivalence relation, likewise we define $\decfact$ for  descendant factorisations.
    
        \subsubsection{Pre-order traversal.} Let  $\rSigma$ be a ranked set. Let \grayball and \grayballbin be two letters, of arities zero and two. Define 
        \begin{align*}
            \ranked{\preorder : \tmonad \Sigma \to \tmonad (\rSigma + \set{\grayball, \grayballbin})}
        \end{align*}
        to be the extension to  terms of the pre-order function discussed in Example~\ref{ex:pre-order}, as explained below:
        \mypic{13}

\todo{Il me semble qu'il manque des operations sur le type Fold. Je propose de faire une section specifique a fold (comme celle pour les terms), avant une section specifique pour l'unfolding, ou on ajouterai, en plus de certaines operations de distributitivite, les operations de "quasi-monad'' sur fold. Il est necessaire d'ajouter la fonctorialite de fold, mais peut etre aussi ce qui ferait office de unit et de flat.}        

\subsubsection{Unfolding}
The general idea behind the  operation 
\begin{align*}
    \ranked{\unfold : \tmonad  \reduce k (\Sigma \otimes \cdots \otimes \Sigma) \to \reduce k ( \tmonad \rSigma \otimes \cdots \otimes \tmonad \rSigma)} 
\end{align*}
is to unfold the edges that are joined using the fold constructor, by matching them with different coordinates of the tensor power in the alphabet. This operation is described in the following picture:
\mypic{39}
To explain unfolding more precisely, we reduce it to simpler parts, by using a special case of terms, called shallow terms. Define a \emph{shallow term} over a ranked set $\rSigma$ and $\rGamma$ to be the special case of  a term over $\rSigma + \rGamma$ where:  the root has label from $\rGamma$, all children of the root are non-port nodes from $\rGamma$, and all grand-children are ports. Here is a picture of a shallow term:
\mypic{54}
It will be easier to define the unfold operation for shallow terms, and then unfolding can be lifted to general terms using structural induction. 
Before explaining the unfolding operation, we describe  the operations associated to shallow terms. First, we can lift functions on the underlying types to shallow terms in the natural way:
\begin{align*}
    \frac
    { \ranked{f : \Sigma \to \Sigma'} \quad \ranked{g : \Gamma \to \Gamma'}}
    {\ranked{\shallowterm f g : \shallowterm \Sigma \Gamma \to \shallowterm {\Sigma'} {\Gamma'}}}
\end{align*}
Every term over $\rSigma$ is either a port, or it is a shallow term over $\rSigma$ and $\tmonad \rSigma$, leading to the following operations:
\begin{align*}
    \ranked{\decomposeterm : \tmonad \Sigma \to 1 +  
\shallowterm \Sigma {\tmonad \Sigma}} \qquad  \ranked{\composeterm : 1 +  
\shallowterm \Sigma {\tmonad \Sigma} \to \tmonad \Sigma }
\end{align*}
We can distribute fold across shallow terms using the  operation
\begin{align*}
    \ranked{\distrfold : \shallowterm \Sigma {\reduce k \Gamma} \to \reduce k (\shallowterm \Sigma  \Gamma)}
\end{align*}
as explained in the following picture:
\mypic{55}
Finally, we have the shallow unfolding operation, which eliminates folding   by matching it with $k$-th tensor power in a shallow term:
\begin{align*}
\ranked{
    \unfoldsmall : \shallowterm {\reduce k   \Sigma} {(\overbrace{\Gamma \otimes \cdots \otimes \Gamma}^{\text{$k$ times}})} \to \shallowterm \Sigma \Gamma
}
\end{align*}
The unfolding operation is explained in the following picture for $k=3$:
\mypic{56}

\newcommand{\expmatrix}[1]{\reduce k \kpower{#1}}
\newcommand{\kpower}[1]{ \overbrace{(#1 \otimes \cdots \otimes #1)}^{\text{$k$ times}}}
 

Using the shallow unfold operation described above, we now give a formal definition of the unfold operation. Suppose that we want to compute the unfold of a term of type 
\begin{align*}
\ranked{\tmonad \expmatrix \Sigma}.
\end{align*}
If the input is the identity term ,  then its unfolding is 
\begin{align*}
    \tensorpair{\portletter,\ldots,\portletter}/(i \mapsto 1),
\end{align*}
i.e.~it consists of $k$ identity terms with their ports folded into a single one.  Otherwise, the term can can be decomposed as a shallow term of the form: 
\begin{align*}
 \ranked{
    \shallowterm{ \expmatrix \Sigma} { \tmonad \expmatrix \Sigma}
}
\end{align*}
Unfold inductively  the children in the shallow term above, yielding a result in
\begin{align*}
   \ranked{
        \shallowterm{ \expmatrix \Sigma} {  \expmatrix {\tmonad \Sigma}}
    }
    \end{align*}
To the result, apply the composition of the following operations, yielding the final result
\begin{align*}
\ranked{
    \xymatrix@C=2cm{
        \shallowterm{\expmatrix \Sigma}{\expmatrix  \Sigma} \ar[r]^{\distrfold} &
        \reduce k (\shallowterm {\expmatrix \Sigma } {\kpower \Sigma}) \ar[d]^-{\reduce k \unfoldsmall}\\
        \reduce k \kpower{\shallowterm \Sigma \Sigma} &
        \reduce k (\shallowterm {\kpower \Sigma} \Sigma) \ar[l]_{\reduce k \distrtensor}
    }
}
\end{align*}

% Distribute out the fold in the children of the shallow term, yielding  a result in
% \begin{align*}
%       \ranked{ \reduce k(
%         \shallowterm{ \expmatrix \Sigma} { \kpower{\tmonad \Sigma})}
%     }
%     \end{align*}
% Apply the shallow unfold (inside the fold), yielding  a result of type
% \begin{align*}
%   \ranked{ \reduce k(
%     \shallowterm{ \kpower \Sigma} {\tmonad \Sigma})}
% \end{align*}
% Distribute the tensor across the shallow term, yielding a result of type
% \begin{align*}
%     \ranked{ \reduce k
%       \kpower {(\shallowterm \Sigma {\tmonad \Sigma})}}
%   \end{align*}
% Compose the shallow terms into (not shallow) terms, yielding the final result  in
% \begin{align*}
%     \ranked{ \reduce k
%       \kpower {(\tmonad \Sigma)}}
%   \end{align*}
This completes the definition of the unfold operation. 