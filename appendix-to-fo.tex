
\section{First-order term functions can be described in logic}
\label{sec:to-logic}


\begin{definition} Let $\Sigma, \Gamma\in \Tt$. We say that a function $f:\Sigma\to \Gamma$ is \emph{definable by an FO-transduction} if there is some FO-transduction $\varphi$ which makes the following diagram commutes:
    \[\begin{array}{lll}
    \Sigma& \stackrel{t\mapsto \underline{t}}{\xrightarrow{\hspace*{3cm}}} & \text{structures over }\underline \Sigma \\
    f&&\varphi\\
    \Gamma& \stackrel{t\mapsto \underline{t}}{\xrightarrow{\hspace*{3cm}}}  & \text{structures over }\underline \Gamma 
    \end{array}\]
    \end{definition}
    
    \begin{proposition}
    Let $\Gamma, \Sigma \in \Tt$. If $f:\Sigma\to\Gamma$ is a first-order term function, then it is definable by an FO-transduction. 
    \end{proposition}
    
    \begin{proof}
    The proof is by induction, following the defintion of first-order term functions. The atomic functions $\llone$, constant functions, projection, co-projection and distributions are easily definable by FO-transductions.  
    We prove now that are also definable by FO-ransductions. 
    \\
    
    In the following we adopt the following shortcuts:
    \begin{align*}
    \sf{FirstPort}_\Sigma (x)& := \Port\Sigma(x)\wedge \neg (\exists y. \Port\Sigma(y)\wedge y\prec_\Sigma x)\\
    y=\sf{succ}_< (x) &:=  y<x \wedge \neg (\exists z. y<z \wedge z<x)
    \end{align*} 
    \begin{enumerate}
    \item $\sf{Flat}$. Given a tree $t$ of type $\tmonad\tmonad\Sigma$, the fucntion $\sf{Flat}(t)$ can be implemented using an FO-transduction as follows: the copying constant is 1, the universe formula is 
    \[\varphi_1(x) = \neg (\Root{\tmonad\Sigma}(x) \vee \Port{\tmonad\Sigma}(x) )  \]
    For every relation symbol $R\in\underline \Sigma$ , we conserve the same interpretation:
    \[\varphi_R^{1,1}(x,y)=R(x,y) \text{ if $R$ is binary }
    \quad \text{ and } \quad\varphi_R^{1}(x)=R(x) \text{ if $R$ is unary }\] 
    For the other symbols of $\underline\tmonad\Sigma$, that is $\Root{\tmonad\Sigma},\Port{\tmonad\Sigma}, \prec_{\tmonad\Sigma}, \leq_{\tmonad\Sigma}$, we "shift by one $\tmonad$" their interpretation, more precisely, if $R\in\{\Root{}, \Port{}, \prec, \leq\}$, we set:
    \[ \varphi^1_{R_{\tmonad\Sigma}}(x) = R_{\tmonad\tmonad\Sigma}(x) \text{ if $R_{\tmonad\Sigma}$ is unary } \]
    \[\varphi^{1,1}_{R_{\tmonad\Sigma}}(x,y) = R_{\tmonad\tmonad\Sigma}(x,y) \text{ if it is binary. }\] 
    \item $\sf{Block}$. We show how to simulation the function $\sf{Block}:\tmonad (\Sigma + \Gamma) \to \tmonad (\tmonad \Sigma + \tmonad \Gamma)$ by an FO-transduction. 
    \item $\sf{Comb}$. The function $\sf{Comb}:\tmonad \llone\to\tmonad\llone$ is implemented by an FO-transduction as follows. Th copying constant is 4. 
    In the first copy we keep the root and the ports, and in the other 3 copies we keep all the ports exept the first one:
    \begin{align*}
    \varphi_1(x)&=\Root{\tmonad\llone}(x) \vee \Port{\tmonad\llone}(x)\\
    \varphi_i(x)&=\Port{\tmonad\llone}(x)\wedge \neg \sf{FirstPort}_{\tmonad\llone}(x)& i=2,3,4
    \end{align*}
    
    \begin{align*}
    \varphi^1_{\Root{\tmonad\llone}}(x)=\Root{\tmonad\llone}(x) \qquad \varphi^1_{\Port{\tmonad\llone}}(x)=\Port{\tmonad\llone}(x)
    \qquad\varphi^{1,1}_{\prec_{\tmonad\llone}}(x,y)=\prec_{\tmonad\llone}(x,y) 
    \end{align*}
    
    \begin{align*}
    \varphi^{2,3}_{\leq_\llone}(x,y) = \varphi^{2,4}_{\leq_\llone}(x,y) = (x=y)\\
    \varphi^{4,3}_{\prec_\llone}(x,y) =(x=y)
    \end{align*}
    
    \begin{align*}
    \varphi^{2,3}_{<_{\tmonad\llone}}(x,y)= \varphi^{2,4}_{<_{\tmonad\llone}}(x,y)= ((x=y) \vee (x\prec_{\tmonad\llone} y))\\
    \varphi^{4,2}_{<_{\tmonad\llone}}(x,y)=\varphi^{4,3}_{<_{\tmonad\llone}}(x,y)=\varphi^{2,2}_{<_{\tmonad\llone}}(x,y)= \varphi^{4,4}_{<_{\tmonad\llone}}(x,y)= (x\prec_{\tmonad\llone} y)\\
    \varphi^{1,1}_{<_{\tmonad\llone}}(x,y)=\varphi^{1,2}_{<_{\tmonad\llone}}(x,y)= \Root{\tmonad\llone}(x)\\
    \varphi^{2,1}_{<_{\tmonad\llone}}(x,y)=\varphi^{3,1}_{<_{\tmonad\llone}}(x,y)=(x=y) \vee (y\prec_{\tmonad\llone} x)\qquad 
    \varphi^{4,1}_{<_{\tmonad\llone}}(x,y)=(y\prec_{\tmonad\llone} x)
    \end{align*}
    \begin{center}
    Add pictures and explanations.
    \end{center}
    \item $\sf{Unfold}$. Let us now consider the function $\sf{Unfold}:\tmonad(\Sigma\otimes \Gamma)\to \tmonad\Sigma\otimes\tmonad\Gamma$. The copying constant is 2. 
    \begin{align*}
    \varphi_1(x)=\neg [\Root{\Sigma\otimes\Gamma}(x)\vee (\Port{\Sigma\otimes\Gamma}(x)\wedge \exists y. y=succ_{<_{\tmonad(\Sigma\otimes\Gamma)}}(x) \wedge \Root{\Sigma\otimes\Gamma}(y))]\\
    \varphi_2(x)=\exists r, y. \Root{\tmonad(\Sigma\otimes\Gamma)}(r) \wedge y=succ_{<_{\tmonad(\Sigma\otimes\Gamma)}}(r) \wedge x= succ_{<_{\tmonad(\Sigma\otimes\Gamma)}}(x) 
    \end{align*}
    In the first copy, all the relations of $\underline{\Sigma}$ and $\underline{\Gamma}$ are interpreted in the same way. 
    Let us discuss the relation of $\tmonad\Sigma$ and $\tmonad\Gamma$.
    \[\varphi^{1,1}_{\tmonad\Sigma}(x,y)=\] 
    \end{enumerate} 
    
     \end{proof}