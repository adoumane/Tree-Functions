
\section{First-order term functions can be described in logic}
\label{sec:to-logic}
We show in this section that first-order tree-to-tree functions can be implemented by first-order transductions.

\begin{proposition}\label{prop:to-logic} If $f : \rSigma \to \rGamma$ can be derived from the atomic functions using combinators, then there is a first-order transduction $g$ 
    which makes the following diagram commute
    \begin{align*}
        \xymatrix@C=3cm{
            \rSigma \ar[d]_{a \mapsto \underline a}\ar[r]^f &  \rGamma \ar[d]^{a \mapsto \underline a} \\
            \text{relational structures over $\voctype \rSigma$} \ar[r]_g &  \text{relational structures over $\voctype \rGamma$}.
        } 
    \end{align*}    
\end{proposition}
    
    \begin{proof}
    The proof is by induction, following the definition of first-order tree-to-tree functions. The basic functions with finite domain, projections, co-projections and distribution laws are easily definable by first-order transductions.  
    
    In the following, it will be convenient to use, as part of the vocabulary $\voctype \rSigma$ of every type $\rSigma$, the unary relation  $\mathsf{Port}_\Sigma$ which selects the ports of the structures over $\voctype\rSigma$; and the binary relation $\sqsubseteq_\Sigma$ which orders these ports. By induction on $\rSigma$, both relations can be defined by first-order formulas over  $\voctype \rSigma$.
    
    \begin{enumerate}
    \item $\unit_\rSigma$. Given an element $x$ of $\Sigma$, let us show how the fucntion $\unit_\rSigma(x)$ can be implemented using an fo transduction.  The copying constant is 2,
    the first copy will contain the whole structure $\underline{x}$ and the second copy will select only the ports of $\underline{x}$ which will serve as the ports of the structure $\underline{\unit_\rSigma(x)}$.  The universe formulas are then:
    \begin{align*}
    \varphi_1(x)=\mathsf{true} \qquad \varphi_2(x)=\mathsf{Port}_\Sigma(x)
    \end{align*}
    In the first copy, the vocabulary $\voctype \rSigma$ will be interpreted as in the original structure, and as the empty set in the second one. That is, for every unary relation $R\in \voctype \rSigma$ and for every binary relation $S \in \voctype \rSigma$, we set:
    \begin{align*}
   \varphi_R^{1}(x)=R(x) \quad&\quad \varphi_S^{1,1}(x,y)=S(x,y)\\
   \varphi_R^{2}(x)=\mathsf{False} \quad&\quad \varphi_S^{2,2}(x,y)=\mathsf{False}
\end{align*}      
Let us interpret the relations $<$ and $\sqsubseteq$ of $\voctype \tmonad\rSigma$. The  ports of $\underline{\unit_\rSigma(x)}$ inherit the order of the ports of $\underline{x}$, this is why we set:
\begin{align*}
\varphi_\sqsubseteq^{2,2}(x,y)=x\sqsubseteq_\Sigma y
\end{align*}
The descendent relation $<$ connects the $i^{th}$ port of $\underline{x}$ to the $i^{th}$ port of $\underline{\unit_\rSigma(x)}$. Since these nodes come from the same node in the original structure, we set:
\begin{align*}
\varphi_<^{1,2}(x,y)=x=y
\end{align*}
    \end{enumerate} 
    
     \end{proof}