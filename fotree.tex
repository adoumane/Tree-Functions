
\section{First-order tree functions}

\begin{definition}[Types] The \emph{atomic types} are:
    \begin{itemize}
        \item every ranked set with finitely many elements;
        \item a ranked set, call it  $1$, which has one element on every arity;
    \end{itemize}
    A type is any ranked set obtained from atomic types and applying the constructors $+$, $\times$, $\otimes$ and $\trees$.
\end{definition}
    




\begin{definition}[Atomic functions]\label{def:atomic-functions}
    Let $\tau,\sigma,\tau_0$ and $\tau_1$ be ranked sets. The following functions are called \emph{atomic functions}. 
\begin{enumerate}
    \item The unique  function $! : \sigma \to 1$
    \item Every arity-preserving function with finite domain
    \begin{align*}
        f : \tau \to \sigma
    \end{align*}
    \item Projection and co-projection
    \begin{align*}
        \pi_i : \tau_0 \times \tau_1 \to \tau_i \qquad \iota_i : \tau_i \to \tau_0 + \tau_1
    \end{align*}
    \item Distribute
\begin{align*}
    (\tau_0 + \tau_1 ) \times \sigma \to (\tau_0 \times \sigma) + (\tau_1 \times \sigma)
\end{align*}
\item For every finite ranked set $\sigma$, the bijection 
 \begin{align*}
    \trees \sigma  \to  \varnames + \coprod_{a \in \sigma} \underbrace{\trees \sigma \otimes \cdots \otimes \trees \sigma}_{\text{arity of $a$ times}}
\end{align*}
and its inverse.

% \item Tree construction: for every $\sigma$ and every $a \in \sigma$  a function
% \begin{align*}
%     \underbrace{\trees \sigma \otimes \cdots \otimes \trees \sigma}_{\text{arity of $a$ times}} \to \trees \sigma
% \end{align*}

% \item Tree deconstruction: for every $\sigma$ and every $a \in \sigma$  a function
% \begin{align*}
%       \trees \sigma \to 1 + \underbrace{\trees \sigma \otimes \cdots \otimes \trees \sigma}_{\text{arity of $a$ times}}
% \end{align*}
\item The block function
    \begin{align*}
        \trees (\sigma + \tau) \to \trees (\trees \sigma + \trees \tau)
    \end{align*}
    \item For every $\sigma$ with one binary and one nullary letter, the comb function 
    \begin{align*}
        \trees \sigma \to \trees \sigma
    \end{align*}
    
% \item The port-order function
% \begin{align*}
% \trees (\sigma + \tau) \to \trees (\sigma + \tau)
% \end{align*}

    % \item For every variables $x,y \in \varnames$ (mabye just $x=1$ and $y=2$) the function 
    % \begin{align*}
    %     swap : \trees \tau \to \trees \tau
    % \end{align*}
    
    % \item Some kind of swapping (maybe not needed)
        \item Let $\circ$ be a ranked set with one element of arity $\emptyset$.
    \begin{align*}
        \trees(\sigma+\tau) \to \trees(\sigma+\tau+\circ) \otimes \trees(\sigma+\tau+\circ)
    \end{align*}

\end{enumerate}
\end{definition}



\begin{definition}
    [Combinators] \label{def:combinators} \ 
    \begin{enumerate}
    \item Function composition
    \begin{align*}
    \frac{f : \tau \to \sigma \quad g : \sigma \to \theta} {g \circ f : \tau \to \theta}
\end{align*}

\item Lifting functions to trees
\begin{align*}
    \frac{f : \tau \to \sigma} {\trees f : \trees \tau \to \trees \sigma}
\end{align*}
\item Cases
\begin{align*}
    \frac{f_0 : \tau_0 \to \sigma \quad f_1 : \tau_1 \to \sigma} {[f_0,f_1] : \tau_0 + \tau_1 \to \sigma}
\end{align*}

\item Pairing functions
\begin{align*}
    \frac{f_0 : \tau \to \sigma_0 \quad f_1 : \tau \to \sigma_1} {(f_0,f_1) : \tau \to \sigma_0 \times \sigma_1}
\end{align*}

\item Tensor product of functions
\begin{align*}
    \frac{f_0 : \tau_0 \to \sigma_0 \quad f_1 : \tau_1 \to \sigma_1} {\langle f_0,f_1 \rangle : \tau_0 \otimes \tau_1 \to \sigma_0 \otimes \sigma_1}
\end{align*}
\end{enumerate}
\end{definition}

\begin{definition}[First-order tree functions] \label{def:fo-tree-functions} \ 
    \begin{itemize}
        \item The class of \emph{first-order term functions} is the smallest class of functions which contains the atomic functions from Definition~\ref{def:atomic-functions} and is closed under the combinators from Definition~\ref{def:combinators}.
        \item  A \emph{first-order tree function} is a first-order term function restricted to trees (i.e.~terms of arity $\emptyset$).
    \end{itemize}    
\end{definition}


We are now ready to state the main result of this paper. 
\begin{theorem}\label{thm:main}
    The following classes of functions are equal\begin{itemize}
        \item First-order tree-to-tree transductions;
        \item First-order tree functions.
    \end{itemize}
\end{theorem}
